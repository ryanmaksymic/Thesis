\chapter{Preliminary Research}

One of my first goals was to get a sense of modern concertgoers' and performers' feelings about participatory performances and interactive technology. I sent out a brief online survey for music fans that helped me to understand how they generally responded to these topics. Interviews were also conducted with multiple musicians to shed light on their perspectives.

\section{Audiences}

% Explain the content of the survey:
An online survey was created in order to obtain a sample of modern music fans' opinions on interactive performances. The survey was completed by ninety-nine participants recruited via social media. The first few questions informed me of what type of concertgoer each participant was -- asking their favourite genre, the size of the venues they frequent, and how often they attend live music performances. I also asked how often the participants communicate directly with musicians through their social media presences. Next, the survey focused on concert behaviours. Participants were asked in which actions they typically partake at live music performances; choices included applauding, headbanging, and holding up lighters. They were asked how they might like to interact with their favourite performer and what sort of message they would send them if they could. I asked for their thoughts on getting involved in performances, bringing new technologies into concert settings, and interacting with musicians using social media services. (For complete results, see Appendix X.)

% Present the direct results:
The results were not shocking but certainly informative. Most participants favoured rock music or some variation (``indie," ``alternative"); the majority attended multiple concerts per year -- some even on a weekly basis; and most usually went to shows at small- to medium-sized venues. The majority of participants claimed to communicate with artists through social media either sometimes or regularly, though a sizeable amount indicated they never do this. The most popular concert actions were applauding, singing along with the performer, clapping or stomping to the beat, dancing, jumping up and down, and chanting words or phrases along with the other audience members. When asked how they might want to interact with a performance, many said they would like to choose the songs that are played, while much fewer expressed interest in manipulating visuals and contributing to the music; around one quarter of participants stated they did not have interest in directly influencing a live music performance at all. Given the opportunity to communicate with their favourite performer, most participants responded with praise or appreciation (``Thank you," ``I love you!"). Other messages included song requests and suggestions like, ``Don't bury the vocals," or ``More rock, less talk!" The majority of participants indicated that they enjoy when performers ask them to participate in a performance -- clapping or singing along or call and response, for example. Lastly, the majority also said they were excited by the idea of bringing new technologies into a live music setting.

% Present the results of the in-depth analysis:
Upon further analysis of the responses, some correlations were uncovered. There are clear relationships between show-going frequency, venue size, and interest in interaction and technology. Participants attending shows more frequently are more likely to visit smaller venues. This group also expressed the most interest in being involved in performances; they are more inclined to interact with their favourite artists via social media; and they are more welcoming to the idea of unfamiliar technology in a concert setting. The opposite, thus, can also be said: participants who go to fewer shows tend to go to larger venues, are more likely to refrain from participating in shows, are less likely to contact artists through social media, and are less interested in new technologies.

% Conclusions:
A few general conclusions can be made from these results that are particularly relevant to my research question. It is encouraging to confirm that most participants are not quietly standing still at live performances; they are cheering, moving, and singing along. A surprising find was that, given the chance to say anything to their favourite artist on stage, most participants would choose simple messages of praise or thanks -- something ostensibly achieved already by applauding. Also intriguing was the relative lack of interest in influencing lights and visualizations. Instead, the majority of participants showed great interest in choosing the set list for the performance. Regardless, it is clear that most respondents have little to no reservations about being directly involved in a show and doing so with new technologies. Seemingly, this willingness to interact is more common in those who frequently attend performances at smaller venues. Perhaps, then, artists that play to smaller crowds and can offer more direct interactions both on and off stage have fan bases that are more willing to experiment with new interactions.


\section{Performers}

% Explain the interview participants and questions:
With a broad overview of audience attitudes, the next step was to speak directly with actual performers and see how their opinions compared. Four musicians -- Christian, Erik, Blake, and Oliver -- were interviewed, all members of different bands that perform some variation of rock or pop music, with typical audience sizes ranging from fifty to five hundred people. After briefly establishing their history as performers, I asked each about how they like to interact with their fans. The musicians were shown video of some of my case study subjects (including Xylobands, Wham City Lights, Kasabian), and I asked for their reactions and general thoughts on technology-enabled performances. Lastly, the artists were asked how they might want to incorporate participatory technologies into their own shows.
% Explain how and why these participants were chosen

% Christian:
Christian is the frontman for a Toronto-based new wave group that has been active for several years. He comes from a theatre background but has been playing in bands since he was fourteen. Christian was performing acoustic music during his post-secondary degree, but he was growing increasingly fond of dance music, and he eventually formed a duo that made use of computer software to play danceable pop music.
