\chapter{Preliminary Research}

One of my first goals was to get a sense of modern concertgoers' and performers' feelings about participatory performances and interactive technology. First, I sent out a brief online survey to music fans that helped me to understand how they generally responded to these topics. I then conducted followup interviews with a few of the participants and discussed their opinions in greater detail. Lastly, interviews were also conducted with multiple musicians to shed light on their perspectives.

\section{Audiences}

An online survey was created in order to obtain a sample of modern music fans' opinions on interactive performances. The survey was completed by 99 participants recruited via social media. The first few questions informed me of what type of concertgoer each participant was -- asking their favourite genre, the size of the venues they frequent, and how often they attend live music performances. I also asked how often the participants communicate directly with musicians through their social media presences. Next, the survey focused on concert behaviours. Participants were asked in which actions they typically partake at live music performances; choices included applauding, headbanging, and holding up lighters. They were asked how they might like to interact with their favourite performer and what sort of message they would send them if they could. I asked for their thoughts on getting involved in performances, bringing new technologies into concert settings, and interacting with musicians using social media services.

The results were equally anticipated and surprising. Most participants favoured rock music or some variation (``indie," ``alternative"); the majority attended multiple concerts per year -- some even on a weekly basis; and most usually went to shows at small- to medium-sized venues. The majority of participants claimed to communicate with artists through social media either sometimes or regularly, though a notable amount indicated they never do this. The most popular concert actions were applauding, singing along with the performer, clapping or stomping to the beat, dancing, jumping up and down, and chanting words or phrases with the other audience members. When asked how they might want to interact with a performance, many said they would like to choose the songs that are played, while fewer expressed interest in manipulating visuals and contributing to the music; around one quarter of participants stated they did not have interest in directly influencing a live music performance. Given the opportunity to communicate with their favourite performer, most participants responded with praise or appreciation (``Thank you," ``I love you!"). Other messages included song requests and suggestions like, ``Don't bury the vocals," or ``More rock, less talk!" The majority of participants indicated that they enjoy when performers ask them to participate in a performance -- clapping or singing along or call and response, for example. Most also said they were excited by the idea of bringing new technologies into a live music setting.

Analyzing the responses further revealed some other trends... 

\section{Performers}