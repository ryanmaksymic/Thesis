\chapter{Research Question}

This chapter will provide an in-depth examination of my research question. Through referencing the existing research outlined in the previous section, I will justify the need for my work. I will then explain the steps I will be taking to address this question.

\section{Research Question}

% Questions:
How can technology best be used to add participatory elements to traditionally presentational rock and pop concerts?
% * How do audiences want to participate? How do performers want audiences to participate?
% * Input: How can data best be captured from the audience?
% * Output: What can the audience interact with that will enhance the performance but not distract the audience or performer?
% * Feedback: What kind of feedback will be informative/useful without being distracting?
% * What's the threshold where a performance can no longer be social? What's the ratio where participation can be sustained?
% * Many maintain that music is inherently social, so is anyone explicitly calling out presentational performances as unnatural? Should I do this?
% * Small shows are much, much cheaper than arena shows, and they offer much more personal experience. So why are we paying \$500 for arena shows? Are we getting our money's worth? Does the spectacle make up for the separation?

% Justification: 
% * Interactivity at rock concerts has not been thoroughly examined, despite the fact that people are making these systems in the real world
% * The performer's perspective is rarely investigated, and this is just as important as the audience's perspective
% * The impact of social media on live performance has not been analyzed in this way


\section{Hypothesis}

% Background:
% Auslander, 1999: Communities are formed based on how the audience interacts, with no dependence on the spectacle at hand
% Turino, 2008:
% * Artists can shift between participatory and presentational performance
% * Different roles of different difficult allow for everyone to feel welcome and achieve flow. "Core" and "elaboration" roles cater to advanced and non-advanced performers respectively.
% * Open form: Basic motives repeated over and over. Easy for newcomers to join in. "Security in constancy." Can facilitate flow.
% * Hall: Repetition can increase intensity. Synchronicity comforts people.
% * Wide tuning, loud volumes, and overlapping textures provide a ``cloaking function'' that makes people more comfortable participating
% * Virtuosic solos are not common
% * Some participatory performances are sequential -- everyone gets a turn (e.g. Karaoke)


% Hypotheses:
% * 


\section{Approach}

% Methods:
% * Questionnaire and interviews (User-centred design; Ethnography)
% * Prototyping and user testing (User-centred design)
% * Real-world testing (Action research, participative objective observation; Ethnography)