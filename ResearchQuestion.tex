\chapter{Research Question}

This chapter will provide an in-depth examination of my research question. Through referencing the existing research outlined in the previous section, I will justify the need for my work. I will then explain the steps I will be taking to address this question.

\section{Research Question}

% Questions:
How might technology be used to add participatory elements to traditionally presentational rock and pop concerts?
% * How do audiences want to participate? How do performers want audiences to participate?
% * Input: How can data best be captured from the audience?
% * Output: What can the audience interact with that will enhance the performance but not distract the audience or performer?
% * Feedback: What kind of feedback will be informative/useful without being distracting?
% * What's the threshold where a performance can no longer be social? What's the ratio where participation can be sustained?

% Justification: 
% * Interactivity at rock concerts has not been thoroughly examined, despite the fact that people are making these systems in the real world
% * The performer's perspective is rarely investigated, and this is just as important as the audience's perspective
% * The impact of social media on live performance has not been analyzed in this way


\section{Hypothesis}

% Background:
% Auslander, 1999: Communities are formed based on how the audience interacts, with no dependence on the spectacle at hand
% Turino, 2008:
% * Artists can shift between participatory and presentational performance
% * Different roles of different difficult allow for everyone to feel welcome and achieve flow. "Core" and "elaboration" roles cater to advanced and non-advanced performers respectively.
% * Open form: Basic motives repeated over and over. Easy for newcomers to join in. "Security in constancy." Can facilitate flow.
% * Hall: Repetition can increase intensity. Synchronicity comforts people.
% * Wide tuning, loud volumes, and overlapping textures provide a ``cloaking function'' that makes people more comfortable participating
% * Virtuosic solos are not common
% * Some participatory performances are sequential -- everyone gets a turn (e.g. Karaoke)
% Kelly, 2007: Displaying clips and themes from her music videos at a Madonna concert creates feelings of a "shared past" in the audience. (How might we create instead a "shared present"?)
% Small, 1998: Performers dressing in uniform are separating themselves and their responsibilities
% Davidson, 1997:
% * Performance etiquette is usually formed by crowd mentality, following the majority
% * Performers pick up information from the audience's broad and specific behaviours
% * Visuals help audiences read the performer's intentions
% Sexton, 2007: Simple synchronous interactions in sound art projects left users with little to explore, resulting in a "flat" experience
% Jourdain, 1997: We move to music in order to "represent" it. This also amplifies, resonates the musical experience.
% Levitin, 2006:
% * "In every society of which we're aware, music and dance are inseparable." Ancient music was based on rhythm and movement. Combining rhythm and melody bridges our cerebellum and cerebral cortex.
% * Ties between music and movement have only been minimized in the last 100 years
% Kelly, 2007: Technology incorporated into a show can either be addressed as part of the show or hidden and made illusory
% Maynes-Aminzade, 2002:
% * Computer vision: Movement-based control were intuitive, but camera required frequent calibration
% * Beach ball: Using a single beach ball as an input was also intuitive, but it only involved a few people at a time
% * Laser pointers: Gave everyone an individual cursor, but got chaotic once more and more people joined
% * Recommendations: Focus on compelling activity over impressive technology. Not everyone must be sensed as long as they feel involved. The control mechanism should be obvious or the users will give up. Make the activity emotionally engaging. Emphasize cooperation.
% Ulyate, 2001:
% * Design guidelines:
% * * Encourage and reward movement
% * * Feedback should be immediate, obvious, and meaningful in the context of the space
% * * No instruction or thinking should be required
% * * Responsiveness is more important than aesthetics
% * *  Modularity is key
% * Lessons learned:
% * * Full-body movement is most satisfying
% * * Form of the object determines how users interact
% * * A practical system is distributed and scalable
% * * Find balance between freedom and constraint
% * * Users will always find a way to create unwanted outputs
% * * Simple, instant gratification is important for feedback
% Barkhuus, 2008:
% * Inputs based on already-present behaviour lead to intuitive systems
% * Don't focus on employing cutting edge technology
% * Events should not rely on the success of the technology
% * Immediate visual or aural feedback is key
% Tseng, 2012: Being excluded from the interaction did not lessen enjoyment of the show
% Reeves, 2010:
% * "Intra-crowd interaction" is a common phenomena to exploit
% * Many actions "snowball" and overtake crowds; highly visible/audible actions promote this
% * People on the fringes of the crowd interact, but there is latency
% * Every crowd is different; designs should reflect the environment
% Gates, 2006: Technologies should reflect the performer's art and not be a burden on them

% Input:
% * 

% Output:
% * 

% Feedback:
% * 


\section{Approach}

% Methods:
% * Questionnaire and interviews (User-centred design; Ethnography). How do artists interact with fans now -- onstage and offstage? What do artists and fans think of interactive technologies? How do artists and fans want to interact at shows?
% * Prototyping and user testing (User-centred design)
% * Real-world testing (Action research, participative objective observation; Ethnography)

% Justification:
% * Why are these useful methods for my project?