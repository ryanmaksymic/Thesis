\chapter{Research Question}

This chapter will provide an in-depth examination of my research question. Through referencing the existing research outlined in the previous section, I will justify the need for my work. I will then explain the steps I will be taking to address this question.

\section{Research Question}

* How might rock performances be made more participatory through the amplification of audience actions?\\
** Sub-questions:\\
- How do audience members want to participate? How do performers want their audience members to participate? How might a system be designed that enhances the concert experience for both parties?\\
- How can an audience best provide inputs for a system? How can the system best provide feedback to the audience?\\
** Provide my definitions for `participatory,' `enhance,' `feedback,' etc...\\
% Questions:
% * How might technology be used to add participatory elements to traditionally presentational rock concerts?
% * How do audiences want to participate? How do performers want audiences to participate?
% * Input: How can data best be captured from the audience?
% * Output: What can the audience interact with that will enhance the performance but not distract the audience or performer?
% * Feedback: What kind of feedback will be informative/useful without being distracting?
% * What's the threshold where a performance can no longer be social? What's the ratio where participation can be sustained?

* Justification:\\
- Music is participatory by nature, so it is worth examining how we might stop this characteristic from being suppressed in modern-day performance.\\
- The case studies in the previous section prove there is interest in creating new participatory experiences at live shows. Observing them from an academic perspective will offer insight for similar projects in the future. Furthermore, the live music industry continues to grow; it is worth investing how they might evolve.\\
- Most of the research done on crowd interaction does not feature an active performer as the center of attention. Designing an interactive system for a rock show is bound to present different challenges than a dance club, for instance.
% Justification: 
% * Music is participatory by nature, so we should think about how to get modern performances back in touch with that
% * There is interest in creating these new participatory concert experiences
% * The live music industry is growing all the time
% * Most HCI research on crowd interfaces don't have performers in the mix
% * The performer's perspective is rarely investigated, and this is just as important as the audience's perspective (?)
% * The impact of social media on live performance has not been analyzed in this way (?)


\section{Approach}

* I will be performing qualitative research to address my research questions.\\
** Ethnography: Examining cultural phenomena within a group. Surveying modern music fans will provide me with a general sense of their feelings towards music, live performance, and technology. I will interview active musicians to understand how and why they interact with fans, on and off stage, and what they think of new technologies in a performance setting. These methods will allow me to gain a deeper understand the users and environments that are related to my work.\\
** User-centred design: The users' wants and needs drive the design. Prototypes will be tested with users. Observations and interviews with the participants will help answer research questions as well as validate or disprove any assumptions that have been made. Both audience and performer perspectives will be represented. This method will help establish design guidelines.

% Methods:
% * Questionnaire and interviews (User-centred design; Ethnography). How do artists interact with fans now -- onstage and offstage? What do artists and fans think of interactive technologies? How do artists and fans want to interact at shows?
% * Prototyping and user testing (User-centred design)
% * Real-world testing (Action research, participative objective observation; Ethnography)

% Justification:
% * Why are these useful methods for my project?