\chapter{Literature Review}

This chapter presents the historical significance of this research and outlines related work that has been completed by both researchers and performers.

\section{Background}

In this section, music's significance as a social activity among humans is investigated. The concepts of participatory and presentational performances are explained and contrasted. I discuss the rise of presentational performance and how it turned into a visual spectacle. Lastly, I consider rock music and detail the emerging participatory culture.
% Don't suggest anything in this chapter. Move any hypotheses to the Introduction.

\subsection{Music and Community}

% Music is a part of being human:
Music is a part of being human. ``The archaeological record," Daniel Levitin (2006) explains, ``shows an uninterrupted record of music making everywhere we find humans, and in every era" (p. 256). Early music making was purely rhythmic, with simple objects being used as percussion instruments. As primitive wind and string instruments were crafted, rhythm was joined by melody. Music making gradually evolved in cultures all over the world, and it has grown to serve many different purposes. In New Guinea it is a gift to one's host; in the Democratic Republic of the Congo it is used to settle lawsuits; Australian aboriginals use music to tell intricate personal stories; and some African tribes believe repeating musical chants can draw harmful spirits out of inflicted individuals (Jourdain, 1997; Turner, 2011). In modern Western cultures, music is everywhere: performed in stadiums or on street corners, scoring film and television, and being shared on the Internet.
% Levitin, 2006: Archaeologists have uninterrupted records of music everywhere and every era humans existed
% Jourdain, 1997: Different cultures -- e.g. New Guinea, the Congo, Australian aboriginals -- have different uses for music -- e.g. as a gift, to settle lawsuits, to tell intricate stories
% Turner, 2011: Repetitive, hypnotic mantras are used by African tribes to draw out harmful spirits

% Music as a social activity:
How did music become such a significant human activity? Anthropologists believe it may have initially been a tool for social bonding, or perhaps a clever survival method (Jourdain, 1997; Levitin, 2006). A drum circle around a fire could have improved a group's coordination, but it may have also served to keep everyone awake and ward off predators. While music may be primarily a source of pleasure and entertainment in Western societies today, it still serves valuable social functions. Musicologist Thomas Turino (2008) talks about the benefits of making music with others -- what he calls ``sonic bonding." Referencing the work of anthropologist Gregory Bateson, Turino posits that such artistic experiences promote deep connections to others that are crucial for ``social survival." Thus, music making may be a strong tool in forming and developing rewarding relationships. Christopher Small (1998) suggests that live music performance forms communities that represent ideal relationships; participants can momentarily forget reality and feel one with those around them. Anthropologist Edith Turner echoes this idea, referring to a phenomenon called ``communitas." As she explains, ``Communitas occurs through the readiness of the people -- perhaps from necessity -- to rid themselves of their concern for status and dependence on structures, and see their fellows as they are" (p. 1). Turner identifies communitas at work at sporting events, in the workplace, and even during disasters, but she claims music to be the most reliable source. Music is ephemeral, emotional, and it cannot be constrained by rules. ``Its life is synonymous with communism, which will spread to all participants and audiences when they get caught up in it" (p. 43). Though we cannot share our bodies with one another, Turner explains, music allows us to share time. It is clear that live music performances can be powerful events; it is important to realize, however, that not all performances are the same.
% Mention that it obviously brings pleasure
% Jourdain, 1997: Anthropologists believe music evolved "to strengthen community bonds and resolve conflicts"
% Levitin, 2006:
% * Why did music evolve? Social bonding and coordination, survival methods.
% * A cluster of genes may control both musicality and outgoingness
% Auslander, 1999:
% * Communities are formed based on how the audience interacts, with no dependence on the spectacle at hand
% * Removing the gap between audience and performer stops it from being a performance. Striving for unity only emphasizes the gap. (I disagree.)
% Small, 1998:
% * Community formed at concerts represent ideals, not realities
% * At a live music performance, the values of a certain social group are "explored, affirmed, and celebrated." Meaning comes from the relationships between the people and the venue, among those taking part, and between the sounds being made.
% Turino, 2008:
% * Making music together leads to "sonic bonding"
% * Connecting with others through art is ``crucial for social and ecological survival'' (Bateson)
% * Music and dance promote "flow" -- optimal experience (Csikszentmihalyi)
% * Music making or dancing is a realization of ideal human relationships
% Turner, 2011:
% * Communitas arrises from people ridding themselves of concern for status and structure, readiness to ``see their fellows as they are''
% * The communitas spirit does not take sides, yet it can be ``prostituted'' to produce prejudice
% * A communitas event (here, Carnaval) is not ordinary, it is brief, yet it is vividly remembered
% * ``Music is a fail-safe bearer of communitas''. It is ephemeral, emotional, it breaks the rules. "Its life is synonymous with communism, which will spread to all participants and audiences when they get caught up in it."
% * We cannot share our bodies, but music allows us to share ``precise time''
% * Flow: In control of the situation, yet fully absorbed "so that it seems to be in control of us." A loss of ego. Related to communitas.

\subsection{Participatory and Presentational Performance}

% Participatory performances:
In his book \textit{Music as Social Life: The Politics of Participation} (2008), Thomas Turino divides music performance into two categories -- participatory and presentational. Most cultures exhibit some sort of participatory performance. Peruvian communities perform in large groups with each participant either dancing or playing a panpipe or flute; many different religious ceremonies involve singing in unison or in a call-and-response structure; line dancing in North America features choreography that directly responds to the music. In general, the emphasis is on the intensity of the interactions over the quality of the performance, and participatory performances have characteristics that support this. In a purely participatory performance, Turino explains, ``there are no artist-audience distinctions, only participants and potential participants performing different roles, and the primary goal is to involve the maximum number of people in some performance role" (p. 26). Equality among participants can lower self-consciousness and lead to a more relaxed atmosphere. Having slightly different roles, on the other hand, allows individuals of different skill levels to contribute accordingly. ``Core" and ``elaboration" roles, in Turino's words, cater to less- and more-advanced performers, respectively; core participants keep the performance moving along while elaboration participants add flourish. Another common feature of participatory performances is repetitiveness. This open form allows newcomer participants to easily join in. Additionally, Turino explains, there is a ``security in constancy" that allows performers to become more comfortably immersed in the music. Performances may also incorporate loud volumes, densely overlapped sounds, and ``wide tuning" as ``cloaking functions" to make individuals more comfortable participating. Solos are not common, although sequential soloing sections are sometimes included; karaoke is an example of sequential participatory performance. Overall, participatory performances allow all participants to feel as though they are contributing, and this makes them quite different from presentational performances.
% Turino, 2008:
% * Artists can shift between participatory and presentational performance
% * No artist/audience distinction, only participants possibly performing different roles. Emphasis on sonic and physical interaction. "Heightened social interaction." The "doing" is more important than the end result.
% * Cultural background will determine if a person is comfortable joining a participatory performance
% * Leads to diminished self-consciousness since everyone is equally contributing
% * Different roles of different difficult allow for everyone to feel welcome and achieve flow. "Core" and "elaboration" roles cater to advanced and non-advanced performers respectively.
% * Intensity of participation comes before quality of performance. Inept performers are noticed but not addressed.
% * Participatory performance does not fit well in a capitalist society. It yields social value, not financial value.
% * Open form: Basic motives repeated over and over. Easy for newcomers to join in. "Security in constancy." Can facilitate flow.
% * Hall: Repetition can increase intensity. Synchronicity comforts people. However, Turino believes repetition can "lead to boredom in presentational contexts... Participatory musical styles do not transfer well to presentational stage situations, in spite of nationalists', folklorists', and academics' attempts to bring them into presentational settings." (How do I feel about this?)
% * Wide tuning, loud volumes, and overlapping textures provide a ``cloaking function" that makes people more comfortable participating
% * Virtuosic solos are not common
% * Some participatory performances are sequential -- everyone gets a turn (e.g. Karaoke)
% * Participatory performance may constrain creativity, but it benefits more people overall
% * The doing of activities as a group makes participatory events more engaging than performative

% Presentational performances:
In a presentational performance, the performer presents prepared music pieces to an audience that does not directly participate in the performance (Turino, 2008). A performance by an orchestra is a good example: professional musicians on a stage perform composed and rehearsed songs to an attentive audience whose role is listening to the music. In contrast with the open form of participatory performances, here there is a focus on detail, smoothness, and coherence. These performances are generally closed form; a performer knows how the show will begin and end. While participatory performances rely on constancy, planned contrasts are implemented in presentational performances in order to keep the audience's attention. A rock band will often break up a song with a guitar solo, for instance. Participatory performances foster connections between participants, Turino explains, whereas presentational performances seem to tease a connection between artist and audience without ever realizing it: `Leave them wanting more.' Indeed, the goal of a presentational performance is typically to sell as many tickets as possible, and this productization of music has been present since presentational performances began gaining popularity just centuries ago.
% Make this last sentence less of a bold assertion
% Turino, 2008:
% * The performer prepares and presents music to the audience, who does not participate in the performance
% * Closed form. Predetermined beginning, middle, and end.
% * Presentational musicians aim for detail, smoothness, coherence. Predictability and control are key. Some participatory-leaning performers may make adjustments to accommodate the audience.
% * Planned contrasts are implemented in performances to keep the audience interested. This would derail participatory performance.
% * Attractive because it partially fulfills desire to connect to an artist. Corbett suggests concerts are designed to tease but not consummate this connection, "leaving consumers wanting more."

\subsection{Presentational Performance, Technology, and Visual Culture}

% Presentational performances bred from capitalism:
Public concerts were virtually unheard of before the 1600s (Jourdain, 1997). Outside of church, commoners rarely had the opportunity to hear ``serious" music, and any other music performance was relaxed and participatory in nature. The `professional' musicians of this era, musicologist Christopher Small (1998) explains, were those hired by aristocrats to accompany them as they played. It was not until the time of the Industrial Revolution that savvy musicians realized that the newly established middle class would regard live performances as opportunities to display their newfound wealth. These ``traveling virtuoso-entrepreneurs" made money touring from town to town and performing in local parlours for a fee. By the 1800s, ticketed concerts were gradually becoming more abundant and began transforming the state of the live performance. These were seen as formal events, says Jourdain, and those wealthy enough to attend were expected to follow the established etiquette -- sitting quietly and listening. Presentational performance thus emerged as a product to be enjoyed by those who could afford it. Over time, it has become increasingly important that they provide as much value for patrons as possible.
% Levitin, 2006: Music has only been a spectator activity in the past 500 years
% Jourdain, 1997:
% * Public concerts were virtually unheard of until the 1600s
% * Only after the industrial revolution in the 1800s did common folk gain disposable income and public concerts gain popularity
% * At early operas people talked, ate, and played cards. Even performers would "jabber away" and speak to audience members.
% * Audiences gradually became annoyances to performers. Some conductors would lock out latecomers, silence applause, and ignore requests.
% * Near the end of the 1800s, "art music" performances became more modern -- formal and quiet
% * Some critics call symphony orchestra performances "the epitome of capitalist oppression" -- hierarchical, no audience involvement
% Small, 1998:
% * Musicians were first hired to play with aristocrats. "Traveling virtuoso-entrepreneurs" first started playing public performances when they realized the desire of the middle class to display their wealth.
% * Private performances surfaced around 1730. Ticketed events were not abundant until the 1800s.

% Technology is used to enhance presentational performances:
Presentational performances have been enhanced in many ways using technology. Arena rock concerts, for instance, make use of arrays of powerful speakers, dense lighting rigs, and multiple giant screens, sometimes also incorporating huge stage pieces and complex mechatronics. The main function of this equipment is to amplify the sights and sounds of the performance. As Kelly (2007) explains, however, it also serves to amplify the persona of the performer. Large screens may show closeups of the performers on stage, but they also often display video clips or abstract visuals designed to reflect the performer's image and communicate underlying themes. Similar technologies are being used in even more abstract ways by groups like Gorillaz -- an alternative rock band fronted by fictional characters. Live performances often feature real musicians silhouetted by projections of the animated cartoon band members. German electronic band Kraftwerk took it a step further for live performances of their song ``Robots," leaving the stage entirely and being replaced by robotic stand ins for the duration of the song. The fact that audiences will cheer for artificial performers indicates the influence technology has had on live performance. Compelling visual elements are becoming increasingly essential in modern concerts.
% Auslander, 1999: Mediatized performances can offer more intense sensory experiences than traditional performances
% Kelly, 2007:
% * Embodied co-presence of performer and audience is an important convention of pop performances. "Immediacy, intimacy, self-expression and certain kinds of performer-spectator interaction" are expected. The venue becomes "not only a hearing place, but also a seeing place and a being place."
% * "Presence, representation and self-expression" can be either amplified or attenuated by audiovisual technologies in pop shows
% * A 1966 Andy Warhol event (featuring the Velvet Underground and Nico) mixed live performance with projection. Attendees viewed the projections as "part of the show."
% * Imagery framing the performance generally amplifies the performer's presence. This goes against Benjamin's claim that technological intervention will "devalue the here and now of the artwork." Kelly argues that screens showing different perspective in fact amplify the sense of presence. Referring to a Madonna performance, the author states that the juxtaposition in scale between screen and performer actually creates a 'theatrically structured intimacy."
% * Kraftwerk left the stage and were replaced by robots for their song Robots. This "problematises traditional modes of presence in pop-concert reception."
% * Gorillaz is a group fronted by animated characters. They can make a bigger emotional impact than human performers.
% * One Gorillaz performance featured a deceased singer using prerecorded video and audio -- effectively remediating presence. (How can audience presence be remediated?)
% * Displaying clips and themes from her music videos at a Madonna concert creates feelings of a "shared past" in the audience. (How might we create instead a "shared present"?)

% Mass media (MTV) made modern concerts dominated by the visual:
So why are visuals such an important part of today's music performances? Media studies professor Jamie Sexton (2007) points out that music is always tied to other media; things like music videos and even album art can affect the way songs are experienced. ``Musical meaning ... emerges from its relationships with other media," Sexton posits (p. 2). Philip Auslander (1999) provides another perspective. Live performance and mass media are in competition, he says, and, since mass media is dominating, live performance has responded by imitating its competitor. Live sporting events make use of big screens and instant replays, for instance, and television shows and movies are regularly adapted for the stage. Live music performances, similarly, began replicating music videos. As Music Television (MTV) reached the height of its popularity in the 80s and 90s, the music video became the ``the primary musical text," a role previously held by the sound recording. Thus, while concerts used to serve to authenticate an artist's musical ability, they now also had to authenticate their image and charisma as portrayed in their music videos. Jaap Kooijman (2006) points to Michael Jackson's performance of ``Billie Jean" at the 1983 \textit{Motown 25} concert, where Jackson's outfit and dancing directly referenced the song's music video. The crowd cheered as Jackson exhibited the dance moves that they had seen on television -- never noticing, or perhaps never caring, that he was lip-synching the vocals throughout. As television became the primary source for music, live performance had to respond by becoming more visual. 
% Sexton, 2007:
% * Music is always associated with other media -- performer movement, words in programme notes, album art. Meaning emerges from these relationships.
% * The original phonograph distressed listeners with its lack of visuals. Edison tried to add visual elements to the invention.
% Auslander, 1999:
% * Theatre and mass media (TV, film, recorded music) are rivals. They are not inherently this way, but were made this way over time. Live performance's response to this rivalry has to imitate mass media.
% * "The function of live performance under this new arrangement is to authenticate the video" -- TV > performance (in 1999)
% * The live/mediatized rivalry is not equal. Television is dominant (in 1999). (The Internet is dominant now?)
% Koojiman, 2006:
% * After MTV started in 1981, television became the main promotion tool for music
% * At the Motown 25 performance, Michael Jackson lip-synched "Billie Jean." The focus was on visuals, with Jackson performing the moonwalk for the first time. This was during the rise of MTV, and the performance was essentially a reenactment of the music video.
% Burns, 2006: Madonna's performance at the 1984 MTV Video Music Awards was designed to be a music video, and it literally became one

\subsection{Rock Concerts and Participatory Culture}

% Modern performances are no longer participatory:
A concertgoer today is much more a consumer of the performance than a participant in it. David Horn (2000) provides the following definition:

\begin{quotation}
\onehalfspacing	
The popular music event is the sum of a number of smaller occurrences, which might include any or all of the following: the origination or the borrowing of a musical idea; the development of the idea; the conversion or arrangement of the idea into a performable piece; the participation of those (musicians, producers, technicians) whose task is to produce musical sound; the execution or performance of this task; the transmission of the resulting sounds; the hearing of those sounds (p. 28).
\end{quotation}

Out of all of the tasks associated with a live show, here the audience is only given one -- hearing. As Jourdain (1997) explains, this spectatorial role developed as performers began ``dominating" over the audience. In the 1800s, conductors began locking latecomers out of the theatre, silencing applause, and ignoring popular requests. Musicians started dressing in matching uniforms, further setting themselves apart from the audience. Audiences are now further dominated by the booming speakers that drown out their voices and the bright lights that make the performer blind to them. Even the audience's primary tool -- applause -- is losing its power. As Baz Kershaw (2001) explains, standing ovations, once rewards that had to be earned by performers, are today dispensed almost without question; the standing ovation is now ``an orgasm of self-congratulation for money so brilliantly spent," he says, rather than a democratic device (p. 144).
% Webster, 2012: The standing ovation has also lost its power...
% Kelly, 2007: Horn (2000) provides a definition of a pop music performance that belittles the role of the audience
% Jourdain, 1997:
% * Audiences gradually became annoyances to performers. Some conductors would lock out latecomers, silence applause, and ignore requests.
% * Technology has made music less participative. (Is that a challenge?)
% Kelly, 2007: Notions of "identity, authenticity and originality" in pop performance are being challenged. This is allowing for active participation for audience members.
% Small, 1998:
% * Division of talented and non-talented performers is based on a falsehood. We are all capable of creating music.
% * Sharing performance with strangers in the West is abnormal. Most societies' performance feature members of the community.
% * Classical music performance turned audience members into spectators, consumers
% * Audience separated from, "dominated" by performer is a social feature of modern performance. Perhaps a desire to make the perform more "mysterious."
% * 60s/70s folk fests cultivated community, but it was an orchestrated "illusion." Constraints are always necessary.
% * Performers dressing in uniform are separating themselves and their responsibilities
% * Modern orchestra concerts emphasize listening, specialization, formal settings, ticketed admission. They celebrate separation of producer and consumer.
% Turino, 2006:
% * Repetition can "lead to boredom in presentational contexts... Participatory musical styles do not transfer well to presentational stage situations, in spite of nationalists', folklorists', and academics' attempts to bring them into presentational settings." (How do I feel about this?)
% Kershaw, 2001:
% * 20th century London theatre audience have become much more prone to standing ovations than they once were. This may indicate disempowerment and the fall of the theatre as a political environment. Over the 20th century the role of the theatregoer changed from patron to customer. Early audiences would boo or yell approval. After theatre became organized and lavish, a standing ovation became `an orgasm of self-congratulation for money so brilliantly spent... The power of community all but evaporated.`
% * Applause may be an attempt to form a momentary community

% Rock shows encourage participation:
While rock concerts can be extremely presentational, fortunately they can also simultaneously challenge the inequality that has been established between audience and performer; they are presentational performances with participatory leanings. Crowds join the music making by clapping or singing along, for example. They add to the light show by holding up lighters or illuminated cellphones. Some musicians invite fans to shout out song requests. Perhaps the simplest way for an audience member to become a participant is to move to the music: swaying, dancing, or joining the mosh pit. Although some performers are particular about how audiences behave (Neil Young\footnote{\url{http://nme.com/news/neil-young/74774}} or Queens of the Stone Age\footnote{\url{http://pitchfork.com/news/53876-watch-queens-of-the-stone-ages-josh-homme-action-bronson-throw-people-off-stage-this-week}}, for example), others go out of their way to make audience members a part of the performance. Elvis Costello has toured with a ``Wheel of Songs," for instance, inviting audience members to spin the large wheel and determine which tune the band will play next. The Flaming Lips give a handful of audience members ridiculous costumes and let them dance behind the band for entire concerts. Green Day, performing for thousands of fans, will pull a select few on stage and allow them to play the band's instruments for one song before sending them backstage to enjoy the after party. These sorts of interactions between musicians and their fans are becoming more and more ordinary, and they reflect music's increasingly participatory culture.
% Participatory events are usually brief, often reserved for specific portions of certain songs
% Carah, 2010: Iggy Pop was not allowed to invite fans to dance on stage, causing him to vocally object, blaming ``money and TV"
% Some artists are known to take questions from the crowd in between songs. One interviewer mentioned a performer who announced his phone number and responded live to text messages he received.
% There are certainly those who try to draw a line; e.g. Neil Young, Josh Homme. Stage diving has led to serious injury and even death.
% Kelly, 2007:
% * Elvis Costello uses a "Wheel of Songs" that allows spectators to determine the setlist
% * Kraftwerk handed out beeping electronic calculators so the audience could join in during their song Pocket Calculator

% Mass media (the Internet) is changing the audience-performer relationship:
The Internet has changed the face of mass media; in particular, it has drastically altered the relationship between content creators and their fans. Communication studies researcher Nancy Baym (2012) describes the traditional audience-performer relationship as ``parasocial" -- one-sided, with most information flowing from the performer to the audience. The notions of `rock gods' and `pop stars' once framed musicians as untouchable beings. Today, however, increased connectivity on the Internet has brought the two parties closer together; some musicians, says Baym, even see their fans more as friends. A multitude of social media services allows performers and audiences to communicate and even collaborate, creating what Patrik Wikstr\"{o}m (2013) fittingly refers to as a ``participatory culture." Artists directly respond to fan messages on Facebook or Twitter, for example, and organize in-depth question-and-answer sessions with the Reddit community. They invite fans in to their personal lives by posting photos from their daily life on Instagram. Soundcloud\footnote{\url{http://soundcloud.com}} allows users to post comments as they listen to a song, praising or critiquing certain parts of the track. With crowdsourcing services like Kickstarter, artists can ask fans to directly fund their projects; musician Amanda Palmer most famously raised over \$1 million from 24 800 supporters to fund an album, a book, and a tour\footnote{\url{http://kickstarter.com/projects/amandapalmer/amanda-palmer-the-new-record-art-book-and-tour}}. Many artists ask fans to create remixes of their work, and new services like BLEND.IO\footnote{\url{http://blend.io}} facilitate this kind of collaboration. ``Social and creative music use," Wikstr\"{o}m explains, ``is the normal way in which music fans use music in the new economy" (p. 171). While this dynamic is prominent on the Internet, it also has implications in the live setting.
% Jourdain, 1997: Technology has made music less participative. (Is that a challenge?)
% Baym, 2012:
% * Audience-performer relationships were traditionally `parasocial` -- one-sided
% * Social media has reconfigured this relationship, eliminating `rock gods` and `pop stars`
% * Interviewed musicians feel social media presence is necessary for success
% * Some musicians consider fans their friends that they can count on for support
% * Scalability limits the interactions
% * Distance may maintain ``mystique," but many artists think this is overrated
% * Some fans may be overly emotional, feeling `false intimacy`
% * Social media can, of course, also be a channel for negativity, but this is easy to block
% * Of course, every artist and fan is different
% Wikstrom, 2013:
% * ``The phenomenon whereby the audience not only passively consumes culture but also contributes in the production of that culture is often referred to as participatory culture (Jenkins, 2006)"
% * Wikstrom argues that ?social and creative music use is the normal way in which music fans use music in the new economy? (171)

% Performance in a participatory culture:
In a participatory culture, social networks form around live events. A simple example is an online event page, where fans can learn more about an upcoming performance, see who else plans on attending, and communicate and share with other fans. Some artists reach out to attendees beforehand as well. Alternative rock band Wilco, for example, allows fans to request songs for specific tour dates on their website. Many music festivals make use of social media, displaying messages or photos shared by audience members on big screens around the festival grounds. Some artists even share photos from their own performances. Toronto band Born Ruffians, for instance, takes photos of their fans from on stage and posts them on Facebook, allowing attendees to find and tag their faces in the crowd. The practice of `bootlegging' -- recording concert audio to be shared with other fans -- has existed for decades in rock music. While this activity is surrounded by legal issues, many artists openly encourage it; the Grateful Dead and Phish are two well-known examples. Today, artists like Bruce Springsteen and the Red Hot Chili Peppers are saving would-be bootleggers the trouble and providing free professional live recordings online after each of their shows. Video recording, meanwhile, has become a contemporary version of this practice. Fans film parts or all of a performance on their personal devices and post the videos online. As with traditional bootlegging, some artists protest this, explicitly asking fans to keep their devices in their pockets\footnote{\url{http://stereogum.com/1400701/she-him-are-the-latest-act-to-ban-camera-phones-via-patronizing-signage/news}}. Radiohead and the Beastie Boys both embraced the concept, on the other hand, using fan-shot footage to create multi-perspective concert videos. Welcoming sharing in this way forms new relationships between performer and audience.
% NOTE: Look at your Introduction for material for this paragraph
% The National repost fans' tagged Instagram photos from their shows
% Auslander, 1999: "Rock must constantly change to survive" (Grossberg)
% Wilco allows users to request songs through their website
% Rich Aucoin performs with projects, sometimes incorporating audience members' names. He usually ends shows by throwing a parachute over the crowd and singing/dancing with audience members.
% Born Ruffians take Instagram photos of their audience at every single show, post it to Facebook, and allow everyone to tag themselves
% Sexton, 2007: Interactivity can benefit sound art. Previous interactive sound art has been designed in order to break down barriers between listener and creator. Synchronous and non-synchronous interactive sound art projects exist, each with their positive and negative qualities.
% Bootlegging has seen popularity with certain bands and genres. Today, bands are beginning to offer bootlegs themselves (e.g. RHCP, Springsteen).
% There have been studies done on using technology for new forms of collaborative music creation -- pure participatory performance, where each user is an equal (Gurevich, Tanaka). But this doesn't reflect the nature of artist-fan relationships (?).
% Recording concerts with cellphones is a divisive for concertgoers and artists alike. Some artists enforce cellphone bans, while others like Radiohead and The Beastie Boys have encouraged filming and even made concert films from the collective footage. It is certainly a new way for fans to participate, but does it remove them too much from the experience?

The above examples are indeed providing new ways for audience members to participate in live music performances. However, in most cases, the actual participation is occurring either before or after the performance itself. The next section details work by researchers and artists that are exploring new methods of audience participation \textit{during} performances.


\section{Related Work}

This section explores research done on crowd-based interfaces and facilitating audience-performer interaction. Case studies are also presented of projects that were implemented at large-scale events with popular music artists.

\subsection{Crowd-Based Interfaces}

Designing for large groups of people has only recently attracted notable interest in the field of human-computer interaction (HCI). The recent ubiquity of public displays, networked personal devices, and location and movement tracking technology is allowing for new possibilities in this area, Brown et al. (2009) explain. These researchers emphasize that designers must take into account not only individual user experiences, but the experience that will emerge from a large assembly of users. Furthermore, there are many different types of crowds to consider. A crowd may be made up of similar or different people, anonymous or acquainted with each other. Their attention may be dispersed, or they may have a shared focus. They could be acting independently or toward a common goal. The following HCI-focused research investigates different methods for giving a crowd of people control over a system. The contexts vary from interactive dance clubs to multiplayer games to scoring systems for sporting events.
% Brown, 2009:
% * Public displays, networked devices, and location/movement tracking allow for new possibilities
% * Crowds can have shared (at a performance) or dispersed (in a park) attention/awareness/focus. Interactions can therefore occur within the crowd or between the crowd and an object of interest.
% * Crowds can consist of similar or heterogenous people, anonymous or acquainted, acting independently or toward a similar goal
% * A crowd?s emergent experience is based on many individual experiences. Social or technological barriers may hinder individuals? participation.
% How are interactive systems best designed for crowds?
% Designing interactive systems for crowds -- large public displays, crowd-based gaming, sports spectators
% Designing interactive systems for crowds in live music environments
% Designing systems for participatory music experiences
% Davidson, 1997:
% * Performance etiquette is usually formed by crowd mentality, following the majority
% * Performers pick up information from the audience's broad and specific behaviours
% * Visuals help audiences read the performer's intentions
% Sexton, 2007: Simple synchronous interactions in sound art projects left users with little to explore, resulting in a "flat" experience
% Jourdain, 1997: We move to music in order to "represent" it. This also amplifies, resonates the musical experience.
% Levitin, 2006:
% * "In every society of which we're aware, music and dance are inseparable." Ancient music was based on rhythm and movement. Combining rhythm and melody bridges our cerebellum and cerebral cortex.
% * Ties between music and movement have only been minimized in the last 100 years
% Kelly, 2007: Technology incorporated into a show can either be addressed as part of the show or hidden and made illusory

\subsubsection{Ulyate and Bianciardi}

Ulyate and Biancardi's 2001 paper describes the Interactive Dance Club, a venue that they designed which delivers audio and visual output based on inputs from multiple participants. The goals of the project were to create coherent feedback for individual and group interactions and to allow non-artistic people to feel artistic. Researchers placed emphasis on intuitive and responsive controls, obvious and meaningful feedback, and modular design. Inputs were based on movement and captured using light sensors, infrared cameras, pressure-sensitive tiles, proximity sensors, and simple mechanical switches. By interacting with these devices, users could make notes sound out, modulate music loops, and manipulate projected video and computer graphics. The authors share the lessons learned while testing the concept. They observed that interactions involving full-body movements were most satisfying for users. The form of an object clearly determined how users first attempted to interact with it. They emphasize the practicality of a system that is both distributed and scalable. Designing the interactions required finding a balance between freedom and constraint. They found that, no matter how elegant the system, some users would still find a way to create unpleasant noises. Lastly, they observed that instant gratification is important; feedback that is too delayed or interactions that require too much concentration are ineffective.

\subsubsection{Maynes-Aminzade, Pausch, and Seitz}

Maynes-Aminzade et al. (2002) developed three different computer vision systems that allowed an audience seated in a theatre to control an on-screen game. The first method tracked the audience as they leaned their bodies left and right; the control mechanism was intuitive, but the system required frequent calibration. The second method followed the shadow of a bouncing beach ball which acted as a cursor on the screen; this was also intuitive, but it only involved a few people in the audience at any given time. The third method tracked multiple laser pointer dots on the screen, giving each audience member a cursor; this became somewhat chaotic once a large number of users started participating. Next, the authors present some guidelines for designing systems for audience participation. They recommend focusing on creating a compelling activity rather than an impressive technology. They state that every audience member does not necessarily need to be sensed as long as they feel like they are contributing. The authors suggest that the control mechanism should be obvious or audience members will quickly lose interest. They also note that making the activity emotionally engaging and emphasizing cooperation between players will increase the audience's enjoyment.

\subsubsection{Feldmeier and Paradiso}

In their 2007 study, Feldmeier and Paradiso present a scalable system for wirelessly tracking the movement of a large number of users, allowing a crowd of dancers to influence music and lighting in a club. Citing limitations in computer vision technologies, the researchers decided to use handheld devices as an input mechanism. Users were given cheap and lightweight sensors that emitted radio frequency (RF) pulses when they experienced accelerations over a certain threshold. The music responded to users' movements subtly at first, triggering long, droning tracks. However, if the crowd managed to dance in sync for an extended amount of time, the system would move to a higher ``energy level" and the users would be rewarded with the ability to trigger more interesting melodies and percussion tracks. Receivers placed throughout the club had low sensitivity, effectively producing small ``zones of interaction" around which users would gather. Reflecting on the experiment, the researchers found the devices to be effective -- with low latency, cost, and power requirements. They felt that they succeeded in giving dancers control over the music and took advantage of the crowd's tendency to move in sync. Future work would investigate how to give users even more control while keeping the output aurally pleasing.

\subsubsection{Tomitsch, Aigner, and Grechenig}

Tomitsch et al. (2007) formulated a system to involve audience members in the scoring process of subjective Olympic events (such as figure skating or gymnastics). Each ticket holder would receive a disposable wristband containing a motion sensor, LED, and RF transmitter. The devices (inspired by those that Feldmeier and Paradiso developed) would send RF pulses when users clap, and their LEDs would illuminate to indicate that the pulse has been sent. Receiver stations would count pulses and analyze the frequency of the clapping. Finally, a combined audience score would be calculated based on clapping frequencies in combination with loudness readings from microphones placed around the venue. The authors felt that clapping was an input that could be universally understood, regardless of a user's background. Because many receiver stations could be networked together, they suggested that the project was easily scalable. To evaluate this concept, a group of users were walked through a hypothetical scenario with paper prototypes. Users embraced the idea but doubted it would actually be effective in accurately representing a crowd's opinion. The authors acknowledged this, maintaining that the devices could still serve to enhance the spectator experience.

\subsection{Audience-Performer Interaction}

There are additional considerations when a crowd-controlled system is based around a performance. For instance, performers can enhance an audience's experience by acting as a ``compere," inviting spectators to participate and providing instructions on how to do so (O'Hara, 2008). However, they also serve as a point of focus for audiences; incorporating an interactive system into a performance could incite a fight for the crowd's attention. This section details research that investigate how an interactive system might be incorporated at a live performance.
% O'Hara, 2008: A ``compere" -- inviting spectators to participate and providing instruction -- enhanced interactions
% A performer can act as a compere, solving some problems. Other problems, however, arise.

\subsubsection{Bongers}

Bert Bongers (2000) provides a theoretical HCI framework for interaction between performers, audience members, and electronic systems in a musical performance. He defines ``performer-system-audience" interactions as groups of control systems wherein events are either control or feedback processes. Electronic systems facilitate these processes through electronic sensors and actuators, whereas humans utilize senses and motor systems. Bongers states that a more convincing interaction is one that provides ``multimodal" feedback -- influencing more than one of the users' senses at a time. A real performer-system-audience environment called ``The Interactorium" is described. This featured a performer playing an experimental electronic instrument. Audience members' chairs were equipped with vibration motors and pressure sensors. The audience's movements in their chairs would trigger visuals projected on stage that were then interpreted by the performer. Bongers also makes a notable distinction between ``reaction" and real ``interaction" in this context:

\begin{quotation}
\onehalfspacing	
Real interaction is a living two-way process of giving, receiving and giving back. In a traditional performance set up the audience is passive, the performer active. The increasing use of ``audience participation" in a traditional concert setting acknowledges the need but does not address the issue in any depth - typically the situation created is one of ``reaction" not ``interaction". A situation can be created where the audience and performer meet, each influencing the other, as if conversing, while maintaining the quality of the performance at a high level (p. 49).
\end{quotation}

\subsubsection{Freeman}

\textit{Glimmer} is a project by Jason Freeman (2005) that allows an audience to shape the sound of music being played by a live orchestra. Taking inspiration from analogue participatory performances and crowd-based video games, Freeman developed a ``continuous interactive feedback loop" using coloured lights and computer vision software. Audience members were given battery-operated light sticks that they could turn off and on. Cameras captured the audience and analyzed light activity. Coloured lights in front of each musician instructed them how to play. The more lights that were illuminated in the crowd, the faster the musicians were instructed to play. The audience was divided into groups, each connected to three or four musicians. Groups were rewarded for being active participants; if their lights were frequently turned on and off, the corresponding musicians were given more complex instructions. An onstage screen provided visual feedback to the crowd, displaying coloured rectangles to represent each group's activity. Freeman reported that, in general, users were engaged in the performance. While the system responded to the light sticks turning on and off, he found that most users preferred keeping their lights illuminated at all times. Some users were frustrated that their input did not seem to be influencing the system. Freeman concludes, ``Large-audience participatory works cannot promise instant gratification: giving each person a critical role; requiring no degree of experience, skill, or talent; and creating a unified result which satisfies everyone" (p. 4).

\subsubsection{Gates, Subramanian, and Gutwin}

Gates et al. (2006) investigated the complex interactions between DJs and audience members in nightclubs and how technology might better facilitate them. The authors gathered their information by observing behaviours at nightclubs, surveying DJs, and conducting lengthy interviews with them. Most DJs had similar preferences and performance styles. For example, all of the interviewees said they preferred venues where audience and DJ are mutually visible; this allows them to adjust their performance based on visual cues from the audience. Using quick glances, DJs can observe audience members' facial expressions and body language and the flux of people on to and off of the dance floor. Small, direct interactions can also occur between DJs and audience members, such as exchanges of facial expressions or gestures. DJs use the information they glean from their audience to shape their performance. Most DJs will craft a playlist before performing based on the venue, event type, and expected audience; during the performance, however, the energy of the crowd ultimately guides how the tracks are mixed. In general, the authors found that, as long as there is sufficient visibility, DJs are extremely competent at adjusting their performance based on the audience. Interviewees saw little need for technology to aid their performances; one of the few wishes the DJs expressed was for a method to discover the musical preferences of a given audience. Based on the information collected, the authors present some design recommendations for those wishing to bring interactive technologies to nightclubs. For example, they state that, considering how skillful DJs are at observing audiences, any technology meant to gather information from the crowd must be more efficient than DJs themselves. Such technology, the authors suggest, would be most useful for gathering ``invisible" information like musical preferences. They recommend against using biofeedback systems or systems where audiences have a direct influence on the performance: these methods do not help DJs do their job. The researchers state that gradual changes are more satisfying than immediate ones. Lastly, they emphasize the importance of respecting the DJ's art; technologies should allow them to stay in control of the music and should not add to their already-demanding cognitive load.

\subsubsection{Barkhuus and J{\o}rgensen}

In their 2008 paper, Barkhuus and J{\o}rgensen investigate interactions between audiences and performers at a concert. The authors used observations from traditional rock and rap shows to inform the design of a simple ``interaction-facilitation technology" -- a cheering meter. By tracking the applause patterns at several concerts, it was determined that the two most common reasons for cheering were to express anticipation and to reward the performers. This led to the creation of a cheering meter, an instrument for measuring the volume of an audience -- in this case, to determine the winner of a rap battle. Microphones captured samples of the crowd's cheering, the signal was filtered, the peak volume was measured, and the rating on an arbitrary scale was displayed on large screens onstage. Researchers observed no major issues while testing the system, and they express confidence that their technology helped to enhance the concert for the audience members. In their paper, they outline the main reasons for the cheering meter's success. First, the authors state that the usability of the system is due to the fact that it is based on an already-present behaviour; they recommend ``designing technology that fits the situation and which utilize present activities rather than aiming to employ the latest cutting edge technology" (p. 2929). Next, they suggest that an event should not rely on the success of the technology; the rap battle, for example, could have easily continued if the cheering meter malfunctioned. Lastly, the authors emphasized the importance of immediate visual and/or aural feedback; seeing direct consequences of their actions gives the audience confidence in using the system.

\subsubsection{Tseng et al.}

This paper from 2012 describes the motivation and creative process behind an interactive theatre experience that let audience members connect with a dance performance. The project was realized using projection mapping, a Kinect, a local area network, and a custom iOS app. Audience members downloaded the app before the show and entered a code corresponding with their seat number to connect to the local network. During the first part of the performance, each user was given control over one ``light dot" projected onto the stage. The dot could be moved by moving the iPhone; users could also point their phone's camera at different light sources to influence the brightness of their dot. Later in the performance, audience members could use their phones to trigger sounds and projected images on stage. The dancer, tracked by the Kinect, interacted with the projections, improvising a dance with the light. The authors approached this project by asking, ``How can the audience become an essential element in a performance?" (p. 561). They claim that, while new media has been incorporated into theatre for decades, mobile phones have not been used to their full potential. Feedback collected after the performance revealed overall positive reactions. Some users, however, were uncomfortable having their personal devices connected to an unfamiliar network. Another negative was that not every audience member owned an iOS device, although one of these spectators maintained that she enjoyed the show even while being excluded from the interaction.

\subsection{Case Studies}

Some popular artists have experimented with incorporating participatory technologies into their performances. The following projects deal with both audio and visual output, a range of audience sizes, and varying genres of music.

\subsubsection{D'CuCKOO's MidiBall}

D'CuCKOO, a band active in the 1990s, frequently incorporated technology into their live shows\footnote{\url{http://telecircus.com/yeold/Side/Dcuckoo}}. They invented and constructed several MIDI-based electronic instruments and played them alongside traditional instruments, performing pop music with hints of techno and dance. The MidiBall was a large, helium-filled ball that triggered sounds and visuals on stage when struck by audience members. As D'CuCKOO performed, they let the ball bounce around the crowd and accompany their music.

\subsubsection{Plastikman's SYNK}

Plastikman, the alias of electronic musician Richie Hawtin, had a smartphone application developed to accompany his 2010/2011 world tour\footnote{\url{http://hexler.net/software/synk}}. The app was activated when it connected to the Wi-Fi network at the performance, and various modes were enabled as the show progressed. ``Konsole" mode displayed live performance information such as the tempo and names of the tracks being played. ``Kamera" provided a live video stream of the performer's perspective. ``Synkotik" displayed visuals that were synchronized with on-stage visuals. Lastly, ``Logikal" allowed users to rearrange audio samples using their touch screen, influencing the sounds played by the performer. Before and after the performance, users could connect with each other in the app's chatroom.

\subsubsection{Kasabian's Interactive Stage Show}

UK-based studio Nanika helped rock group Kasabian bring audience members into their performances by making their faces part of the stage show during their 2011 tour\footnote{\url{http://nanikawa.com/projects/kasabian-tour-2011-interactives}}. Cameras captured video of the audience, and face-tracking software identified individual audience members. Graphics were then applied to the footage, highlighting the tracked faces and drawing lines between them. The resulting video was projected on the backdrop behind the band. Upon seeing their faces on the large screen, most audience members became excited and began engaging with the camera.

\subsubsection{Amex Unstaged: Usher}
An Usher concert streamed online in 2012 allowed at-home viewers to participate in the show\footnote{\url{http://momentfactory.com/en/project/stage/Amex_Unstaged:_Usher}}. After posting tweets through the website, users' words appeared on screen behind the performers, mixed with stylized visuals. Online users could also create an animated avatar which virtually danced behind Usher during the performance of his last song. All user-submitted content was screened by producers to ensure it was suitable to display. 

\subsubsection{PixMob}

PixMob is a patented wireless technology that enables the control of multiple LED-embedded objects\footnote{\url{http://pixmob.com}}. By giving PixMob objects to spectators, concert producers can create a controllable LED light show within the audience. The objects are activated with signals from infrared transmitters. Like normal lighting fixtures, the transmitters' beams can be shaped with lenses and controlled using the DMX512 digital communication protocol. Objects light up when they are hit by a beam, so patterns of moving light can, in essence, be painted across the audience. Light shows are programmed, simulated, and controlled using a MIDI controller or an iOS application. Previous PixMob objects include balls, wristbands, pendants, and beads, and custom object creation is available as well. Past clients include Microsoft, Eurovision, Heineken, and Arcade Fire (as described in Chapter 1).

\subsubsection{Wham City Lights}

Wham City Lights is a smartphone application that allows multiple device screens to display light shows in sync during a concert\footnote{\url{http://whamcitylights.com}}. Audience members with an iOS or Android device can download the app before the show. Once the show has begun, an operator activates lighting cues by playing encoded, ultrasonic tones; devices with the app open ``hear" these tones and perform the corresponding cues. This can be done at nearly any scale as long as every device is able to hear the tones. Light shows can be created live or programmed in advance using an online editor. The concept was originally developed by US musician Dan Deacon. His intention was to prevent concertgoers from using their personal devices and disengaging during live performances. Organizations like Billboard and Intel have since made use of this technology at their events.

\subsubsection{Xylobands}

Xylobands are controllable, multicoloured LED wristbands designed to be worn by potentially thousands of users at entertainment events\footnote{\url{http://xylobands.com}}. They are controlled with a radio transmitter. Using accompanying software, operators can turn all of the Xylobands on and off or illuminate only certain colours. The transmitter has a range of around 300 meters. Each wristband contains a small printed circuit board that holds an RF receiver and an 8-bit microcontroller, all powered by small coin cell batteries. The technology was originally developed for the band Coldplay, and wristbands were handed out to all concertgoers during their 2012 world tour. Giving the wristbands to each audience member at every performance reportedly cost the band \euro{}490 000 (around \$680 000 CAD) per night\footnote{\url{http://rte.ie/ten/news/2012/0615/438139-coldplay/}}.