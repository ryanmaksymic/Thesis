\chapter{Conclusion}

A live music performance can be extremely effective in evoking an emotional response in audience members. An avid concertgoer myself, I am always hoping that each performance I attend will be something of a transcendent experience. Light, sound, and atmosphere can come together in a truly captivating way, but only every so often. When it does happen, I find my mind clear of all worries and anxieties; I am effortlessly living in the moment. At the onset of this thesis project, my goal was to find ways to more effectively bring these kinds of experiences to others. I hypothesized that effectively making the audience members a part of the performance would intensify their emotional responses. Through my research, I have found that simple and affordable technologies can be utilized for this very purpose. Literature reviews, examining precedent technologies, interning at StudioFeed, primary research, prototyping, and carrying out real-world user testing led me to develop a device that proved to more deeply connect audience members to a live music performance.

A literature review provided the theoretical base upon which the rest of the project was built. My initial research viewed the concept of live music through a psychological and sociological lens. I first investigated why humans have such an innate connection to music. It is clear that music communicates with people in a very unique way; researchers believe it can help listeners achieve a state of ``authentic happiness without any apparent negative side-effects" (Lamont, 2011, p. 244). I also sought the ingredients for a successful performance; ``Original content, social connection, environmental context, and the wonder of firsthand experience," Rose (2008, p. 16) says, are the most important components. With these as a guide, I moved on and began considering technology's place in modern music performances. It seemed widely agreed upon that modern technologies are the key to advancing the practice of live music performance. I identified a divide in the mindsets of performers in concertgoers -- those who approve of using personal mobile devices (e.g. smartphones) at concerts and those who do not. Finding no clear way to reconcile these two attitudes, I made the decision that my project would not be based on personal mobile devices. Lastly, I cited sources that identified the recent economic shift in the music industry towards live concerts; this reinforced the relevance of my project in general.

The number of similar precedent projects I identified throughout my research is admittedly large, but each provided me with at least some inspiration and guidance. The initial inspiration for my thesis came from a project called ``Summer Into Dust" by Montreal-based studio Moment Factory. This project involved hundreds of wireless, LED-embedded balls that were poured over the audience at a music festival. Each ball was individually controlled and programmed to blink and change colour in time with the music. I loved the way music, lighting, interactivity, and the element of surprise were combined to create what was surely an unforgettable moment for the festival-goers. Similar projects have emerged since. Coldplay's Xylobands were on the wrist of every member of their audience -- usually thousands of people -- and blinked along with the music. Electronic musician Dan Deacon has a personalized smartphone app that turns concertgoers' smartphones into a synchronized light show. I wanted to produce a similar experience with my work, but I felt that the audience should have a more direct impact on the performance. I looked to other projects for direction and soon came across an electronic music event called Up Next; here, attendees could control a digital fireworks show using their smartphones. This was the kind of direct interaction I had in mind. These precedents gave me much to think about as I was conducting my research.

Also coinciding with the early stages of my project was an internship at StudioFeed, a Toronto-based organization who aim to support independent music through technology development and community engagement. StudioFeed founded the Sound In Motion electronic music and arts festival and developed the SubPac -- a ``personal subwoofer" that sends music's low frequencies to the listener's body via physical vibrations. My work involved helping with festival setup and creating promotional installations for the SubPac. Spending time at the Sound In Motion festival gave me a new perspective on live music; as a fan of rock and pop music, attending an electronic music festival helped me think about the different kinds of audiences and what their different desires might be. Working with the SubPac also inspired new ways of thinking. The way this technology converts sound into physical sensations reminded me that music can be much more than just an auditory experience. These lessons helped to inform the design of my project.

I had some more questions about people's relationship with music and performance, so I carried out some primary research to investigate. The main part of this research was a questionnaire that was distributed electronically. My subjects had varying degrees of fondness for music. Asking questions like, ``In what setting are you most likely to experience a strong emotional response to music?" and, ``What most often distracts you at a live music performance?" helped me form a model of the average concertgoer. In general, I found that most people had experienced strong, positive reactions to a live music performance before and that they were open to interacting with subtle, non-intrusive technologies in order to make future concerts more engaging.

With a groundwork of theoretical and real-world knowledge, I began playing with various technologies in hopes of finding one or more that would align with my goal. I wanted a small device for each member of my audience that was capable of measuring and transmitting simple acceleration data to a receiver on or near the stage. Given my experience with the Arduino platform, I decided it would be the best environment for my prototyping. I investigated all of the smallest Arduino-based devices, looking at the Arduino Mini, Micro, and Fio, as well as the incredibly tiny Femtoduino and Digispark. Finding the best wireless solution required sorting through the numerous available options -- Bluetooth, XBee, Pinoccio, etc. There were also several accelerometer devices to choose from. Ultimately, combining the Digispark microcontroller with a Bluetooth transmitter and low-power, three-axis accelerometer chip provided accuracy and low latency. With a working prototype, I set out to design a printed circuit board that would minimize the space. After much polishing, I ordered and assembled five of the boards. The devices were strapped on to my classmates' wrists and turned on; after some tweaking, the receiver laptop was successfully reading five consistent streams of accelerometer data. Clapping, hand waving, and fist pumps each produced a unique signal. The next step was to work on the software that would be processing this data. Using Processing, I developed a simple program that displayed the incoming data and allowed the user to select which streams would be allowed through the serial port, simultaneously mapping them into a desired range of numbers. From here, this data could be fed into other applications. This effectively allowed the audience members themselves to control things like on-stage lights, synthesized sound, or any other software-driven effect.

With a working system, my last remaining task was to test it in a real-world situation. I needed to find a venue and musicians who were interested in trying out my technology. Luckily, there is no shortage of venues and musicians in Toronto. The Great Hall is a sizable, open performance space in the city's Art and Design District. There was a concert scheduled at this venue in mid-March featuring two notable pop-rock bands from the Toronto area; I approached the artists with my project and was pleased when they both responded positively. The weeks leading up to the show involved a lot of organization. I had to assemble and test twenty of my wireless devices. I met with the artists and familiarized them with the tech; this also involved making some small changes to the programming at their request. I nervously watched ticket sales for the event, hoping for a decent turnout. On the night of the show, the devices were tested during the soundcheck, and everything seemed to be working fine. Before the first band went on, I distributed my devices to the twenty audience members nearest to the stage, explaining a bit about them and ensuring they were worn properly. As the first song started, I observed the audience members playing around with the devices, seeing how the lights on stage reacted to their movements, and they seemed amused. By the middle of the set, the users seemed to have forgotten they were wearing the devices but continued to move and participate in the performance. For the second band, some users passed on the devices to other audience members, allowing them to try it out. During one song, the musicians and I had programmed a section where the devices became connected to software synthesizers, and the audience expressed great joy in helping to create music with the band. Overall, the show went well without any major problems.

After the concert, I spoke to the musicians and audience members involved. The users confirmed their enjoyment, the musicians expressed excitement in finding new ways to use the technology, and both provided useful criticisms. For example, audience members noticed when their device was not reacting correctly; accuracy has to be improved in future iterations. The device's casing could also be more attractive and comfortable. Additionally, the software interface should be polished for improved usability. Reflecting on the test, I feel that I addressed my research question in a very exciting way. I connected audience members to a live music performance in a unique manner, resulting in a unanimously positive reaction. With each device costing less than \$25 to build, this is certainly a cheap solution in comparison to some of the multi-million-dollar stage shows being produce today. Simple and affordable technologies can indeed enhance the emotional response in audience members at a live music performance.