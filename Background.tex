\chapter{Background}

This chapter provides a brief history of music as a human activity, live performance, and performance technologies. I also conduct case studies on some specific live music events that are closely related to my interests. Lastly, previous research investigating designing interactive systems for crowds is reviewed.

\section{Music, Performance, and Technology}

% Music history:
% Levitin, 2006:
% * Archaeologists have uninterrupted records of music everywhere and every era humans existed
% * Music has only been a spectator activity in the past 500 years
% Jourdain, 1997: Different cultures -- e.g. New Guinea, the Congo, Australian aboriginals -- have different uses for music -- e.g. as a gift, to settle lawsuits, to tell intricate stories
% Turner, 2011: Repetitive, hypnotic mantras are used by African tribes to draw out harmful spirits

% Music as a social activity:
% Jourdain, 1997: Anthropologists believe music evolved "to strengthen community bonds and resolve conflicts"
% Levitin, 2006:
% * Why did music evolve? Social bonding and coordination, survival methods.
% * A cluster of genes may control both musicality and outgoingness
% Auslander, 1999:
% * Communities are formed based on how the audience interacts, with no dependence on the spectacle at hand
% * Removing the gap between audience and performer stops it from being a performance. Striving for unity only emphasizes the gap. (I disagree.)
% Small, 1998:
% * Community formed at concerts represent ideals, not realities
% * At a live music performance, the values of a certain social group are "explored, affirmed, and celebrated." Meaning comes from the relationships between the people and the venue, among those taking part, and between the sounds being made.
% Turino, 2008:
% * Making music together leads to "sonic bonding"
% * Connecting with others through art is ``crucial for social and ecological survival'' (Bateson)
% * Music and dance promote "flow" -- optimal experience (Csikszentmihalyi)
% * Music making or dancing is a realization of ideal human relationships
% Turner, 2011:
% * Communitas arrises from people ridding themselves of concern for status and structure, readiness to ``see their fellows as they are''
% * The communitas spirit does not take sides, yet it can be ``prostituted'' to produce prejudice
% * A communitas event (here, Carnaval) is not ordinary, it is brief, yet it is vividly remembered
% * ``Music is a fail-safe bearer of communitas''. It is ephemeral, emotional, it breaks the rules. "Its life is synonymous with communism, which will spread to all participants and audiences when they get caught up in it."
% * We cannot share our bodies, but music allows us to share ``precise time''
% * Flow: In control of the situation, yet fully absorbed "so that it seems to be in control of us." A loss of ego. Related to communitas.

% Participatory performances:
% Turino, 2008:
% * Artists can shift between participatory and presentational performance
% * No artist/audience distinction, only participants possibly performing different roles. Emphasis on sonic and physical interaction. "Heightened social interaction." The "doing" is more important than the end result.
% * Cultural background will determine if a person is comfortable joining a participatory performance
% * Leads to diminished self-consciousness since everyone is equally contributing
% * Different roles of different difficult allow for everyone to feel welcome and achieve flow. "Core" and "elaboration" roles cater to advanced and non-advanced performers respectively.
% * Intensity of participation comes before quality of performance. Inept performers are noticed but not addressed.
% * Participatory performance does not fit well in a capitalist society. It yields social value, not financial value.
% * Open form: Basic motives repeated over and over. Easy for newcomers to join in. "Security in constancy." Can facilitate flow.
% * Hall: Repetition can increase intensity. Synchronicity comforts people. However, Turino believes repetition can "lead to boredom in presentational contexts... Participatory musical styles do not transfer well to presentational stage situations, in spite of nationalists', folklorists', and academics' attempts to bring them into presentational settings." (How do I feel about this?)
% * Wide tuning, loud volumes, and overlapping textures provide a ?cloaking function? that makes people more comfortable participating
% * Virtuosic solos are not common
% * Some participatory performances are sequential -- everyone gets a turn (e.g. Karaoke)
% * Participatory performance may constrain creativity, but it benefits more people overall
% * The doing of activities as a group makes participatory events more engaging than performative

% Presentational performances:
% Turino, 2008:
% * The performer prepares and presents music to the audience, who does not participate in the performance
% * Closed form. Predetermined beginning, middle, and end.
% * Presentational musicians aim for detail, smoothness, coherence. Predictability and control are key. Some participatory-leaning performers may make adjustments to accommodate the audience.
% * Planned contrasts are implemented in performances to keep the audience interested. This would derail participatory performance.
% * Attractive because it partially fulfills desire to connect to an artist. Corbett suggests concerts are designed to tease but not consummate this connection, "leaving consumers wanting more."

% Evolution of presentational performances:
% Jourdain, 1997:
% * Public concerts were virtually unheard of until the 1600s
% * Only after the industrial revolution in the 1800s did common folk gain disposable income and public concerts gain popularity
% * At early operas people talked, ate, and played cards. Even performers would "jabber away" and speak to audience members.
% * Audiences gradually became annoyances to performers. Some conductors would lock out latecomers, silence applause, and ignore requests.
% * Near the end of the 1800s, "art music" performances became more modern -- formal and quiet
% * Some critics call symphony orchestra performances "the epitome of capitalist oppression" -- hierarchical, no audience involvement
% Small, 1998:
% * Musicians were first hired to play with aristocrats. "Traveling virtuoso-entrepreneurs" first started playing public performances when they realized the desire of the middle class to display their wealth.
% * Private performances surfaced around 1730. Ticketed events were not abundant until the 1800s.

% Recorded music dominates today:

% Live music > Recorded music:
% Kelly, 2007: Pop performances are typically re-presentations of recorded material ("reviving the musical corpse")
% Jourdain, 1997: Technology put music everywhere, lessening its "nourishment." "We live in an age of widespread musical obesity."
% Auslander, 1999:
% * The desire for live experiences is a product of mediatization. People want to see concerts so they can experience beloved recorded songs in person.
% * Live performances have great cultural value, but Auslander believes mediatized performances will soon be more valuable
% Davidson, 1997: Performers perform better in live situations due to "social facilitation"
% Inglis, 2006: 
% * Live performance is the principal source of musical change/innovation
% * It is unpredictability that makes live better than mediatized performance
% Marshall, 2006: When Bob Dylan went electric at a festival in 1965, it changed the course of pop music
% Small, 1998:
% * Treating the piece of music as the "the supreme reality of art" (Benjamin) implies some fallacies
% * "Musical works exist in order to give performers something to perform"
% * Music is not a thing. "Musicking" is an action.
% * Since people today can hear and see concerts without having to leave their home, the concert "takes on a new and more concentrated ritual significance"

% Technology can enhance presentational performances:
% Auslander, 1999: Mediatized performances can offer more intense sensory experiences than traditional performances
% Kelly, 2007:
% * Embodied co-presence of performer and audience is an important convention of pop performances. "Immediacy, intimacy, self-expression and certain kinds of performer-spectator interaction" are expected. The venue becomes "not only a hearing place, but also a seeing place and a being place."
% * "Presence, representation and self-expression" can be either amplified or attenuated by audiovisual technologies in pop shows
% * A 1966 Andy Warhol event (featuring the Velvet Underground and Nico) mixed live performance with projection. Attendees viewed the projections as "part of the show."
% * Imagery framing the performance generally amplifies the performer's presence. This goes against Benjamin's claim that technological intervention will "devalue the here and now of the artwork." Kelly argues that screens showing different perspective in fact amplify the sense of presence. Referring to a Madonna performance, the author states that the juxtaposition in scale between screen and performer actually creates a 'theatrically structured intimacy."
% * Displaying clips and themes from her music videos at a Madonna concert creates feelings of a "shared past" in the audience. (How might we create instead a "shared present"?)
% * Kraftwerk left the stage and were replaced by robots for their song Robots. This "problematises traditional modes of presence in pop-concert reception."
% * Gorillaz is a group fronted by animated characters. They can make a bigger emotional impact than human performers.
% * One Gorillaz performance featured a deceased singer using prerecorded video and audio -- effectively remediating presence. (How can audience presence be remediated?)

% MTV influencing live performance:
% Auslander, 1999:
% * Theatre and mass media (TV, film, recorded music) are rivals. They are not inherently this way, but were made this way over time. Live performance's response to this rivalry has to imitate mass media.
% * "The function of live performance under this new arrangement is to authenticate the video" -- TV > performance (in 1999)
% * The live/mediatized rivalry is not equal. Television is dominant (in 1999). (The Internet is dominant now?)
% Koojiman, 2006:
% * After MTV started in 1981, television became the main promotion tool for music
% * At the Motown 25 performance, Michael Jackson lip-synched "Billie Jean." The focus was on visuals, with Jackson performing the moonwalk for the first time. This was during the rise of MTV, and the performance was essentially a reenactment of the music video.
% Burns, 2006: Madonna's performance at the 1984 MTV Video Music Awards was designed to be a music video, and it literally became one

% Modern performances are no longer participatory:
% Kelly, 2007: Horn (2000) provides a definition of a pop music performance that belittles the role of the audience
% Jourdain, 1997: Technology has made music less participative. (Is that a challenge?)
% Kelly, 2007: Notions of "identity, authenticity and originality" in pop performance are being challenged. This is allowing for active participation for audience members.
% In Western culture, technology at live performances overpowers the voice of the audience. Additionally, audience actions like applause are also losing their meaning (Kershaw, Webster).
% Small, 1998:
% * Division of talented and non-talented performers is based on a falsehood. We are all capable of creating music.
% * Sharing performance with strangers in the West is abnormal. Most societies' performance feature members of the community.
% * Classical music performance turned audience members into spectators, consumers
% * Audience separated from, "dominated" by performer is a social feature of modern performance. Perhaps a desire to make the perform more "mysterious."
% * 60s/70s folk fests cultivated community, but it was an orchestrated "illusion." Constraints are always necessary.
% * Performers dressing in uniform are separating themselves and their responsibilities
% * Modern orchestra concerts emphasize listening, specialization, formal settings, ticketed admission. They celebrate separation of producer and consumer.
% Kershaw, 2001:
% * 20th century London theatre audience have become much more prone to standing ovations than they once were. This may indicate disempowerment and the fall of the theatre as a political environment. Over the 20th century the role of the theatregoer changed from patron to customer. Early audiences would boo or yell approval. After theatre became organized and lavish, a standing ovation became `an orgasm of self-congratulation for money so brilliantly spent... The power of community all but evaporated.`
% * Applause may be an attempt to form a momentary community

% Social media:
% Baym, 2012:
% * Audience-performer relationships were traditionally `parasocial` ? one-sided
% * Social media has reconfigured this relationship, eliminating `rock gods` and `pop stars`
% * Interviewed musicians feel social media presence is necessary for success
% * Some musicians consider fans their friends that they can count on for support
% * Scalability limits the interactions
% * Distance may maintain `mystique,` but many artists think this is overrated
% * Some fans may be overly emotional, feeling `false intimacy`
% * Social media can, of course, also be a channel for negativity, but this is easy to block
% * Of course, every artist and fan is different
% Listeners are more active; remixes, mashups, fan vids, sharing, commenting, etc. are creating a new paradigm for music enjoyment.
% Karpovich, 2007: Accessible technologies allowed for the popularity of fan vids, non-professional narrative clips inspired by MTV-style music videos


\section{Participating in Performances}

In the past few years alone, a number of large-scale projects have been realized that use technology to expand live performances into the audience. Some were implemented as experiments at one-time events, while others found commercial success and are being used by big-name companies.
% Auslander, 1999: "Rock must constantly change to survive" (Grossberg)
% Divide into types of interactions and include brief descriptions of as many relevant instances as possible (e.g. handing out tambourines, Wilco's online song requests)
% Wilco allows users to request songs through their website
% Rich Aucoin incorporates attendees' names in his projections (before the show only?)
% Born Ruffians take Instagram photos of their audience at every single show, post it to Facebook, and allow everyone to tag themselves
% Sexton, 2007: Interactivity can benefit sound art. Previous interactive sound art has been designed in order to break down barriers between listener and creator. Synchronous and non-synchronous interactive sound art projects exist, each with their positive and negative qualities.
% Kelly, 2007:
% * Elvis Costello uses a "Wheel of Songs" that allows spectators to determine the setlist
% * Kraftwerk handed out beeping electronic calculators so the audience could join in during their song Pocket Calculator
% Bootlegging has seen popularity with certain bands and genres. Today, bands are beginning to offer bootlegs themselves (e.g. RHCP, Springsteen).
% There have been studies done on using technology for new forms of collaborative music creation -- pure participatory performance, where each user is an equal (Gurevich, Tanaka). But this doesn't reflect the nature of artist-fan relationships (?).
% Recording concerts with cellphones is a divisive for concertgoers and artists alike. Some artists enforce cellphone bans, while others like Radiohead and The Beastie Boys have encouraged filming and even made concert films from the collective footage. It is certainly a new way for fans to participate, but does it remove them too much from the experience?

\subsection{Recent Examples}

\subsubsection{Wham City Lights}

Wham City Lights is a smartphone application that allows multiple devices to display light shows in sync during a concert. Audience members with an iOS or Android device can download the app before the show. Once the show has begun, an operator activates lighting cues by playing encoded, ultrasonic tones; devices with the app open ``hear" these tones and perform the corresponding cues. This can be done at nearly any scale as long as every device is able to hear the tones. Users generally hold their devices up or wave them above their heads during the show. Light shows can be created live or programmed in advance using an online editor; cues include flashing colours, camera flashes, GIFs, text, and sound.

The concept was originally developed by US musician Dan Deacon. His intention was to prevent concertgoers from using their personal devices and disengaging during live performances. Deacon tested the app at his own shows and received a positive response. Today, Wham City Lights licenses their general-purpose app for different kinds of events; they also develop custom apps to include branding, tour dates, etc. Musicians and organizations like Brad Paisley, the Billboard Music Awards, and Intel have made use of this technology at their events.

\subsubsection{Xylobands}

Xylobands are controllable LED wristbands designed to be worn by potentially thousands of users at entertainment events. They are controlled using a proprietary piece of software downloaded to a laptop; the laptop must then be connected to a radio transmitter. With the software, an operator can turn the Xylobands on or off, select which colours are illuminated, and control the speed of the LEDs' flashing. The transmitter has a range of around 300 meters. Each wristband contains a small printed circuit board that holds, among other components, an RF receiver and an 8-bit microcontroller. The electronics are powered by three 3 V coin cell batteries.

The technology was originally developed for the band Coldplay, and wristbands were handed out to all concertgoers during their 2012 world tour. Giving the wristbands to each audience member at every performance reportedly cost the band \euro{}490 000 (around \$680 000 CAD) per night. UK-based toy development company RB Concepts Ltd. are the creators of the Xyloband. Their website advertises that Xylobands can be customized and used at concerts, festivals, sports stadiums, or corporate events.

\subsubsection{PixMob}

PixMob is a patented wireless technology that enables the control of multiple LED-embedded objects. By giving PixMob objects to spectators, concert producers can create a controllable LED light show within the audience. The objects are activated with signals from infrared transmitters. Like normal lighting fixtures, the transmitters' beams can be shaped with lenses and controlled via the DMX512 protocol. The objects light up when they are hit by a beam, so patterns of moving light can, in essence, be painted across the audience. Light shows are programmed, simulated, and controlled through a software package called LAVA; they can also be controlled in real time using a MIDI controller or the LAVA iPad app. Previous PixMob objects include balls, wristbands, pendants, and beads, and custom object creation is available as well. PixMob also offers ``second life" customization: objects can be programmed to react to sounds, play an mp3 track, or communicate with the user's personal computer after the show is over. Past clients include Microsoft, Arcade Fire, Eurovision, and Heineken.


\section{Designing for Crowds}

Designing for large groups of people has only recently attracted notable interest in the field of human-computer interaction (HCI). As interactive systems become increasingly ubiquitous, HCI researchers are asking how the needs of multiple people in a public space differ from those of an independent user. The characteristics of live performance make it an especially useful venue for these investigations; thus, conveniently, much of the research done in this field focuses on concerts, theatre performances, and dance clubs.
% How are interactive systems best designed for crowds?
% Designing interactive systems for crowds -- large public displays, crowd-based gaming, sports spectators. This has only been on CHI's radar since 2009 (Brown's workshop).
% Designing interactive systems for crowds in live music environments
% Designing systems for participatory music experiences
% Davidson, 1997:
% * Performance etiquette is usually formed by crowd mentality, following the majority
% * Performers pick up information from the audience's broad and specific behaviours
% * Visuals help audiences read the performer's intentions
% Sexton, 2007:
% * Music is always associated with other media -- performer movement, words in programme notes, album art. Meaning emerges from these relationships.
% * The original phonograph distressed listeners with its lack of visuals. Edison tried to add visual elements to the invention.
% * Interactivity may bring up authorship issues
% * Simple synchronous interactions in sound art projects left users with little to explore, resulting in a "flat" experience
% Jourdain, 1997: We move to music in order to "represent" it. This also amplifies, resonates the musical experience.
% Levitin, 2006:
% * "In every society of which we're aware, music and dance are inseparable." Ancient music was based on rhythm and movement. Combining rhythm and melody bridges our cerebellum and cerebral cortex.
% * Ties between music and movement have only been minimized in the last 100 years
% Kelly, 2007: Technology incorporated into a show can either be addressed as part of the show or hidden and made illusory
% ^
% THIS SECTION IS ONLY ABOUT HCI AND DESIGNING FOR CROWDS. MOVE UNRELATED STUFF ELSEWHERE

\subsection{Maynes-Aminzade, Pausch, and Seitz}

In their 2002 paper, Maynes-Aminzade et al. describe three different computer vision systems that allow an audience to control an on-screen game; they also outline the lessons they learned about designing such systems. The first method tracks the audience as they lean to the left and right. The control mechanism was intuitive, but the system required frequent calibration. The second method tracked the shadow of a beach ball which acted as a cursor on the screen. This was also intuitive, but it only involved a few people in the audience at a time. The third method tracked multiple laser pointer dots on the screen, giving each audience member a cursor; this was a more chaotic system once the number of dots got overwhelming. Lastly, the authors presented some guidelines for designing systems for interactive audience participation. They recommend focusing on creating a compelling activity rather than an impressive technology; they state that every audience member does not necessarily need to be sensed as long as they feel like they are contributing; and they suggest that the control mechanism should be obvious or audience members will quickly lose interest. The authors also note that making the activity emotionally engaging and emphasizing cooperation between players will increase the audience's enjoyment.

These conclusions provide both guidance and new questions to consider for my own research. While I hope to work with some relatively advanced technologies, it will be important to remember that it is the actual interaction that will determine how engaging the experience is. User-centered design will need to be a major part of my development process. While this paper dealt with accurate control of a video game, my work will address interactions that are more passive and abstract. The authors stress the importance of an obvious control mechanism; considering how much their environment (a movie theatre) will likely differ from mine (a loud, dark music venue), I believe this will be especially crucial for my project. Audience members whose senses are already being overloaded will have even less patience for figuring out how something works. I will have to consider how this and other factors apply to my unique environment. How can I best tap into the emotional sensibilities of an audience at a concert? How might I create a cooperative environment in a situation that is not goal oriented? These are questions I will have to address in my primary research.

\subsection{Ulyate and Bianciardi}

In their paper, the authors describe their ``Interactive Dance Club" -- a venue that delivers audio and video feedback to inputs from multiple participants -- and they present the ``10 Commandments of Interactivity" that guided its creation. The goals of the project were to create coherent musical and visual feedback for individual and group interactions and to allow non-artistic people to feel artistic. Inputs included light sensors, infrared cameras, pressure-sensitive tiles, proximity sensors, and simple mechanical switches. By interacting with them, users could make notes sound out, manipulate projected video and computer graphics, modulate music loops, and control the position of cameras in the space.

This project's ``10 Commandments of Interactivity" contain the following points:
\begin{itemize*}
	\item Movement is encouraged and rewarded.
	\item Feedback from interactions is immediate, obvious, and meaningful in the context of the space.
	\item No instructions, expertise, or thinking is required.
	\item A more responsive system is better than a more aesthetically pleasing system.
	\item Modularity is key.
\end{itemize*}

Lastly, the authors share the lessons that they learned while running the Interactive Dance Club. They observed that interactions involving full-body movements were most satisfying. The form of an object, they found, determined how users first attempted to interact with it. They emphasize the practicality of a system that is both distributed and scalable. Designing the interactions required finding a balance between freedom and constraint. They found that, no matter how elegant the system, some users would still find a way to create unpleasant noise. Lastly, they observed that instant gratification is important; feedback that is too delayed or interactions that require too much concentration are ineffective.

\subsection{Bongers}

In his 1999 paper, Bongers provides a theoretical HCI framework for physical interaction between performers, audience members, and electronic systems in a musical performance. He defines three types of interaction -- ``performer-system", ``system-audience", and ``performer-system-audience." Bongers models the interactions as control systems wherein actions are either a \emph{control} or \emph{feedback} process. Electronic sensors and actuators are discussed, followed by human senses and motor systems. Bongers states that a more convincing interaction is one that provides ``multimodal" feedback -- influencing more than one of the users' senses. Lastly, a few prototypes of novel interaction systems are described. Especially notable is the ``Interaction Chair", which most easily fits in the performer-system-audience category. Here, the performer has the ability to send vibrations through each audience member's seat back, while the chairs contain sensors that allow audience members to influence visuals projected behind the performer. Other projects like this one can benefit from Bongers' theoretical framework; thinking in terms of control and feedback processes may provide new perspectives on a  system's design.

\subsection{Barkhuus and Jorgensen}

Barkhuus and Jorgensen's paper investigated interactions between audiences and performers at a concert. The authors used observations from traditional rock and rap shows to inform the design of a simple "interaction-facilitation technology" -- a cheering meter. By tracking the applause patterns at several concerts, it was determined that the two most common reasons for cheering were to express anticipation and to reward the performers. This led to the creation of a cheering meter, an instrument for measuring the volume of an audience -- in this case, to determine the winner of a rap battle. Microphones captured samples of the crowd's cheering, the signal was filtered, the peak volume was measured, and the rating on an arbitrary scale was displayed on large screens onstage.

The researchers reported no major issues while testing the system, and they express confidence that their technology helped to enhance the concert for the audience members. In their paper, they outline the main reasons for the cheering meter's success. First, the authors state that the usability of the system is due to the fact that it is based on an already-present behaviour; they recommend ``designing technology that fits the situation and which utilize present activities rather than aiming to employ the latest cutting edge technology" (p. 2929). Next, they suggest that an event should not rely on the success of the technology; the rap battle, for example, could have easily continued if the cheering meter malfunctioned. Lastly, the authors emphasized the importance of immediate visual and/or aural feedback; seeing direct consequences of their actions gives the audience confidence in using the system. This research focused on a very specific type of event using an almost-gimmicky system, but the design principles it yielded are valuable.

\subsection{Tseng et al.}

This paper described the motivation and creative process behind a Taiwanese interactive theatre experience that let audience members connect with a dance performance. The project was realized using projection mapping, a Kinect, a local area network, and a custom iPhone app. Audience members downloaded the app before the show and entered a code corresponding to their seat number to connect to the local network. During the first part of the performance, each user was given control over one ``light dot" projected onto the stage. The dot could be moved by moving the iPhone; users could also point their phone's camera at different light sources to influence the brightness of their dot. Later in the performance, audience members could use their phones to trigger sounds and projected images onstage. The dancer, tracked by the Kinect, interacted with the projections, improvising a dance with the light.

The authors approached this project by asking, ``How can the audience become an essential element in a performance?" (p. 561). They claim that, while new media has been incorporated into theatre for decades, mobile phones have not been used to their full potential. Feedback collected after the performance revealed overall positive reactions. Some users, however, were uncomfortable having their personal devices connected to an unfamiliar network. Another negative was that not every audience member owned an iPhone; one of these spectators, though, maintained that she enjoyed the show even while being excluded from the interaction.

\subsection{Reeves, Sherwood, and Brown}

This paper investigates the design of technology for crowds by observing and analyzing the behaviour of a group of football fans gathered at a pub. The authors note that most related work has focused on spectators at a performance or on exceptional circumstances like riots. This work instead looks at everyday crowd-based settings where there is no attention-grabbing ``spectacle." To accomplish this, the researchers video recorded a crowd gathered at a pub during a football match and examined the group's behaviour for recurring themes. People were seen singing, jumping, and pumping their fists in the air in sync with each other. In general, these instances of collective participation were all visible or hearable from far away. Once a small group of people began the actions, they would quickly ``snowball" and overtake the crowd. Researchers also noticed the importance of ``shared objects;" an inflatable object bouncing between people, for example, connected individuals at a distance. It was also observed, of course, that not every person in the crowd cared to participate in these group activities.

After outlining these observations, the authors present a list of design lessons that they extrapolated. First, they suggest treating a crowd as a unit rather than a collection of individuals -- for example, exploiting already-present crowd behaviours or allowing for only partial participation. The importance of ``intra-crowd interaction" is also emphasized: allow for shared objects and space-dependent interactions, and take advantage of snowballing by encouraging highly visible/audible actions. Additionally, one should allow for interaction with people on the fringes of the crowd but be aware of problems that could be caused by latency. Lastly, the researchers note that every crowd is different and that each design should reflect the nature of the environment.


\subsection{Gates, Subramanian, and Gutwin}

This paper examines the complex interactions between DJs and audience members in nightclubs from an HCI perspective. The authors gathered their information by observing behaviours at nightclubs, surveying DJs, as well as conducting lengthy interviews with them. Most DJs had similar preferences and performance styles. For example, all of the interviewees said they preferred venues where audience and DJ are mutually visible; this allows them to adjust their performance based on visual cues from the audience. Using quick glances, DJs can observe audience members' facial expressions and body language and the flux of people on to and off of the dance floor. Many DJs stated that they will often exaggerate their movement or speak into a microphone to energize the crowd. Small, direct interactions can also occur between DJs and audience members, such as exchanges of facial expressions or gestures. DJs use the information they glean from their audience to shape their performance. Most DJs will craft a playlist before performing based on the venue, event type, and expected audience; during the performance, however, the energy of the crowd ultimately guides how the tracks are mixed. In general, the authors found that, as long as there is sufficient visibility, DJs are extremely competent at adjusting their performance based on the audience. Interviewees saw little need for technology to aid their performances; one of the few wishes the DJs expressed was for a method to discover the musical preferences of a given audience.

Based on the information collected, the authors present some design recommendations for those wishing to bring interactive technologies to nightclubs. For example, they state that, considering how skillful DJs are at observing audiences, any technology meant to gather information from the crowd must be more efficient than DJs themselves. Such technology, the authors suggest, would be most useful for gathering ``invisible" information like musical preferences. They recommend against using biofeedback systems or systems where audiences have a direct influence on the performance; these methods do not help DJs do their job. The researchers state that gradual changes are more satisfying than immediate ones. Lastly, they emphasize the importance of respecting the DJ's art; technologies should allow them to stay in control of the music and should not add to their already-demanding cognitive load.