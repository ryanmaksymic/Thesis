\chapter{Background}

In this 2002 paper, \citeauthor{maynes02} describe three different computer vision systems that allow an audience to control an on-screen game. They also outline the lessons they learned about designing such systems. The first method tracks the audience as they lean to the left and right. The control mechanism was intuitive, but the system required frequent calibration. The second method tracked the shadow of a beach ball which acted as a cursor on the screen. This was also intuitive, but it only involved a few people in the audience at a time. The third method tracked multiple laser pointer dots on the screen, giving each audience member a cursor. This was a more chaotic system once the number of dots got overwhelming. Lastly, the authors presented some guidelines for designing systems for interactive audience participation. They recommend focusing on creating a compelling activity rather than an impressive technology; they state that every audience member does not necessarily need to be sensed as long as they feel like they are contributing; and they suggest that the control mechanism should be obvious or audience members will quickly lose interest. The authors also note that making the activity emotionally engaging and emphasizing cooperation between players will increase the audience's enjoyment.

These conclusions provide both guidance and new questions to consider for my own research. While I hope to work with some relatively advanced technologies, it will be important to remember that it is the actual interaction that will determine how engaging the experience is. User-centered design will need to be a major part of my development process. While this paper dealt with accurate control of a video game, my work will address interactions that are more passive and abstract. The authors stress the importance of an obvious control mechanism. Considering how much their environment (a movie theatre) will likely differ from mine (a loud, dark music venue), I believe this will be especially crucial for my project. Audience members whose senses are already being overloaded will have even less patience for figuring out how something works. I will have to consider how this and other factors apply to my unique environment. How can I best tap into the emotional sensibilities of an audience at a concert? How might I create a cooperative environment in a situation that is not goal oriented? These are questions I will have to address in my primary research.