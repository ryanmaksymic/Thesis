\chapter{Research Approach}

This chapter presents the primary and secondary research questions to be answered and the rationale behind them. The research design is described and methods are explained and justified.

\section{Questions}

\subsubsection{Primary Question}

\begin{itemize}
	\item How might rock performances be made more participatory using digital technology?
\end{itemize}

\subsubsection{Secondary Questions}
% Tighten these up

\begin{itemize}
	\item \textbf{Experience:} Do participatory technologies actually improve the concert experience for both the audience and performer?

	\item \textbf{Input:} How can inputs from the audience be captured and processed? Which mapping strategies can effectively translate these inputs into an output?

	\item \textbf{Feedback:} How can users be informed that their input has been received and has influenced the system?

	\item \textbf{Output:} How -- if at all -- do audiences want to contribute to rock performances? How much control do rock musicians feel comfortable giving audience members? What aspects of a live performance can be reasonably controlled by a large group of people?
\end{itemize}


\section{Rationale}

These research questions are both culturally and technologically relevant. They also make up a design problem that has not yet been fully addressed by existing research.

% Background; participatory culture:
As we have seen, performance responds to the culture within which it exists. For many of the world's societies, music is purely participatory, and it can be more functional than entertaining. Presentational performance eventually emerged in the West as a way for the middle class to flaunt their newfound wealth. Live music morphed into a visual spectacle when mass media created a visual culture and MTV made images and recorded sound inseparable. Today, there is a growing participatory culture. Music fans are able to communicate and even collaborate with artists online. A handful of artists are already beginning to welcome this sort of connectivity into their concerts. I believe that live performances will continue to be influenced by this participatory culture, and participatory technologies are one method of exploiting and supporting this trend.

% A young research field:
The ubiquity of public displays, connected personal devices, and location-aware technology is making crowd-based interactions increasingly of interest to researchers. Indeed, HCI researchers have only recently begun investigating how systems might be designed for a large assembly of users. In addition to investigating crowd-based interfaces, some researchers have begun asking how these systems might be implemented in a live music environment. There are, however, questions that have not yet been addressed.

% A design problem:
This thesis investigates participatory technologies in the context of a rock concert. While there exists a substantial amount of relevant research, the research questions at hand present a design problem that has yet to be explored. Freeman (2005) created a system to allow an audience to influence music, for instance, whereas Tseng et al. (2012) gave users control over projected visuals. What kinds of outputs might rock audiences wish to control? Gates (2006) found that DJs generally have no desire to implement interactive technologies in their performance environments. Could rock musicians feel the same way? Some previous work treated a crowd as a single source of input (Maynes-Aminzade et al., 2002; Barkhuus \& J{\o}rgensen, 2008), while others gave each participant their own controller (Feldmeier \& Paradiso, 2007; Tomitsch et al., 2007). Which method is most suitable for a group of rock fans? By exploring the specific context of a live rock performance, this thesis contributes to this growing field of research.

% Remaining questions from existing work:
% * Collaborative or individual? Similarly: Fully representative or partly fabricated? (Maynes-Aminzade, et al.)
% * Gates: DJs don't really want audience-interaction technology. How do rock musicians feel? Should participatory technologies help the musicians ``do their job"?
% * Barkhuus and Jorgensen did not involve the performers at all
% * Gates: Gradual changes are more satisfying than immediate changes
% * This work has only looked at dance performances, rap competitions, and nightclubs, however; there is insufficient research on incorporating these systems into rock performances
% The goal is to make a rock show that takes a step towards participatory.

% Live music is a growing industry:
%It should also be noted that the live music industry is growing rapidly (Wikstr\"{o}m, 2013). The digital revolution caused record sales to plummet, whereas live music revenues are larger than ever. While touring used to be a method for promoting recorded music, the opposite is now true. Wikstr\"{o}m speculates that ``live music will soon dominate the entire music industry in the same fashion as recorded music has done during more than half a century" (p. 142-143). Thus, researching new methods for creating impactful live music experiences is a good investment.
% Where should this go???


\section{Methods}

Multiple participatory systems were developed in order to explore the research questions. Given the questions' user-centric nature, it naturally followed to implement a user-centred design process. The ISO standard ``Human-Centred Design For Interactive Systems" (ISO 9241-210, 2010) outlines some key principles that will ensure a design process is user-centred. These include the following: understanding the context of use -- the users, their environment, and the tasks they perform; utilizing iterative processes; and ensuring that users are involved in all design phases. These principles are reflected in the design research methods that were employed -- a combination of ethnographic study and prototyping.

\subsection{Ethnography}
Ethnography is a qualitative exploration of cultural phenomena within a community. Rather than surveying a large number of people about a few topics, ethnographers instead obtain a deep understanding of the lifestyles within a small group of people. Common research tools include participant observation, questionnaires, and interviews. This research method originated in anthropology as a means of investigating ``the practices, artifacts, sensibilities and ideas that constitute and inform our everyday lives" (Plowman, 2003, p. 30). Such information can be very useful for designers. As Plowman explains, the ways in which people interpret and use products are ``deeply cultural activities." In the 1980s, designers began utilizing ethnographic techniques as a means of exploration and concept generation. While true ethnography can involve months or even years of immersion in a culture, design ethnography is typically abbreviated; designers make ``time-sampled observations" of potential users (Martin \& Hanington, 2012). Because of the small sample size, an important part of ethnography is how the data is interpreted and extrapolated.
% Ethnography: Examining cultural phenomena within a group. Surveying modern music fans will provide me with a general sense of their feelings towards music, live performance, and technology. I will interview active musicians to understand how and why they interact with fans, on and off stage, and what they think of new technologies in a performance setting. These methods will allow me to gain a deeper understand the users and environments that are related to my work.
% *Exploration*
% Plowman, 2003:
% * Ethnography originated in anthropology, the study of culture -- ``the practices, artifacts, sensibilities and ideas that constitute and inform our everyday lives" (30)
% * The ways we use and experience/interpret products are ``deeply cultural activities" (31)
% * ``Instead of looking at a small set of variables among a large number of people (the typical approach in survey research), ethnographers attempt to get a deep, detailed understanding of the life and circumstances of fewer people'' in order to extrapolate information about a larger culture (32)
% * The researcher captures as much detail as possible and carefully interpret its significance. ``Ethnography requires analytic rigour and process as well as inductive analysis (reasoning from the particular cases to the general theories)" (32)
% * Ethnography has only been used in design processes since the 1980s (probably at Xerox PARC) (36), but the author believes that ``design [would] benefit from the introduction of powerful social theory into its practice" (37)
% Martin, 2012:
% * Design ethnography is ``a descriptive account of social life and culture in a defined social system ... focused on a comprehensive and empathetic understanding of the users, their lives, their language, and the context of their artifacts and behaviors." Design ethnography approximates the immersion methods of traditional ethnography, to deeply experience and understand the user's world for design empathy and insight." Designers seek information from time-sampled observations -- much less immersed than true ethnographers. Analyses are ``built from deciphering patterns and themes emerging from research materials, and articulated in a set of design implications or guidelines in preparation for generative research and concept development" (60).
% * ``Surveys are a method of collecting self-reported information from people about their characteristics, thoughts, feelings, perceptions, behaviors, or attitudes." Like any self-reported data, however, the results may not be accurate (172).
% * ``Questionnaires are survey instruments designed for collecting self-report information from people about their characteristics, thoughts, feelings, perceptions, behaviors, or attitudes, typically in written form." They are frequently triangulated with other methods to verify or challenge other data (140).
% * ``Interviews are a fundamental research method for direct contact with participants, to collect firsthand personal accounts of experience, opinions, attitudes, and perceptions." In-person interviews provide the most information (102).

As a frequent concertgoer, I have been exposed to a wide variety of live music environments over the past decade. This personal experience has provided me with a significant understanding of audience members' perspectives. However, to dig deeper and explore others' viewpoints, I conducted a survey of music fans. An online questionnaire asked approximately one hundred regular concertgoers about their feelings towards audience participation, interacting with performers, and new technologies. My ethnographic study of performers was more exhaustive. Three experienced musicians were invited to participate in hour-long, semi-structured interviews. Subjects were asked about their backgrounds as performers and how they interact with their audiences; they were also asked to give their opinions on some of the case studies outlined in Chapter 2. These ethnographic studies helped me formulate possible solutions to my research problems, as I began identifying what types of interactions piqued the interests of concertgoers and performers.
% I posted a questionnaire for music fans; many people go to concerts frequently, and I was able to acquire over one hundred responses. To investigate the perspective of a performer, I conducted one-hour semi-structured interviews with three active musicians.
% My personal experience -- going to shows, putting on shows, stage managing, personal relationships with musicians -- shaped my perspective

\subsection{Prototyping}
Prototypes are tangible artifacts that are used to test a designer's ideas and gauge client and user responses. They can be of varying resolution; paper prototypes may aid with initial concept generation, for example, while more advanced iterations can help designers make final adjustments to near-complete products. By conducting user testing with prototypes, researchers can gauge their ``ergonomics, usability, aesthetic response, and emotional resonance" (Martin \& Hanington, 2012, p. 74).  This process is ideally iterative: a prototype is created, it is tested with users, and their feedback shapes the design of the next version.
% *Generation* *Evaluation*
% Martin, 2012:
% * ``Prototyping is the tangible creation of artifacts at various levels of resolution, for development and testing of ideas within design teams and with clients and users." ``Design prototypes are defined by their level of fidelity, or resolved finish." ``Low-fidelity prototypes are an excellent tool for the early testing of ideas with clients and users in generative research" (138).
% * ``Evaluative or evaluation research attempts to gauge human expectations against the designed artifact in question, determining whether something is useful, usable, and desirable ... gauge human factors and ergonomics, usability, aesthetic response, and emotional resonance." It is ideally iterative, allowing the product to be refined. Testing environment can either be realistic and difficult to control or artificial and controlled (74).

Three prototypes were developed and tested for this study. The prototypes' concepts were generated in order to answer the research questions, with certain design decisions influenced by the existing research, the results of the ethnographic study, and feedback from the previous prototypes. User testing was conducted for each iteration. The final prototype was created in collaboration with a local band and tested at one of their performances. By evaluating the reactions of both the performers and audience participants during these small-scale experiments, I was able to begin developing informed answers to the research questions.
% Prototyping: Prototypes will be tested with users. Observations and interviews with the participants will help answer research questions as well as validate or disprove any assumptions that have been made. Both audience and performer perspectives will be represented. This method will help establish design guidelines.