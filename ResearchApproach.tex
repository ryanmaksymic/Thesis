\chapter{Research Approach}

This chapter presents the primary and secondary research questions to be answered and the rationale behind them. The research design is described and the methods are explained and justified.

\section{Questions}

\subsubsection{Primary Question}

\begin{itemize}
	\item How might rock performances be made more participatory using digital technology?
\end{itemize}

\subsubsection{Secondary Questions}

\begin{itemize}
	\item \textbf{Output:} How -- if at all -- do audiences want to be involved in performances, and how much control do rock musicians feel comfortable giving up? What aspects of a performance can be reasonably controlled by a large number of people?

	\item \textbf{Input:} How can inputs from each audience member be captured and processed? Which mapping strategies are effective?

	\item \textbf{Feedback:} How can users be informed that their input has been received and had influenced the system?

	\item \textbf{Experience:} When such a system is implemented, does it actually improve the concert experience for both the audience and performer?
\end{itemize}


\section{Rationale}

% Background; participatory culture:
As we have seen, performance responds to the culture it exists within. For primitive societies, music is purely participatory, more functional than entertaining. Presentational performance eventually emerged as a way for the middle class to flaunt their newfound wealth. It morphed into a visual spectacle when mass media created a visual culture and MTV made music and images inseparable. Today, there is a participatory culture forming, and I believe that live performances will continue changing to reflect this. The next section presents research that investigates audience participation and new technologies that facilitate it.
% Take out the ``I believe" part. Don't make any hypotheses here.
% The word `primitive' should be qualified or replaced

% A young research field:
This research question is not only relevant culturally, but technologically as well. The ubiquity of public displays, connected personal devices, and movement sensing technology means crowd-based interactions are becoming increasingly practical (Brown et al., 2009). Indeed, human-computer interaction (HCI) researchers are only recently beginning to investigate how systems might be designed for a large assembly of users. In addition to investigating crowd-based interfaces (Maynes-Aminzade et al., 2002; Feldmeier \& Paradiso, 2007), researchers have been asking how these systems should be implemented when a performer is introduced (Gates et al., 2006; Barkuus \& J{\o}rgensen, 2008). This work has only looked at dance performances, rap competitions, and nightclubs, however; there is insufficient research on incorporating these systems into rock performances.

% Artists are recently showing interest:
Despite the lack of research, many inquisitive rock musicians are already showing interest in participatory technologies. Groups like Coldplay\footnote{\url{http://xylobands.com}} and Kasabian\footnote{\url{http://nanikawa.com/projects/kasabian-tour-2011-interactives}} have experimented in recent years with new ways to involve audiences in their performances. These technologies have been appearing at large outdoor music festivals and incorporated into performances streamed online where anyone with an internet connection can get involved.
% Participation is limited. How can we amplify audience input?

% Live music is a growing industry:
Lastly, it should be noted that the live music industry is growing rapidly (Wikstr\"{o}m, 2013). The digital revolution caused record sales to plummet, whereas live music revenues are larger than ever. While touring used to be a method for promoting recorded music, the opposite is now true. Wikstr\"{o}m speculates that ``live music will soon dominate the entire music industry in the same fashion as recorded music has done during more than half a century" (p. 142-143). Thus, researching new methods for creating impactful live music experiences is a good investment.

% A design problem:
Output, input, feedback, and experience...
The goal is to make a rock show that takes a step towards participatory.
% Remaining questions from existing work:
% * Collaborative or individual? Similarly: Fully representative or partly fabricated? (Maynes-Aminzade, et al.)
% * Gates: DJs don't really want audience-interaction technology. How do rock musicians feel? Should participatory technologies help the musicians ``do their job"?
% * Barkhuus and Jorgensen did not involve the performers at all
% * Gates: Gradual changes are more satisfying than immediate changes


\section{Methods}

In order to understand participatory technologies, I created some myself. Multiple interactive systems were developed in order to explore the research questions, implementing a user-centred design process. The ISO standard Human-centred design for interactive systems (ISO 9241-210, 2010) outlines some key principles that will ensure a design process is user-centred. These principles include the following: understand the context of use -- the users, their environment, and the tasks they perform; utilize iterative processes; and ensure that users are involved in all design phases. These are reflected in the design research methods that were employed -- a combination of ethnographic study and prototyping. 

\subsection{Ethnography}
Exploration: Ethnography is the exploration of a cultural phenomenon...
% Ethnography: Examining cultural phenomena within a group. Surveying modern music fans will provide me with a general sense of their feelings towards music, live performance, and technology. I will interview active musicians to understand how and why they interact with fans, on and off stage, and what they think of new technologies in a performance setting. These methods will allow me to gain a deeper understand the users and environments that are related to my work.
% Plowman, 2003:
% * Ethnography originated in anthropology, the study of culture -- ``the practices, artifacts, sensibilities and ideas that constitute and inform our everyday lives" (30)
% * The ways we use and experience/interpret products are ``deeply cultural activities" (31)
% * ``Instead of looking at a small set of variables among a large number of people (the typical approach in survey research), ethnographers attempt to get a deep, detailed understanding of the life and circumstances of fewer people'' in order to extrapolate information about a larger culture (32)
% * The researcher captures as much detail as possible and carefully interpret its significance. ``Ethnography requires analytic rigour and process as well as inductive analysis (reasoning from the particular cases to the general theories)" (32)
% * Ethnography has only been used in design processes since the 1980s (probably at Xerox PARC) (36), but the author believes that ``design [would] benefit from the introduction of powerful social theory into its practice" (37)
% Martin, 2012:
% * Design ethnography is ``a descriptive account of social life and culture in a defined social system ... focused on a comprehensive and empathetic understanding of the users, their lives, their language, and the context of their artifacts and behaviors." Design ethnography approximates the immersion methods of traditional ethnography, to deeply experience and understand the user?s world for design empathy and insight." Designers seek information from time-sampled observations -- much less immersed than true ethnographers. Analyses are ``built from deciphering patterns and themes emerging from research materials, and articulated in a set of design implications or guidelines in preparation for generative research and concept development" (60).
% * ``Surveys are a method of collecting self-reported information from people about their characteristics, thoughts, feelings, perceptions, behaviors, or attitudes." Like any self-reported data, however, the results may not be accurate (172).
% * ``Questionnaires are survey instruments designed for collecting self-report information from people about their characteristics, thoughts, feelings, perceptions, behaviors, or attitudes, typically in written form." They are frequently triangulated with other methods to verify or challenge other data (140).
% * ``Interviews are a fundamental research method for direct contact with participants, to collect firsthand personal accounts of experience, opinions, attitudes, and perceptions." In-person interviews provide the most information (102).

The context of this research is very particular. There are two different user roles -- audience members and performers. A participatory system will serve different functions for both roles but in doing so must enhance the live music experience for all users. Furthermore, the environment of a live rock performance is very unique.

I posted a questionnaire for music fans; many people go to concerts frequently, and I was able to acquire over one hundred responses. To investigate the perspective of a performer, I conducted one-hour semi-structured interviews with three active musicians.
% Mention how your personal experience -- going to shows, putting on shows, stage managing, personal relationships with musicians -- shape your perspective

\subsection{Prototyping}
Generation, and evaluation: I developed three prototypes. The prototypes' concepts were generated in order to answer the research questions, with certain design decisions influenced by the results of the ethnographic study as well as the literature review. Each prototype was tested with groups of users. The last prototype was implemented at a real-world concert; I worked closely with a local band to develop the prototype, and audience reactions were closely observed.
% Prototyping: Prototypes will be tested with users. Observations and interviews with the participants will help answer research questions as well as validate or disprove any assumptions that have been made. Both audience and performer perspectives will be represented. This method will help establish design guidelines.
% Martin, 2012:
% * ``Evaluative or evaluation research attempts to gauge human expectations against the designed artifact in question, determining whether something is useful, usable, and desirable ... gauge human factors and ergonomics, usability, aesthetic response, and emotional resonance." It is ideally iterative, allowing the product to be refined. Testing environment can either be realistic and difficult to control or artificial and controlled (74).

Due to the need for research involving human participants, Research Ethics Board approval was obtained (see Appendix A).