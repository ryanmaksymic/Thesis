\chapter{Conclusion}

This chapter summarizes the results of the research that was carried out and their implications with regard to the research questions. Lastly, possible directions for future research are discussed.

\section{Discussion}

% How might rock performances be made more participatory using digital technology?
% * Output: How -- if at all -- do audiences want to participate in rock performances? How much control do rock musicians feel comfortable giving audience members? What aspects of a live performance can be reasonably controlled by a large group of people?
% * Input: How can inputs from the audience be captured and processed? Which mapping strategies can effectively translate these inputs to the output?
% * Feedback: How can users be informed that their input has been received and has influenced the system?
% * Experience: Does the participatory technology actually improve the concert experience for both the audience and performer?

% Hardware and software used was effective...

% Participatory culture...?

% Neither fans nor artists are generally comfortable with audiences participating in primary output. Creating a new mode of output is a solution.

% Precedent projects (Freeman, one of the interactive dance clubs) have built-in reward systems -- more activity rewarded with more control. Could this be explored?

% Having a central focus -- the performer -- decreases interactions amongst the audience. Thus, participation is more a one-to-one connection than a collaborative experience. Could this change at a larger venue when there is more separation? There's more chatter at arena shows than coffee shop shows.

% ``In the theatre the audience wants to be surprised -- but by things that they expect" (Tristan Bernard). Prototype #3 had no surprises in store after the initial concept was understood. But Turino says there is ``security in constancy" in participatory performances...

% Balances:
% * Freedom & Constraint
% * Simplicity & Performativity
% * 

This thesis investigated how digital technology might be used to introduce new forms of audience participation at rock concerts. A user-centred design process was implemented, breaking the research question down into modular problems -- investigating the input, feedback, output, and overall experience of a participatory technology. Ethnographic studies were conducted, followed by three phases of prototyping. These research methods proved to be effective tools. Obtaining a deeper understanding of both concertgoers' and performers' perspectives provided a solid foundation from which the prototypes could emerge. Each prototype then managed to address the research questions in different ways, building off of previous iterations' findings. Talking to users -- both audience members and performers -- provided valuable insight into how the prototypes had performed.

A number of major themes emerged upon analyzing this work, yielding numerous possible solutions to each of the research questions.

\subsubsection{Experience}

It is clear that participatory technologies are of interest to both performers and music fans. While it is to be expected that not all musicians will agree, those interviewed expressed curiosity towards experimenting with new methods of audience participation. Christian Hansen is an active member of the participatory culture and is especially open minded. After experiencing the third prototype firsthand, he was interested in developing the technology further. In particular, he believed that the system could become a useful storytelling device. Concertgoers surveyed in the ethnographic study did not voice a need for a new method of participation, but they were generally interested in how new technologies could be incorporated into shows. According to the questionnaire, music fans that frequently go to small-sized concerts are more inclined to engage in participatory culture. Indeed, those that attended the relatively small show during which Prototype \#3 was tested were eager to grab a controller. These users all responded positively to being in control of the lights, and many became physically engaged in the performance. Thus, participatory technologies have the potential to enhance live music experiences for both performers and audiences.
% Questionnaire:
% * Small venues seem to attract people that enjoy participating, are interested in technology, etc.
% * People don't really want a new way to participate
% Interviews:
% * A predictable yet notable result of this study was that all musicians are different. The performers interviewed all had substantial experience performing, for instance, but their interactions with audiences differed considerably.
% * All noted that context -- venue size, audience makeup -- is important and can change how decisions are made
% * Some performers are distracted by audience involvement
% * A participatory technology should reflect that every show is unique
% * Alcohol is an important factor in determining a crowd's behaviour
% Prototype #1:
% * People are interested in participating in performances
% Prototype #2:
% Prototype #3:
% * How could different types of interactions be incorporated into the length of a performance? The performer could provide instructions, but is it prohibitive to apply rules to the users? Turino suggests that constancy is a main feature of participatory performance.
% * Alcohol didn't have any noticeable negative effects here

\subsubsection{Input}

Multiple methods of input were analyzed to find how to best retrieve information from audience members. While their buttons caused some distraction, the Wii controllers were mostly flexible and problem-free input devices. As related work (Ulyate \& Biancardi, 2001; Maynes-Aminzade et al., 2002) and the ethnographic study indicated, movement-based inputs were welcomed by users. Previous research more specifically recommended audience-controlled systems be based on already-present behaviour (Barkhuus \& J{\o}rgensen, 2008). Methods like clapping and doing the wave certainly were intuitive for users testing the second prototype. However, those participating in the third and final experiment exhibited more abstract motions. It became clear that allowing for creativity yielded the best results. Giving the input device an ambiguous form let participants hold it and move with it in the way that came naturally to them. Mirroring Turino's (2008) concept of ``wide tuning," robust programming was also important to ensure that, if the device was used in an unexpected way, the system still responded correctly; users holding their Wii controller sideways, for example, were still able to illuminate their light. Other precedent research investigated promoting collaboration in groups (Freeman, 2005; Feldmeier \& Paradiso, 2007), and Prototypes \#1 and \#2 confirmed that collaboration can occur naturally in interactive systems. Prototype \#3 conversely isolated each user's inputs and instead connected them to their own object on stage. While this did not facilitate collaboration within the crowd, a satisfying connection between the audience members and performers instead emerged.
% Wii controllers were a fine choice
% Questionnaire:
% * Music fans like to move
% Interviews:
% * Performers aim to raise the energy of the audience and feel that this tends to increase the overall quality of the show
% * People should be able opt out of the interaction without affecting others
% * Smartphones could be distracting, and people may drop them
% Prototype #1:
% * Collaboration happened naturally
% * A mute button is useful -- regulated audience input
% * Decision based inputs could create a goal-oriented environment instead of a creative one
% Prototype #2:
% * The buttons were misleading
% * On/off is most appealing. Choosing between two options caused confusion (i.e. no more left/right voting). This is just like regular audience behavior I guess?
% * Collaboration came naturally and was enjoyable. Many users counted time.
% * Users should be able to opt out without affecting others' experiences (i.e. not Thumbs up/down or Clap-o-meter)
% * No forced goals (i.e. no prompts to perform in sync). Allow for creativity.
% * Be clear about the rules of the system or users will begin experimenting and become distracted
% Prototype #3:
% * The ambiguous form of the device gave people options on how they'd like to use it
% * Audience members did not make an attempt to work together. Was this due to attitude or size of the crowd? The fact that a performer was present? A limitation of the technology? Does the one-to-one mapping isolate each user despite the lights being grouped together?
% * This did not really allow for opting out. If half of the participants stood still, for example, would this lessen the effect too much? How can this be remedied?

\subsubsection{Feedback}

Great importance was placed on immediate and obvious feedback in past work (Ulyate \& Biancardi, 2001; Barkhuus \& J{\o}rgensen, 2008). The first two prototypes confirmed that this was necessary in order for users to easily understand the effects of their inputs. Users did not want their inputs combined with others', and they preferred looking at direct representations of their actions over obscured visualizations. Prototype \#3 provided direct feedback to each user individually. The twelve users were able to easily identify the light under their control. This simple one-to-one mapping was the most direct way to avoid the uncertainty that could arise in a many-to-one system, such as in Prototype \#1. The binary, on/off functionality also got rid of the goal-oriented nature created by the first prototype's left-versus-right voting mechanism. While the third prototype made use of LEDs on the controllers for additional visual feedback, this was rendered meaningless to users who held the devices in an unexpected orientation. Thus, it is important that feedback systems are not based on assumptions about user behaviour.
% Maynes-Aminzade, 2002: Every audience member does not necessarily need to be sensed as long as they feel like they are contributing. Really?
% Questionnaire:
% Interviews:
% Prototype #1:
% * Many-to-one mapping sucks for feedback
% Prototype #2:
% * Preference for individual visualizations for each user (i.e. not Thumbs up/down or Clap-o-meter)
% * Direct control of visual elements trumps indirect control (i.e. ditch the video crossfading)
% * The output must be more captivating than the input (i.e. no digital wave)
% Prototype #3:
% * Direction-dependent LEDs were rendered meaningless when users flipped the devices upside down or sideways

\subsubsection{Output}

Freeman (2005) successfully gave an audience control over an orchestra's musical output, but, after speaking with musicians, it became obvious to me that this model would not work for a rock concert. A typical performer does not want to give their audience too much control. Similar to the PixMob and Wham City Lights projects, Prototype \#3 successfully made the audience a part of the performance by letting the control the light show. In addition to engaging audience members, the performer felt that this complemented the music without being a distraction. A particularly pleasing outcome came from where the audience's and performer's influence intersected. The final prototype's lamps became communal interfaces, being physically handled by the performers yet illuminated by audience members. In Bongers' (2000) words, this turned the system from ``reactive" into ``interactive." Perhaps a participatory technology is most effective when this threshold is crossed; `true interactivity' may be a requirement for a truly participatory performance. Allowing an audience and performer to participate in the same performance, then, is not just accomplished by providing the crowd with a new output, but by augmenting the performer's output as well.
% Bongers, 2000: Interactive vs Reactive
% Interviews:
% * There was hesitance about giving the audience too much control
% * The projects impressed all of the artists, though certain aspects raised concerns for each. Spectacle could become more important than musical performance.
% * The technologies may be making up for the distance separating performer and audience
% Prototype #1:
% * A participatory system should probably only influence secondary output in a rock concert setting
% Prototype #2:
% Prototype #3:
% * The lamps were a communal instrument, `played,' albeit in different ways, by both the audience and the performer. Performers are controlling secondary output (lighting). Reactive became Interactive (Bongers).


\section{Future Directions}

% Specifically discuss the next prototype. Are you going to continue working with Christian Hansen?

Future work will advance the last prototype and address questions unanswered by this thesis. Continuing the collaboration with Christian Hansen, the next iteration will be implemented for a longer portion of a performance. The technology's parameters will change as the show progresses, introducing new forms of interaction that reflect the flow of the performance. This could be facilitated by the performers utilizing the lamps in new ways, or the system could activate new rules or features, such as lights of different colours. Will this variety of participatory forms keep the audience engaged, or will the continual changes frustrate and confuse them? Another improvement that must be made is to allow for more audience members to be involved. An increasing number of users, however, will eventually make the current prototype's one-to-one mapping system ineffective; a new method of feedback will have to be formulated. Future iterations will also be able to adapt to individuals' behaviours. A listless user and one that is moving frantically, for instance, should not be producing the same type of output. If some users are not participating at all, the system should reassign their bandwidth rather than letting it go unused.

New technologies will also be investigated. Bluetooth low energy (BLE) is a recently introduced standard that will make input devices smaller and lighter with a long-lasting battery. The LightBlue Bean\footnote{\url{http://launch.punchthrough.com}} makes use of BLE and would serve as a flexible prototyping platform. These devices can be networked together and allow data to flow between a scalable group of users.

Looking further ahead, additional uses of participatory technologies will be investigated. The data being provided by audience members, for instance, could be collected and analyzed. Perhaps examining audience activity throughout a concert could help performers understand the responses to particular songs and inform how they organize their sets. With the world of rock music thoroughly explored, experiments will also be run within other genres of music. How would a participatory technology fare at a jazz club? If enough interest is shown in the system, a market study will be performed to evaluate its viability as a commercial product. Artists and promoters may be interested in purchasing participatory technologies, prepackaged or personalized to their needs, in order to add a special element to their live show.

% Interview lighting designers about the topic, treating them as performers
% A dynamic system that allows for performance-long audience participation but changes the rules periodically to maintain their interest
% Use colour to indicate associations or rule changes
% User testing can involve those present but not participating. How was the experience changed for them?
% How can output reflect audience involvement? For example, an enthusiastic user's light may overtake that of a user who is not moving
% What changes when the venue becomes large -- for the performer, the audience, and the technology?
% Data collection; artists can see how audience activity changed over the course of a show

\section{Conclusion}

This thesis showed that simple digital technologies can give audiences enjoyable new methods for participating in rock performances. While presentational performances will always divide audience and performer, promoting participation reinforces one of music's most important functions -- bringing people together.
% Rewrite all of this. Way too broad. Keep it in the scope of your work: ``I have taken the first step toward making traditional rock shows more participatory."
% Tie it all up in a bow. Did you answer your questions?