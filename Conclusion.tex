\chapter{Conclusion}

This chapter reviews the results of the research that was carried out and summarizes how it answers the research question and what uncertainties remain. Lastly, some possible directions for future research are discussed.

\section{Discussion}

This thesis investigated how digital technology might be used to introduce new forms of audience participation at rock concerts. Ethnographic studies were performed, and user-centred design was implemented through three phases of prototyping. A number of major themes emerged upon analyzing this research, taking significant steps towards answering the research questions.
% When such a system is implemented, does it actually improve the concert experience for both the audience and performer?
% How can inputs from each audience member be captured and processed?
% How -- if at all -- do modern audiences want to be involved in performances, and how much control do today's rock musicians feel comfortable giving up?
% What aspect of the performance can be reasonably controlled by a large number of people?

% Hardware and software used was effective...

% Participatory culture...?

% Neither fans nor artists are generally comfortable with audiences participating in primary output. Creating a new mode of output is a solution.

% Precedent projects (Freeman, one of the interactive dance clubs) have built-in reward systems -- more activity rewarded with more control. Could this be explored?

% Having a central focus -- the performer -- decreases interactions amongst the audience. Thus, participation is more a one-to-one connection than a collaborative experience. Could this change at a larger venue when there is more separation? There's more chatter at arena shows than coffee shop shows.

% ``In the theatre the audience wants to be surprised -- but by things that they expect" (Tristan Bernard). Prototype #3 had no surprises in store after the initial concept was understood. But Turino says there is ``security in constancy" in participatory performances...

% Tie it all up in a bow. Did you answer your questions?

Firstly, it is clear that participatory technologies are of interest to both performers and music fans. While it is to be expected that not all musicians will agree, those interviewed expressed curiosity towards experimenting with new methods of audience participation. Christian Hansen was especially open minded and, after the testing of the third prototype, was interested in developing the technology further. In particular, he believed that the system could become a useful storytelling device. Concertgoers surveyed in the ethnographic study were generally interested in how new technologies could be incorporated into shows -- especially those who frequently attend performances at smaller venues. Audience members who used the final prototype all responded positively to being in control of the lights, and many became physically engaged in the performance.

Multiple methods of input were analyzed to find how to best retrieve information from audience members. Previous research recommended audience-controlled systems be based on already-present behaviour. While methods like clapping and doing the wave were intuitive for users testing the second prototype, those participating in the final experiment improvised more abstract motions. It became clear that allowing for creativity yielded the best results. Giving the input device an ambiguous form lets participants use it in the way that comes naturally to them. Robust programming is also important to ensure that, if the device is used in an unexpected way, the system still responds correctly; users holding their Wii controller sideways, for example, were still able to illuminate their light.
% Wii controllers were a fine choice

The final prototype provided direct feedback to each user individually. Twelve users were able to easily identify the light under their control. This simple one-to-one mapping was the most direct way to avoid the uncertainty that could arise in a many-to-one system, such as in Prototype \#1. The on/off functionality also got rid of the goal-oriented nature created by the first prototype's voting mechanism. While those testing the earlier prototypes were eager to collaborate with those around them, users of Prototype \#3 seemed content performing with their isolated light bulb. It could be that the `unity' that was hoped to emerge from this technology was not among all users but between individual users and the performance.

A surprising outcome came from where the audience's and performer's influence intersected. The final prototype's lamps became communal interfaces, being physically handled by the performers yet illuminated by audience members. This left the primary output of the performance -- the music -- fully in the band's control and facilitated a new kind of interaction between artist and audience using the performance's secondary output -- light. Allowing an audience and performer to participate in the same performance, then, is not only accomplished by augmenting the crowd's abilities, but by augmenting the performer's responsibilities as well.
% Reference Bongers


\section{Future Directions}

% Specifically discuss the next prototype. Are you going to continue working with Christian Hansen?

Future work will advance the last prototype and attempt to answer new questions that were raised by this thesis. The next iteration will be implemented for the duration of an entire performance. The technology's parameters will change as the show progresses, introducing new forms of interaction that reflect the flow of the performance. This could be facilitated by the performers utilizing the lamps in new ways, or the system could activate new rules or features like lights of different colours. Will this variety of participatory forms keep the audience engaged, or will the continual changes frustrate and confuse them? Another improvement that must be made is to allow for more audience members to be involved. An increasing number of users, however, will eventually make the current prototype's one-to-one mapping system ineffective; a new method of feedback will have to be formulated. Future iterations will also be able to adapt to individuals' behaviours. A listless user and one that is moving frantically, for instance, should not be producing the same type of output. If some users are not participating at all, the system should reassign their bandwidth rather than letting it go unused.

Looking further ahead, new technologies will be investigated. Bluetooth low energy (BLE) is a recently introduced standard that will make input devices smaller and lighter with a long-lasting battery. The LightBlue Bean\footnote{\url{http://launch.punchthrough.com}} makes use of this technology and would serve as a flexible prototyping platform. These devices can be networked together and allow data to flow between a scalable group of users.

Additional uses of these technologies will be investigated. This data being provided by audience members, for instance, could be collected and analyzed. Perhaps examining audience activity throughout a concert could help performers understand the responses to particular songs and inform how they organize their sets. With the world of rock music thoroughly explored, experiments could also be run within other genres of music. How would a participatory technology fare at a jazz club? If enough interest is shown in the system, a market study will be performed to evaluate its viability as a commercial product. Artists and promoters may be interested in purchasing participatory technologies, in a box or personalized to their needs, in order to add a special element to their live show.

% A dynamic system that allows for performance-long audience participation but changes the rules periodically to maintain their interest
% Use colour to indicate associations or rule changes
% User testing can involve those present but not participating. How was the experience changed for them?
% How can output reflect audience involvement? For example, an enthusiastic user's light may overtake that of a user who is not moving
% What changes when the venue becomes large -- for the performer, the audience, and the technology?
% Data collection; artists can see how audience activity changed over the course of a show
% Interview lighting designers about the topic, treating them as performers

\section{Conclusion}

This thesis showed that simple digital technologies could give audiences and rock musicians new ways to perform with each other. While the divide between performer and audience will always be present at live performances, promoting participation reinforces music's most important function -- bringing people together.
% Rewrite all of this. Way too broad. Keep it in the scope of your work: ``I have taken the first step toward making traditional rock shows more participatory."
