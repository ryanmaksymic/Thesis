\chapter{Ethnography}

One of my first goals was to get a sense of modern concertgoers' and performers' feelings about participatory performances and interactive technology. I sent out a brief online survey for music fans that helped me to understand how they generally responded to these topics. Interviews were also conducted with multiple musicians to shed light on their perspectives.
% Talk more about motivation, participatory culture

\section{Audiences}

% Explain the content of the survey:
An online survey was created in order to obtain a sample of modern music fans' opinions on interactive performances. The survey was completed by ninety-nine participants recruited through links posted on several social media websites. The first few questions informed me of what type of concertgoer each participant was -- asking their favourite genre, the size of the venues they frequent, and how often they attend live music performances. I also asked how often the participants communicate directly with musicians through their social media presences. Next, the survey focused on concert behaviours. Participants were asked in which actions they typically partake at live music performances; choices included applauding, headbanging, and holding up lighters. They were asked how they might like to participate in a performance and what sort of message they would send their favourite performer if they could. I asked for their thoughts on getting involved in performances, bringing new technologies into concert settings, and interacting with musicians using social media services.
% Explain the details of how participants were recruited, how data was collected, recorded, and processed/analyzed, and the timeline over which this process unfolded
% Add raw data to an Appendix and refer to it here

% Present the direct results:
The results are not shocking but certainly informative. Most participants favour rock music or some variation (``indie," ``alternative"); the majority attend multiple concerts per year -- some on a weekly basis; and most usually go to shows at small- to medium-sized venues. The majority of participants claimed to communicate with artists through social media either occasionally or regularly, though a sizeable amount indicated they never do this. The most popular concert actions are applauding, singing along with the performer, clapping or stomping to the beat, and dancing. The less-common actions include holding up signs, lighters, or illuminated electronic devices. When asked how they might want to participate in a performance in a new way, many said they would like to choose the songs that are played, while much fewer expressed interest in manipulating visuals and contributing to the music; around one quarter of participants stated they did not have interest in directly influencing a live music performance at all. Given the opportunity to communicate with their favourite performer, most participants responded with praise or appreciation (``Thank you," ``I love you!"). Other messages included song requests and suggestions like, ``Don't bury the vocals," or ``More rock, less talk!" The majority of participants indicated that they enjoy when performers ask them to participate in a performance -- clapping or singing along or call and response, for example. Lastly, the majority also said they were excited by the idea of bringing new technologies into a live music setting.
% Include percentages

% Present the results of the in-depth analysis:
Upon further analysis of the responses, some correlations were uncovered. There are clear relationships between show-going frequency, typical venue size, and interest in interaction and technology. Participants attending shows more frequently are more likely to visit smaller venues. This group also expressed the most interest in being involved in performances; they are more inclined to interact with their favourite artists via social media; and they are more welcoming to the idea of unfamiliar technology in a concert setting. The opposite, thus, can also be said: participants who go to fewer shows tend to go to larger venues, are more likely to refrain from participating in shows, are less likely to contact artists through social media, and are less interested in new technologies.

\subsection{Analysis}

These results give some insight into how audiences feel about participating in performances. Respondents enjoy dancing, clapping, and singing along with live music, but very few are interested in new ways to contribute to the performance. Concertgoers seem satisfied with the current audience-performer relationship. Their interest in new technologies, however, could indicate a willingness to experiment. This experimentation seems to be best suited for small venues. Music fans who frequent these venues are more inclined to participate in performances and more interested in new technologies, and their relatively high social media activity means they are already members of Wikstr\"{o}m's (2013) ``participatory culture." Also noteworthy is that most popular audience actions tend to involve consistent movement. Enabling motion-based inputs would likely receive a better response than something button based, for example.

% Survey Outcomes:
% * 


\section{Performers}

% Explain the recruitment process and describe the participants:
With a broad overview of audience attitudes, the next step was to speak directly with actual performers and discuss the same topics with them. Three musicians were recruited: Erik Grice, Blake Enemark, and Christian Hansen. These three were selected because I was familiar with their bands and felt that they represented three distinct performance styles within the rock genre. The subjects have different backgrounds as performers, experience playing in different parts of Canada, and for a range of audience sizes. They also have slightly different relationships with modern technology. While their opinions are representative of modern rock musicians, it should be kept in mind that the subjects are music fans who enjoy watching others perform as well; thus, they are also sharing their perspective as concertgoers.

% Outline the interview questions
After briefly establishing their history as performers, I asked them each about what audience participation means to them. The musicians were shown video of some of my case study subjects (including Xylobands, Wham City Lights, and Kasabian's 2011 tour), and I recorded their reactions and general thoughts on technology-enabled performances. Lastly, the artists were asked if and how they would want to incorporate similar participatory technologies into their own shows. Each interview lasted around one hour.

\subsection{The Subjects}

\subsubsection{Erik}
Erik Grice grew up outside of Edmonton, Alberta. He performed musical theatre as an adolescent and started playing in bands in his teens. After acting as a vocalist and guitarist in previous bands, Erik now plays drums for Edmonton-based The Fight -- an alternative rock band with soul undertones. The Fight's typical audience size can range from twenty at tour shows to over one hundred people in Edmonton. Venues are usually small clubs or halls. They have also played acoustic shows at cafes or similar venues. A typical Fight concert is energetic yet composed, and audience members can comfortably move to the music.

\subsubsection{Blake}
Blake Enemark is originally from Victoria, British Columbia. He learned to play guitar as a teenager and started performing cover songs in bands. Blake recorded and toured with his alt-country band Forestry through 2010. Shortly after, he was recruited to join We Are The City, a Vancouver based progressive rock band who had just gained nation-wide recognition after winning the \$150 000 Peak Performance award, a revered radio station contest. With this group, Blake played to audiences more sizeable and fanatic than he had ever before, culminating in a performance for around two thousand people in Vancouver's Stanley Park. Blake parted ways with We Are The City after one year. He joined folk group Northcote for a Canadian tour before settling down in Toronto and starting his own project called Snoqualmie. Snoqualmie, described as ``high-gain, sad-sack Canadiana," marks a return to songwriting and more intimate shows for Blake.

\subsubsection{Christian}
Christian Hansen has a theatre background, a graduate of the University of Alberta's acting program. He began playing in bands as a teenager. During his university degree, Christian rediscovered his desire to perform music and began playing acoustic shows. He eventually started performing with his now-wife Molly and began incorporating drum machines and prerecorded tracks into his work, which was becoming more like dance music. Christian Hansen \& The Autistics were formed. When Molly could not make it to one show, Christian was inspired to put all of his energy into his performance to make up for it; this was a ``lightbulb moment for him." Christian Hansen \& The Autistics soon gained notoriety around Edmonton for their high-energy shows, and their songs received a lot of radio play. Christian and Molly moved to Toronto in 2011, shortening their band name to Christian Hansen and now playing music described as new wave. While they are currently working their way into Toronto's music scene, Christian Hansen typically draw around five hundred excited showgoers when they play in Alberta.

\subsection{Interacting With Audiences}

\subsubsection{Erik}
The Fight encourage some forms of participation at their shows. Their lead singer can often walk out and physically touch audience members, looking them in the eye as he sings. They will invite the crowd to sing and/or clap along for suitable songs. Erik felt that this sort of participation typically makes for better shows: ``An attentive and participating crowd of fifty people is always going to be better than two hundred people who are standing there with blank faces." He explained that the ideal audience will match the energy that the band exudes. In addition to making the show more enjoyable, he said, this also makes musical flubs less significant to everyone present. Erik felt that it is the band's job to keep the audience's attention. They must make decisions based on the venue, the audience, the length of the set. Stage banter is usually only implemented to convey pertinent information to the crowd. Talking to audience members after the show has gotten the band extra shows, radio spots, places to stay for the night.

The Fight make use of popular social media channels. They use Facebook to advertise, organize contests and giveaways, and share information on their live dates. Twitter is used for communicating with fans and other artists. This has helped them open slots for touring bands. At shows, the band hands out cards with links to their social media pages. Erik believed there is a levelling out between artists and fans, and he likes this. He explained that some contemporaries try to maintain personas and seem ``untouchable": ``It's stupid to have those kinds of pretences," he said.

\subsubsection{Blake}
Blake is a self-described introvert. While he is technically the leader of his current band, he expressed a preference for playing in the background. Despite being typically shy with audiences, Blake recognized the significance of even basic audience participation. Singing and clapping along makes you feel like a part of the show. He also expressed the impact this might have on performers; ``It would be the most flattering thing in the world for me for someone to sing my song back to me," he said. Blake noted, however, that different performers react to audience participation differently. A guitarist in his former band, for instance, would not allow the audience to clap along. Furthermore, ``There's a fine line," Blake stated, ``between drunken participation and intentional participation." He acknowledged the prominence of alcohol in live music settings and the importance of alcohol sales at most venues. It is a part of the industry, he said, and it plays a role in how audiences behave.

The internet presence of Blake's current band is ``not very good." Blake himself recently closed his Facebook account for personal reasons. Despite this, he acknowledged the importance it holds for artists: ``It's bridging gaps that have never been bridged before." He had an especially meaningful experience with the Reddit community; after an anonymous user posted a link to his music, Blake began receiving orders from all over the world -- the southern United States, Poland, Japan. ``You never know who's listening," he said, ``It's empowering and terrifying."

\subsubsection{Christian}
Christian felt that it is the responsibility of the performer alone to ensure a concert is enjoyable. If he puts all of his energy into a show, he explained, all he can hope is that the audience reciprocates: ``If we come in at 200\%, maybe the audience will get to 100\%." Christian emphasized the importance of responding to the audience. If they are standing far from the stage, he will encourage them to get closer. If certain individuals are especially invested in the show, he will make eye contact with them and sing directly to them. Christian encourages singalongs and will sometimes hold the microphone in front of those who are singing loudly. He may even leave the stage and make physical contact with the crowd if they seem particularly comfortable. At their most recent Edmonton concert, Christian entered the audience and performed the last song unplugged, the crowd surrounding him and singing along with him. For him, this moment was ``amazing, magical." While he acknowledged that every audience is different, Christian felt that getting the audience involved generally increases the intensity of a show.

Christian embraces connecting with fans through social media. He accepts friend requests from fans on his personal Facebook account and does his best to respond to every message he receives. Despite a few negative online experiences, Christian enjoys interacting with fans in this way.

\subsection{Participatory Technologies}

At this point in the interview, the musicians were shown images of some of the projects that were outlined in Chapter 2. These included PixMob, Xylobands, Wham City Lights, Plastikman's SYNK, and the Amex Unstaged: Usher project.

\subsubsection{Erik}
Erik responded negatively to the projects that relied on smartphones; he felt they would be mostly distracting, and he expressed concern about possible being responsible for people dropping and damaging their devices. ``What can't your phone do now?" he asked. Rather, he favoured the work that had other tangible elements, like the PixMob beach balls. Erik felt that every show should be unique. An attendee should be able to go home and say, ``I was at that one" -- a digital analogue to the classic concert tee with dates listed on the back. Some artists post photos and set lists from each of their shows, he explained, and even this makes a show feel personal for those who attended. If the set changes slightly each night, this amplifies the effect, as well as keeping things interesting for the performer. Erik felt that these sorts of technologies should allow people to opt out without affecting the others' experience. However, he also expressed concern about giving power to all audience members. Something like a Twitter feed is surely edited. Bands that hand out percussion instruments like tambourines have to deal with participants with no rhythm. Erik provided an anecdote about a band that gave miniature harmonicas to audience members to be played during one part of one song; the crowd continued to play the instruments throughout the whole set and the other bands' sets.

\subsubsection{Blake}
Blake commented on the importance of context. An experimental performance may only be successful if the crowd is filled with fans of the artist. He felt that a festival-type setting might be more suitable than a small rock club. Blake remarked on the effectiveness of these projects with large audiences. He recounted an experience seeing U2 perform; the stadium lights were extinguished, and the crowd was instructed to hold up their open cellphones, filling the space with an electric glow. While Blake was concerned that the technologies I showed him may border on gimmicky, he admitted that creating a spectacle has become a significant part of performing: ``You gotta have something that's more than the music," he said, ``You can't be like the Beatles anymore and just record albums and be successful. There has to be an angle." Although Blake admitted the importance of creating a memorable experience for concertgoers, he seemed to lament the current attitude towards live music: ``A lot of people are just there to have a good time. And if you can make them have a good time then you're a good musician. It's a little discouraging." He was also wary of the amount of video recording at modern concerts, explaining that something is lost in a recording. ``A concert's an experience," he said, ``Go and soak it in and remember it and let it sit in your memory."

\subsubsection{Christian}
Christian admired the way the projects all seem to aim to ``unite" the participants; ``There's not a lot of times when we feel that we're united," he said. He also remarked on the size of the audience in the examples and wondered if similar effects could be replicated with smaller venues and smaller budgets. Christian speculated that perhaps these technological spectacles are especially useful at large shows because the performers are so distant from audience members. Having played a handful of shows at larger venues, he has dealt with open spaces that dissipate energy and guardrails that divide performer and audience; these technologies could be responses to this divide. When asked about incorporating similar technologies in his own shows, Christian was at first hesitant. He expressed concern about giving up the ``rawness" of the performance; technology could take audiences out of the show. He was also wary of giving up control of the show. However, he quickly backed up, noting that ``some of the best gigs are when I felt pretty out of control." Giving the audience some control over the lights or even the set list could be appealing, he admitted, although he would not want the crowd dictating the whole structure of the performance. Christian explained that his sets are organized around tension and release; any audience interaction would have to keep the overall flow of the performance in tact. An ideal system would follow a plan while allowing for the spontaneity that will make the show memorable.  Overall, Christian had no reservations about tech-enabled shows. ``It's a natural evolution," he explained; technology has always helped to move rock music forward. ``That's why I love rock and roll... There's no rules." 

\subsection{Analysis}

The subjects all have considerable backgrounds as performers. They have been involved in various types of performances in different venues and for many types of audiences. They all recognized the importance of connecting with fans online and have benefited from doing so. Thus, these are experienced performers who operate within the online participatory culture, making them prime candidates for this study.

A number of themes were present in many of the performers' remarks, and some notable points were brought up. Firstly, all subjects recognized the positive impact that audience involvement can have on the quality of a performance. Erik and Christian specifically mentioned the importance of raising a crowd's energy. A participatory technology might focus on high-energy activity. Each artist also expressed the significance of context. A participatory technology should either be designed for a specific type of venue and audience or be made adaptive and scalable. It was mentioned that many small venues allow performers to enter the audience and even touch audience members. A participatory technology could complement this physical commingling in such venues or emulate it in those where a barrier exists. Blake noted that audience-performer interactions can be especially meaningful for the performer. An effective system should enhance the performer's experience as well as the crowd's. There were different opinions on the overall purpose of participatory technologies; while Christian focused on their ability to unit a group, Erik was more interested in them serving as unique, lasting souvenirs.  Although a souvenir element seems less relevant to the research question, it has been implemented before (with PixMob and Xylobands, for example) and would likely add value for audience members. 

Several concerns were also raised by the musicians. Blake and Christian expressed worry about new technology distracting the audience; the spectacle could overshadow the music. A participatory technology should be balanced, complementing the performance without drawing the crowd's attention away from the primary output. The technology must not be a distraction to the performer as well. A balance should also be struck in how much control the audience is given. An operator could be implemented to monitor and even edit audience inputs. Subjects suggested that audience members should not be forced to participate if they do not wish to. A participatory technology should be effective when only part of the audience is participating; ideally, it would also adapt if users stopped providing input. Erik was unexcited by the use of smartphones in a participatory technology, citing them as a possible source of distraction; however, he also noted that users could be prone to dropping and damaging the devices. A participatory technology should avoid the incorporation of personal devices. Lastly, the subjects spoke about the influence of alcohol at most rock music performances. Anyone encouraging audience involvement should be aware of the possibility of intoxicated participants and have a plan for dealing with this.

A predictable yet important result of this study was the emphasis that all musicians are different. The subjects all had substantial experience performing, for instance, but their interactions with audiences differed considerably. While all three musicians were openminded about the projects that they were shown, they certainly have contemporaries that would immediately dismiss such technologies. A participatory technology should reflect the performance style of the artist; ideally, a system would be designed in close cooperation with the performer who would be implementing it.

% Interview Outcomes:
% * All subjects recognize the importance of connecting with fans online and have benefited from doing so
% * Performers may enter the crowd or physically touch audience members
% * Performers aim to raise the energy of the audience and feel that this tends to increase the overall quality of the show
% * Audience participation can be perhaps be more meaningful for the performer
% * Alcohol is an important factor in determining a crowd's behaviour
% * Some performers are distracted by audience involvement
% * The projects impressed all of the artists, though certain aspects raised concerns for each. Spectacle could become more important than musical performance.
% * There was hesitance about giving the audience too much control
% * The technologies may be making up for the distance separating performer and audience
% * Smartphones could be distracting, and people may drop them; plus, they are uninspiring
% * All noted that context -- venue size, audience makeup -- is important and can change how decisions are made
% * A participatory technology should reflect that every show is unique
% * People should be able opt out of the interaction without affecting others
% * A predictable yet notable result of this study was that all musicians are different. The performers interviewed all had substantial experience performing, for instance, but their interactions with audiences differed considerably.
