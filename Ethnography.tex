\chapter{Ethnography}

One of my first goals was to get a sense of modern concertgoers' and performers' feelings about participatory performances and interactive technology. I sent out a brief online survey for music fans that helped me to understand how they generally responded to these topics. Interviews were also conducted with multiple musicians to shed light on their perspectives.
% Explain the details of how participants were recruited, how data was collected, recorded, and processed/analyzed, and the timeline over which this process unfolded

\section{Audiences}

% Explain the content of the survey:
An online survey was created in order to obtain a sample of modern music fans' opinions on interactive performances. The survey was completed by ninety-nine participants recruited via social media. The first few questions informed me of what type of concertgoer each participant was -- asking their favourite genre, the size of the venues they frequent, and how often they attend live music performances. I also asked how often the participants communicate directly with musicians through their social media presences. Next, the survey focused on concert behaviours. Participants were asked in which actions they typically partake at live music performances; choices included applauding, headbanging, and holding up lighters. They were asked how they might like to interact with their favourite performer and what sort of message they would send them if they could. I asked for their thoughts on getting involved in performances, bringing new technologies into concert settings, and interacting with musicians using social media services. (For complete results, see Appendix X.)

% Present the direct results:
The results were not shocking but certainly informative. Most participants favoured rock music or some variation (``indie," ``alternative"); the majority attended multiple concerts per year -- some even on a weekly basis; and most usually went to shows at small- to medium-sized venues. The majority of participants claimed to communicate with artists through social media either sometimes or regularly, though a sizeable amount indicated they never do this. The most popular concert actions were applauding, singing along with the performer, clapping or stomping to the beat, dancing, jumping up and down, and chanting words or phrases along with the other audience members. When asked how they might want to interact with a performance, many said they would like to choose the songs that are played, while much fewer expressed interest in manipulating visuals and contributing to the music; around one quarter of participants stated they did not have interest in directly influencing a live music performance at all. Given the opportunity to communicate with their favourite performer, most participants responded with praise or appreciation (``Thank you," ``I love you!"). Other messages included song requests and suggestions like, ``Don't bury the vocals," or ``More rock, less talk!" The majority of participants indicated that they enjoy when performers ask them to participate in a performance -- clapping or singing along or call and response, for example. Lastly, the majority also said they were excited by the idea of bringing new technologies into a live music setting.

% Present the results of the in-depth analysis:
Upon further analysis of the responses, some correlations were uncovered. There are clear relationships between show-going frequency, venue size, and interest in interaction and technology. Participants attending shows more frequently are more likely to visit smaller venues. This group also expressed the most interest in being involved in performances; they are more inclined to interact with their favourite artists via social media; and they are more welcoming to the idea of unfamiliar technology in a concert setting. The opposite, thus, can also be said: participants who go to fewer shows tend to go to larger venues, are more likely to refrain from participating in shows, are less likely to contact artists through social media, and are less interested in new technologies.


\section{Performers}

% Explain the recruitment process and describe the participants:
With a broad overview of audience attitudes, the next step was to speak directly with actual performers and discuss the same topics with them. Three musicians were recruited -- Erik Grice, Blake Enemark, and Christian Hansen. These three were selected because I was familiar with their bands and felt that they represented three distinct performance styles within the rock genre. The subjects have different backgrounds as performers, experience playing in different parts of Canada, and for a range of audience sizes. They also have slightly different relationships with modern technology. While their opinions are representative of modern rock musicians, it should be kept in mind that these subjects are also music fans who enjoy watching others perform as well; thus, they are also sharing their perspective as concertgoers.

% Outline the interview questions
After briefly establishing their history as performers, I asked them each about what audience participation means to them. The musicians were shown video of some of my case study subjects (including Xylobands, Wham City Lights, and Kasabian's 2011 tour), and I recorded their reactions and general thoughts on technology-enabled performances. Lastly, the artists were asked if and how they would want to incorporate similar participatory technologies into their own shows.

\subsection{The Performers}

% About Erik:
Erik Grice grew up outside of Edmonton, Alberta. He performed musical theatre as an adolescent and started playing in bands in his teens. After acting as a vocalist and guitarist in previous bands, Erik now plays drums for Edmonton-based The Fight -- an alternative rock band with soul undertones. The Fight's typical audience size can range from 25 at road shows to 150 people at home shows. Venues are usually small clubs or halls. They have also played acoustic shows at cafes or similar venues. A typical Fight concert is energetic yet composed. Audience members can comfortably move to the music and may be welcomed to sing along with certain choruses.

% About Blake:
Blake Enemark is originally from Victoria, British Columbia. He picked up guitar as a teenager and started playing covers in bands. Blake recorded and toured with alt-country band Forestry in 2010. Shortly after, he was recruited to join We Are The City, a Vancouver based progressive rock band who had just gained nation-wide recognition after winning the \$150 000 Peak Performance award. With this group, Blake experienced audience more sizeable and fanatic than he had experienced before, culminating in a performance for around two thousand people in Vancouver's Stanley Park. Blake parted ways with We Are The City shortly thereafter, going on to tour with folk group Northcote, before starting his own project called Snoqualmie. Snoqualmie, described as ``high-gain, sad-sack Canadiana," marks a return to songwriting and more intimate shows for Blake.

% About Christian:
Christian Hansen has a theatre background, a graduate of the University of Alberta's acting program. He began playing in bands as a teenager. During his university degree, Christian rediscovered his desire to perform music and began playing acoustic shows. He eventually started performing with his now-wife Molly and began incorporating drum machines and prerecorded tracks into his work, which was becoming more like dance music. Christian Hansen \& The Autistics were formed. When Molly could not make it to one show, Christian was inspired to put all of his energy into his performance to make up for it; this was a ``lightbulb moment for him." Christian Hansen \& The Autistics soon gained notoriety around Edmonton for their high-energy shows, and their songs received a lot of radio play. Christian and Molly moved to Toronto in 2011, shortening their band name to Christian Hansen and now playing music described as new wave. While they are currently working their way into Toronto's music scene, Christian Hansen typically draw around five hundred excited showgoers when they play in Alberta.

\subsection{Audience Participation}

% Erik:
The Fight encourage some forms of participation at their shows. Their lead singer can often walk out and physically touch audience members, looking them in the eye as he sings. They will invite the crowd to sing and/or clap along for suitable songs. Erik felt that this sort of participation typically makes for better shows: ``An attentive and participating crowd of fifty people is always going to be better than two hundred people who are standing there with blank faces.? He explained that the ideal audience will match the energy that the band exudes. In addition to making the show more enjoyable, he said, this also makes musical flubs less significant to everyone present. Erik felt that it is the band's job to keep the audience's attention. They must make decisions based on the venue, the audience, the length of the set. Stage banter is usually only implemented to convey pertinent information to the crowd. Talking to audience members after the show has gotten the band extra shows, radio spots, places to stay for the night. 

% Blake:
Blake is a self-described introvert. While he is certainly the leader of his current band, he expressed a preference for playing in the background. Despite being typically shy with audiences, Blake recognized the significance of even basic audience participation. Singing and clapping along makes you feel like a part of the show. He also expressed the impact this might have on performers; ``It would be the most flattering thing in the world for me for someone to sing my song back to me," he said. Blake noted, however, that different performers react to audience participation differently. A guitarist in his former band, for instance, would not allow the audience to clap along. Furthermore, ``There's a fine line," Blake stated, ``between drunken participation and intentional participation." He acknowledged the prominence of alcohol in live music settings and the importance of alcohol sales at most venues. It is a part of the industry, he said, and it plays a role in how audiences behave.

% Christian:


\subsection{Social Media}

% Erik:
The Fight make use of popular social media channels. They use Facebook to advertise, organize contests and giveaways, and share information on their live dates. Twitter is used for communicating with fans and other artists. This has helped them open slots for touring bands. At shows, the band hands out cards with links to their social media pages. Erik believed there is a levelling out between artists and fans, and he likes this. He explained that some contemporaries try to maintain personas and seem ``untouchable": ``It's stupid to have those kinds of pretences," he said. He noted that, for larger bands, direct interaction with fans is more difficult, but overall the extra freedom is nice.

% Blake:
The internet presence of Blake's current band is ``not very good." Blake himself recently closed his Facebook account for personal reasons. Despite this, he recognizes the importance it holds for artists: ``It's bridging gaps that have never been bridged before." 

% Christian:


\subsection{Participatory Technologies}

I showed the musicians these projects...

% Erik:
Erik responded negatively to the projects that relied on smartphones; he feels they would be mostly distracting, and he expressed concern about possible being responsible for people dropping and damaging their devices. ``What can't your phone do now?" he asked. Rather, he favoured the work that had other tangible elements, like the PixMob beach balls. Erik felt that every show should be unique. An attendee should be able to go home and say, ``I was at that one" -- a digital analogue to the classic concert tee with dates listed on the back. Some artists post photos and set lists from each of their shows, he explained, and even this makes a show feel personal for those who attended. If the set changes slightly each night, this amplifies the effect, as well as keeping things interesting for the performer. Erik felt that these sorts of technologies should allow people to opt out without affecting the others' experience. However, he also expressed concern about giving power to all audience members. Something like a Twitter feed is surely edited. Bands that hand out percussion instruments like tambourines have to deal with participants with no rhythm. Erik provided an anecdote about a band that gave miniature harmonicas to audience members to be played during one part of one song; the crowd continued to play the instruments throughout the whole set and the other bands' sets.

% Blake:


% Christian:



\section{Analysis}

A few general conclusions can be made from the questionnaire results that are particularly relevant to my research question. It is encouraging to confirm that most participants are not quietly standing still at live performances; they are cheering, moving, and singing along. A surprising find was that, given the chance to say anything to their favourite artist on stage, most participants would choose simple messages of praise or thanks -- something ostensibly achieved already by applauding. Also intriguing was the relative lack of interest in influencing lights and visualizations. Instead, the majority of participants showed great interest in choosing the set list for the performance. Regardless, it is clear that most respondents have little to no reservations about being directly involved in a show and doing so with new technologies. Seemingly, this willingness to interact is more common in those who frequently attend performances at smaller venues. Perhaps, then, artists that play to smaller crowds and can offer more direct interactions both on and off stage have fan bases that are more willing to experiment with new interactions.
