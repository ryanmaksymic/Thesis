\chapter{Related Work}

\section{Designing for Audiences}

Designing for large groups of people has only recently attracted notable interest in the field of human-computer interaction (HCI). As interactive systems become increasingly ubiquitous, HCI researchers are asking how the needs of multiple people in a public space differ from those of an independent user. The characteristics of live performance make it an especially useful venue for these investigations; thus, conveniently, much of the research done in this field focuses on concerts, theatre performances, and dance clubs.
% How are interactive systems best designed for crowds?
% Designing interactive systems for crowds -- large public displays, crowd-based gaming, sports spectators. This has only been on CHI's radar since 2009 (Brown's workshop).
% Designing interactive systems for crowds in live music environments
% Designing systems for participatory music experiences
% Davidson, 1997:
% * Performance etiquette is usually formed by crowd mentality, following the majority
% * Performers pick up information from the audience's broad and specific behaviours
% * Visuals help audiences read the performer's intentions
% Sexton, 2007: Simple synchronous interactions in sound art projects left users with little to explore, resulting in a "flat" experience
% Jourdain, 1997: We move to music in order to "represent" it. This also amplifies, resonates the musical experience.
% Levitin, 2006:
% * "In every society of which we're aware, music and dance are inseparable." Ancient music was based on rhythm and movement. Combining rhythm and melody bridges our cerebellum and cerebral cortex.
% * Ties between music and movement have only been minimized in the last 100 years
% Kelly, 2007: Technology incorporated into a show can either be addressed as part of the show or hidden and made illusory

\subsection{Maynes-Aminzade, Pausch, and Seitz}

In their 2002 paper, Maynes-Aminzade et al. describe three different computer vision systems that allow an audience to control an on-screen game; they also outline the lessons they learned about designing such systems. The first method tracks the audience as they lean to the left and right. The control mechanism was intuitive, but the system required frequent calibration. The second method tracked the shadow of a beach ball which acted as a cursor on the screen. This was also intuitive, but it only involved a few people in the audience at a time. The third method tracked multiple laser pointer dots on the screen, giving each audience member a cursor; this was a more chaotic system once the number of dots got overwhelming. Lastly, the authors presented some guidelines for designing systems for interactive audience participation. They recommend focusing on creating a compelling activity rather than an impressive technology; they state that every audience member does not necessarily need to be sensed as long as they feel like they are contributing; and they suggest that the control mechanism should be obvious or audience members will quickly lose interest. The authors also note that making the activity emotionally engaging and emphasizing cooperation between players will increase the audience's enjoyment.

\subsection{Aigner et al.}

% Blah...

\subsection{Reeves, Sherwood, and Brown}

This paper investigates the design of technology for crowds by observing and analyzing the behaviour of a group of football fans gathered at a pub. The authors note that most related work has focused on spectators at a performance or on exceptional circumstances like riots. This work instead looks at everyday crowd-based settings where there is no attention-grabbing ``spectacle." To accomplish this, the researchers video recorded a crowd gathered at a pub during a football match and examined the group's behaviour for recurring themes. People were seen singing, jumping, and pumping their fists in the air in sync with each other. In general, these instances of collective participation were all visible or hearable from far away. Once a small group of people began the actions, they would quickly ``snowball" and overtake the crowd. Researchers also noticed the importance of ``shared objects;" an inflatable object bouncing between people, for example, connected individuals at a distance. It was also observed, of course, that not every person in the crowd cared to participate in these group activities.

After outlining these observations, the authors present a list of design lessons that they extrapolated. First, they suggest treating a crowd as a unit rather than a collection of individuals -- for example, exploiting already-present crowd behaviours or allowing for only partial participation. The importance of ``intra-crowd interaction" is also emphasized: allow for shared objects and space-dependent interactions, and take advantage of snowballing by encouraging highly visible/audible actions. Additionally, one should allow for interaction with people on the fringes of the crowd but be aware of problems that could be caused by latency. Lastly, the researchers note that every crowd is different and that each design should reflect the nature of the environment.


\section{Creative Collaboration}

\subsection{Ulyate and Bianciardi}

In their paper, the authors describe their ``Interactive Dance Club" -- a venue that delivers audio and video feedback to inputs from multiple participants -- and they present the ``10 Commandments of Interactivity" that guided its creation. The goals of the project were to create coherent musical and visual feedback for individual and group interactions and to allow non-artistic people to feel artistic. Inputs included light sensors, infrared cameras, pressure-sensitive tiles, proximity sensors, and simple mechanical switches. By interacting with them, users could make notes sound out, manipulate projected video and computer graphics, modulate music loops, and control the position of cameras in the space.

This project's ``10 Commandments of Interactivity" contain the following points:
\begin{itemize*}
	\item Movement is encouraged and rewarded.
	\item Feedback from interactions is immediate, obvious, and meaningful in the context of the space.
	\item No instructions, expertise, or thinking is required.
	\item A more responsive system is better than a more aesthetically pleasing system.
	\item Modularity is key.
\end{itemize*}

Lastly, the authors share the lessons that they learned while running the Interactive Dance Club. They observed that interactions involving full-body movements were most satisfying. The form of an object, they found, determined how users first attempted to interact with it. They emphasize the practicality of a system that is both distributed and scalable. Designing the interactions required finding a balance between freedom and constraint. They found that, no matter how elegant the system, some users would still find a way to create unpleasant noise. Lastly, they observed that instant gratification is important; feedback that is too delayed or interactions that require too much concentration are ineffective.

\subsection{Feldmeier and Paradiso}

The authors present a scalable system for wirelessly tracking the movement of a large number of users, allowing a crowd of dancers to influence techno music and lighting in a club...
% The authors present a scalable system for wirelessly tracking the movement of a large number of users, allowing a crowd of dancers to influence techno music and lighting in a club
% The sensors are cheap, lightweight, and portable. They transmit RF pulses when they sense acceleration values over a certain threshold. Batteries can last for years. Receiver stations have low sensitivity, creating ``zones of interaction" around them.
% Machine vision solutions are low cost but have lighting and line-of-sight requirements; they also fail to represent individual inputs (52)
% The design goals were to create a system that was intuitive and predictable with good feedback, resulting in an engaging and enjoyable music experience (54)
% Human behaviour: People will naturally school together and build off of each other's actions; e.g. chaotic applause gradually growing synchronous. This can only occur with sufficient feedback (visual or aural); it is more difficult for larger groups to reach synchronicity without extra feedback (55).
% Music response was carefully mapped to user activity. As more users began dancing in sync, the system progressed through increasingly complex ``energy levels." At the higher levels, users were rewarded with more direct control over the sounds. A rolling FFT determined the dominant tempo of the dancers (56-57).
% Conclusions:
% * Device pros: Low latency, no instructions needed, low cost, low power
% * Successfully gave dancers control over music
% * Took advantage of a crowd's tendency to gradually move in sync
% * Outstanding question: How can we provide close interaction for all users while keeping the output orderly and pleasant?

\subsection{Oh and Wang}

% Blah...


\section{Audience-Performer Interaction}

\subsection{Bongers}

In his 1999 paper, Bongers provides a theoretical HCI framework for physical interaction between performers, audience members, and electronic systems in a musical performance. He defines three types of interaction -- ``performer-system", ``system-audience", and ``performer-system-audience." Bongers models the interactions as control systems wherein actions are either a \emph{control} or \emph{feedback} process. Electronic sensors and actuators are discussed, followed by human senses and motor systems. Bongers states that a more convincing interaction is one that provides ``multimodal" feedback -- influencing more than one of the users' senses. Lastly, a few prototypes of novel interaction systems are described. Especially notable is the ``Interaction Chair", which most easily fits in the performer-system-audience category. Here, the performer has the ability to send vibrations through each audience member's seat back, while the chairs contain sensors that allow audience members to influence visuals projected behind the performer. Other projects like this one can benefit from Bongers' theoretical framework; thinking in terms of control and feedback processes may provide new perspectives on a  system's design.

\subsection{Gates, Subramanian, and Gutwin}

This paper examines the complex interactions between DJs and audience members in nightclubs from an HCI perspective. The authors gathered their information by observing behaviours at nightclubs, surveying DJs, as well as conducting lengthy interviews with them. Most DJs had similar preferences and performance styles. For example, all of the interviewees said they preferred venues where audience and DJ are mutually visible; this allows them to adjust their performance based on visual cues from the audience. Using quick glances, DJs can observe audience members' facial expressions and body language and the flux of people on to and off of the dance floor. Many DJs stated that they will often exaggerate their movement or speak into a microphone to energize the crowd. Small, direct interactions can also occur between DJs and audience members, such as exchanges of facial expressions or gestures. DJs use the information they glean from their audience to shape their performance. Most DJs will craft a playlist before performing based on the venue, event type, and expected audience; during the performance, however, the energy of the crowd ultimately guides how the tracks are mixed. In general, the authors found that, as long as there is sufficient visibility, DJs are extremely competent at adjusting their performance based on the audience. Interviewees saw little need for technology to aid their performances; one of the few wishes the DJs expressed was for a method to discover the musical preferences of a given audience.

Based on the information collected, the authors present some design recommendations for those wishing to bring interactive technologies to nightclubs. For example, they state that, considering how skillful DJs are at observing audiences, any technology meant to gather information from the crowd must be more efficient than DJs themselves. Such technology, the authors suggest, would be most useful for gathering ``invisible" information like musical preferences. They recommend against using biofeedback systems or systems where audiences have a direct influence on the performance; these methods do not help DJs do their job. The researchers state that gradual changes are more satisfying than immediate ones. Lastly, they emphasize the importance of respecting the DJ's art; technologies should allow them to stay in control of the music and should not add to their already-demanding cognitive load.

\subsection{Barkhuus and Jorgensen}

Barkhuus and Jorgensen's paper investigated interactions between audiences and performers at a concert. The authors used observations from traditional rock and rap shows to inform the design of a simple "interaction-facilitation technology" -- a cheering meter. By tracking the applause patterns at several concerts, it was determined that the two most common reasons for cheering were to express anticipation and to reward the performers. This led to the creation of a cheering meter, an instrument for measuring the volume of an audience -- in this case, to determine the winner of a rap battle. Microphones captured samples of the crowd's cheering, the signal was filtered, the peak volume was measured, and the rating on an arbitrary scale was displayed on large screens onstage.

The researchers reported no major issues while testing the system, and they express confidence that their technology helped to enhance the concert for the audience members. In their paper, they outline the main reasons for the cheering meter's success. First, the authors state that the usability of the system is due to the fact that it is based on an already-present behaviour; they recommend ``designing technology that fits the situation and which utilize present activities rather than aiming to employ the latest cutting edge technology" (p. 2929). Next, they suggest that an event should not rely on the success of the technology; the rap battle, for example, could have easily continued if the cheering meter malfunctioned. Lastly, the authors emphasized the importance of immediate visual and/or aural feedback; seeing direct consequences of their actions gives the audience confidence in using the system. This research focused on a very specific type of event using an almost-gimmicky system, but the design principles it yielded are valuable.

\subsection{Tseng et al.}

This paper described the motivation and creative process behind a Taiwanese interactive theatre experience that let audience members connect with a dance performance. The project was realized using projection mapping, a Kinect, a local area network, and a custom iPhone app. Audience members downloaded the app before the show and entered a code corresponding to their seat number to connect to the local network. During the first part of the performance, each user was given control over one ``light dot" projected onto the stage. The dot could be moved by moving the iPhone; users could also point their phone's camera at different light sources to influence the brightness of their dot. Later in the performance, audience members could use their phones to trigger sounds and projected images onstage. The dancer, tracked by the Kinect, interacted with the projections, improvising a dance with the light.

The authors approached this project by asking, ``How can the audience become an essential element in a performance?" (p. 561). They claim that, while new media has been incorporated into theatre for decades, mobile phones have not been used to their full potential. Feedback collected after the performance revealed overall positive reactions. Some users, however, were uncomfortable having their personal devices connected to an unfamiliar network. Another negative was that not every audience member owned an iPhone; one of these spectators, though, maintained that she enjoyed the show even while being excluded from the interaction.

\subsection{D'CuCKOO's MidiBall}

D'CuCKOO, a band active in the 1990s, frequently incorporated technology into their live shows. Their MidiBall was a large, wireless beach ball that triggered sounds and visuals on stage when struck by audience members...

\subsection{Plastikman's SYNK}

A smartphone application to accompany Plastikman's 2010/2011 world tour, SYNK allowed audience members to view the performance from the musician's perspective and even influence samples he played...

% The Flaming Lips' vibrating underwear


\section{Interactive Light Shows}

\subsection{Wham City Lights}

Wham City Lights is a smartphone application that allows multiple devices to display light shows in sync during a concert. Audience members with an iOS or Android device can download the app before the show. Once the show has begun, an operator activates lighting cues by playing encoded, ultrasonic tones; devices with the app open ``hear" these tones and perform the corresponding cues. This can be done at nearly any scale as long as every device is able to hear the tones. Users generally hold their devices up or wave them above their heads during the show. Light shows can be created live or programmed in advance using an online editor; cues include flashing colours, camera flashes, GIFs, text, and sound.

The concept was originally developed by US musician Dan Deacon. His intention was to prevent concertgoers from using their personal devices and disengaging during live performances. Deacon tested the app at his own shows and received a positive response. Today, Wham City Lights licenses their general-purpose app for different kinds of events; they also develop custom apps to include branding, tour dates, etc. Musicians and organizations like Brad Paisley, the Billboard Music Awards, and Intel have made use of this technology at their events.

\subsection{Xylobands}

Xylobands are controllable LED wristbands designed to be worn by potentially thousands of users at entertainment events. They are controlled using a proprietary piece of software downloaded to a laptop; the laptop must then be connected to a radio transmitter. With the software, an operator can turn the Xylobands on or off, select which colours are illuminated, and control the speed of the LEDs' flashing. The transmitter has a range of around 300 meters. Each wristband contains a small printed circuit board that holds, among other components, an RF receiver and an 8-bit microcontroller. The electronics are powered by three 3 V coin cell batteries.

The technology was originally developed for the band Coldplay, and wristbands were handed out to all concertgoers during their 2012 world tour. Giving the wristbands to each audience member at every performance reportedly cost the band \euro{}490 000 (around \$680 000 CAD) per night. UK-based toy development company RB Concepts Ltd. are the creators of the Xyloband. Their website advertises that Xylobands can be customized and used at concerts, festivals, sports stadiums, or corporate events.

\subsection{PixMob}

PixMob is a patented wireless technology that enables the control of multiple LED-embedded objects. By giving PixMob objects to spectators, concert producers can create a controllable LED light show within the audience. The objects are activated with signals from infrared transmitters. Like normal lighting fixtures, the transmitters' beams can be shaped with lenses and controlled via the DMX512 protocol. The objects light up when they are hit by a beam, so patterns of moving light can, in essence, be painted across the audience. Light shows are programmed, simulated, and controlled through a software package called LAVA; they can also be controlled in real time using a MIDI controller or the LAVA iPad app. Previous PixMob objects include balls, wristbands, pendants, and beads, and custom object creation is available as well. PixMob also offers ``second life" customization: objects can be programmed to react to sounds, play an mp3 track, or communicate with the user's personal computer after the show is over. Past clients include Microsoft, Arcade Fire, Eurovision, and Heineken.

\subsection{Kasabian's 2011 Tour}

UK-based studio Nanika helped Kasabian bring audience members into their live shows by turning cameras on the crowd and displaying their faces on onstage screens...
