\chapter{Introduction}

% Describe Arcade Fire's 2011 Coachella finale:
At the 2011 Coachella Valley Music and Arts Festival, Monteal-based indie rock group Arcade Fire are about to play one of the final songs of their headlining set. The guitar riff from the band's hit song ``Wake Up" is instantly recognized by the audience, who cheer loudly with excitement. The song reaches the first chorus, and, suddenly, one thousand white beach balls begin tumbling over the top of the stage and gently fall onto the crowd below. The cheers swell into a roar as the balls disperse over the mass of people. When the band hits the song's final chorus, to the audience's surprise, the balls begin to light up -- flashing different colours to the beat of the music. Arcade Fire finish their set, grins on the band members' faces, as they watch the glowing orbs bounce across the crowd. After the show, festivalgoers hold on to the beach balls; vehicles leaving the festival grounds are seen glowing with the light from what have now become souvenirs from an unforgettable live music experience.

% Outline the goal of the thesis:
This project was made possible by several teams that managed the logistics, developed the wireless LED devices, fabricated and tested hundreds of beach balls, and ultimately executed the launch\footnote{\url{http://www.momentfactory.com/en/project/stage/Arcade_Fire}}. The result was an awe-inspiring, albeit momentary, event that allowed the audience to participate in the performance. Rock concerts are growing more technically complex and spectacular all the time. Powerful equipment makes shows louder, larger, and flashier. Only recently, however, have artists and researchers begun investigating how technology can amplify not just the performer's actions, but those of the audience members as well. This thesis examines these sorts of technologies, asking how they might be used to make conventional rock concerts more participatory.

% The following ensures page 1 is not numbered; be aware of changes in ToC
%\pagenumbering{arabic}
%\setcounter{page}{2}

\section{Motivation}

% Introduce Turino's idea of presentational vs participatory performances:
In his 2008 book \textit{Music As Social Life}, musicologist and anthropologist Thomas Turino divides live music performances into two categories -- presentational and participatory. In presentational performances, the artist prepares music and presents it to a separate group, the audience. An example of a presentational performance would be a typical rock concert; a band rehearses and plans a set list and then performs it for a generally attentive audience. Participatory performances, on the other hand, deal only with participants and potential participants, and there is no artist-audience distinction. Peruvian communities, for example, perform in large groups with each participant either dancing or playing a panpipe or flute.

% Evolution of music and performance:
Researchers have found evidence of musical activity from every known era of humanity (Levitin, 2006). While music has served various function for different cultures -- used to remove curses (Turner, 2011) or even to settle lawsuits (Jourdain, 1997) -- it is most widely regarded as a relationship-forming social activity (Levitin, 2006; Turner, 2011). It is only in the last few centuries that presentational performances have become ordinary public events. These grew in popularity when entrepreneurial musicians began taking advantage of the emerging middle class, members of which were eager to display their newfound wealth (Small, 1998). Eventually, art music (now typically referred to as classical music) performances established a strict divide between performer and audience; musicians dressed alike and performed composed and rehearsed pieces while audience members sat silently and attentively.
% You can't say `art music' and `classical music' are the same without a reference. Reword this.

% Today's concerts are visual:
While today's popular music concerts are more relaxed, the divide between performer and audience is still prominent. Furthermore, performances today are more presentational than ever, with all sorts of additional stimulus being blasted at audiences. Giant screens display complex visuals and extreme closeups of performers, and heavy-duty mechatronics are incorporated to dynamically change the shape of the stage. Auslander (1999) suggests that this focus on the visual is a response to the visual culture established by mass media -- namely, MTV. Many concertgoers could be satisfied by an entirely lip-synced performance if it is enough of a spectacle for the eyes.

% Rock shows are participatory:
Despite this focus on the presentational, audience involvement has managed to survive as a minor participatory element in rock concerts. Audience members become a part of the performance by singing or clapping along, holding their illuminated lighters or cellphones above their heads, or simply moving to the music. Some musicians take song requests, and some even invite audience members on stage alongside them. In rock music, the barrier between the audience and the performer is frequently crossed.

% Participatory culture:
Just as concerts became more visual in response to an increasingly visual culture, I believe that they are now becoming more participatory in response to an emerging ``participatory culture" (Wikstr\"{o}m, 2013). The unprecedented connectivity afforded by the Internet and digital technologies is bringing artists and their followers closer than ever. Musicians can talk to fans directly on social media websites and receive instant feedback on their work. Fans can remix their favourite artists' songs or create their own music videos to accompany them. Social networks form around concerts as attendees connect with each other beforehand and artists curate fan-shot images and video online afterwards. Rather than flowing from performer to audience, information now freely flows between the two parties.
% This connectivity mostly lives off stage. My goal is to take advantage of this shift ON stage.

% Researchers are looking at crowd-based interfaces. Popular artists are beginning to experiment with participatory technologies.

% Segue:
By taking advantage of this participatory culture and the technology that produced it, we might begin seeing more performances in which music can better function as the communal activity it has always been.


\section{Purpose}

This thesis investigates the following question: How might rock performances be made more participatory using digital technology? 

This question leads to many others. For instance, how -- if at all -- do modern audiences want to be involved in performances, and how much control do today's rock musicians feel comfortable giving up? What aspect of the performance can be reasonably controlled by a large number of people? How can inputs from each audience member be captured and processed? And when such a system is implemented, does it actually improve the concert experience for both the audience and performer?
% Questions:
% * How might technology be used to add participatory elements to traditionally presentational rock concerts?
% * How do audiences want to participate? How do performers want audiences to participate?
% * Input: How can data best be captured from the audience?
% * Output: What can the audience interact with that will enhance the performance but not distract the audience or performer?
% * Feedback: What kind of feedback will be informative/useful without being distracting?
% * What's the threshold where a performance can no longer be social? What's the ratio where participation can be sustained?

% Brief rationale

To address the research question, I created multiple participatory technologies using a user-centred design process that implemented ethnographic study and prototyping.

% Ethnographic study:
Ethnography is the exploration of a cultural phenomenon. The phenomenon I am interested in is a rock concert. I aimed to find what this event means to both performers and audience members. I posted a questionnaire for music fans; many people go to concerts frequently, and I was able to acquire over one hundred responses. To investigate the perspective of a performer, I conducted one-hour interviews with three active musicians.
% Ethnography: Examining cultural phenomena within a group. Surveying modern music fans will provide me with a general sense of their feelings towards music, live performance, and technology. I will interview active musicians to understand how and why they interact with fans, on and off stage, and what they think of new technologies in a performance setting. These methods will allow me to gain a deeper understand the users and environments that are related to my work.
% Mention how your personal experience -- going to shows, putting on shows, stage managing, personal relationships with musicians -- shape your perspective

% Prototying:
I developed three prototypes and tested them with groups of users. The last prototype was implemented at a real-world concert; I worked closely with a local band to develop the prototype, and audience reactions were closely observed.
% User-centred design: The users' wants and needs drive the design. Prototypes will be tested with users. Observations and interviews with the participants will help answer research questions as well as validate or disprove any assumptions that have been made. Both audience and performer perspectives will be represented. This method will help establish design guidelines.


\section{Scope and Limitations}

Some delimitations will be enforced to ensure the research remains focused. In this context, `rock music' refers to Western rock and roll music, but this could also include pop, hip hop, rap, R\&B, folk, and more. The ultimate requirement is that the music is performed in an environment where audience and performer roles are clearly defined, but the atmosphere is relaxed and audience members are comfortable participating. Results of this research may be applicable to other performance environments, but this is not discussed in this thesis. While the related work involved large-scale events, this thesis deals with small-sized audiences and venues. This is due to limited access to large bands and venues; however, it also reflects my personal experience, as I am most comfortable working with small bands in small venues. While many principles are mainly unaffected by audience or venue size, it is in some cases an important factor, and I note when this is so. Lastly, market viability of the prototypes is not discussed at this point.

% Turino, 2008: Identifying the frame of a given type of music is important to guide your interpretation
% Auslander, 1999: Rock vs Pop = Authentic vs Inauthentic
% Contrary to Auslander, I will be grouping them together and referring to both as "rock" music. This encompasses any music that is typically performed at live music venues or festivals with audiences that are typically comfortable moving and being vocal throughout the performance.
% Small, 1998:
% * The focus on classical music is odd. It's claimed to be intellectual, but it's not very popular.
% * Classical music performance turned audience members into spectators, consumers
% * Modern orchestra concerts emphasize listening, specialization, formal settings, ticketed admission. They celebrate separation of producer and consumer.
% I am interviewing independent artists who play in small venues, so this is the lens through with I will be viewing the problem
% Brown, 2009: % Crowd psychology rarely ``takes a practical orientation to how crowds might interact with technological artifacts or events?


\section{Overview}

The remainder of the document is organized as follows:

\subsubsection{Chapter 2: Literature Review}
	
An overview of the history of music and performance is provided, and presentational and participatory performances types are compared. I examine how presentational performance has grown into what it is today and how rock music and modern technologies are bringing participation back into live music. A survey of related research includes work in group-controlled systems, creative collaboration, and audience-performer interaction. Real-world participatory technologies are also examined.
	
\subsubsection{Chapter 3: Research Approach}
	
Primary and secondary research questions are listed and justified. The research methods that were implemented are outlined.
			
\subsubsection{Chapter 4: Ethnography}
	
The research questions are examined from both the audience's and the performer's point of view. This is accomplished through a survey of music fans and interviews with musicians.
	
\subsubsection{Chapter 5: Prototyping}

I describe the production of three prototypes. This includes the objectives of the prototypes, their development processes, user testing, and analyses of the results.
		
\subsubsection{Chapter 6: Conclusion}
	
To conclude, I summarize the overall outcomes of my work and discuss possible future directions for the project.