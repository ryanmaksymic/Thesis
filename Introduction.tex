\chapter{Introduction}

% Describe Arcade Fire's 2011 Coachella finale:
At the 2011 Coachella Valley Music and Arts Festival, Monteal-based indie rock group Arcade Fire are about to play one of the final songs of their headlining set. The guitar riff from the band's hit song ``Wake Up" is instantly recognized by the audience, who cheer loudly with excitement. The song reaches the first chorus, and, suddenly, one thousand white beach balls begin tumbling over the top of the stage and gently fall onto the crowd below. The cheers swell into a roar as the balls disperse over the mass of people. When the band hits the song's final chorus, to the audience's surprise, the balls begin to light up -- flashing different colours to the beat of the music. Arcade Fire finish their set, grins on the band members' faces, as they watch the glowing orbs bounce across the crowd. After the show, festivalgoers hold on to the beach balls; vehicles leaving the festival grounds are seen glowing with the light from what have now become souvenirs from an unforgettable live music experience.

% Outline the goal of the thesis:
This project was made possible by several teams that managed the logistics, developed the wireless LED devices, fabricated and tested hundreds of beach balls, and ultimately executed the launch\footnote{\url{http://www.momentfactory.com/en/project/stage/Arcade_Fire}}. The result was an awe-inspiring, albeit momentary, event that allowed the audience to participate in the performance. Rock concerts are growing more technically complex and spectacular all the time. Powerful equipment makes shows louder, larger, and flashier. Only recently, however, have artists and researchers begun investigating how technology can amplify not just the performer's actions, but those of the audience members as well. This thesis examines these sorts of technologies, asking how they might be used to make conventional rock concerts more participatory.

% This ensures page 1 is not numbered:
%\pagenumbering{arabic}
%\setcounter{page}{2}

\section{Motivation}

% Introduce Turino's idea of presentational vs participatory performances:
In his 2008 book \textit{Music As Social Life}, musicologist and anthropologist Thomas Turino divides live music performances into two categories -- presentational and participatory. In presentational performances, the artist prepares music and presents it to a separate group, the audience. An example of a presentational performance would be a typical rock concert; a band rehearses and plans a set list and then performs it for a generally attentive audience. Participatory performances, on the other hand, deal only with participants and potential participants, and there is no artist-audience distinction. Peruvian communities, for example, perform in large groups with each participant either dancing or playing a panpipe or flute.

% Evolution of music and performance:
Researchers have found evidence of musical activity from every known era of humanity (Levitin, 2006). While music has served various function for different cultures -- used to remove curses (Turner, 2011) or even to settle lawsuits (Jourdain, 1997) -- it is most widely regarded as a relationship-forming social activity (Levitin, 2006; Turner, 2011). It is only in the last few centuries that presentational performances have become ordinary public events. These grew in popularity when entrepreneurial musicians began taking advantage of the emerging middle class, members of which were eager to display their newfound wealth (Small, 1998). Eventually, art music (now typically referred to as classical music) performances established a strict divide between performer and audience; musicians dressed alike and performed composed and rehearsed pieces while audience members sat silently and attentively.

% Today's concerts are visual:
While today's popular music concerts are more relaxed, the divide between performer and audience is still prominent. Furthermore, performances today are more presentational than ever, with all sorts of additional stimulus being blasted at audiences. Giant screens display complex visuals and extreme closeups of performers, and heavy-duty mechatronics are incorporated to dynamically change the shape of the stage. Auslander (1999) suggests that this focus on the visual is a response to the visual culture established by mass media -- namely, MTV. Many concertgoers could be satisfied by an entirely lip-synced performance if it is enough of a spectacle for the eyes.

% Rock shows are participatory:
Despite this focus on the presentational, audience involvement has managed to survive as a minor participatory element in rock concerts. Audience members become a part of the performance by singing or clapping along, holding their illuminated lighters or cellphones above their heads, or simply moving to the music. Some musicians take song requests, and some even invite audience members on stage alongside them. In rock music, the barrier between the audience and the performer is frequently crossed.

% Participatory culture:
Just as concerts became more visual in response to an increasingly visual culture, I believe that they are now becoming more participatory in response to an emerging ``participatory culture" (Wikstr\"{o}m, 2013). The unprecedented connectivity afforded by the Internet and digital technologies is bringing artists and their followers closer than ever. Musicians can talk to fans directly on social media websites and receive instant feedback on their work. Fans can remix their favourite artists' songs or create their own music videos to accompany them. Social networks form around concerts as attendees connect with each other beforehand and artists curate fan-shot images and video online afterwards. Rather than flowing from performer to audience, information now freely flows between the two parties.

% Segue:
By taking advantage of this participatory culture and the technology that produced it, we might begin seeing more performances in which music can better function as the communal activity it has always been.

% === OLD === %
% Describe technologies that promote the presentational:
%Many technologies exist for creating enhanced presentational performances. In general, they are implemented to aid the artist in presenting their music to the audience. When The Beatles did their first tour in North America, for example, they had 100-watt amplifiers custom made to ensure their music could be heard over the incredibly loud cheers of the fans; ultimately, the equipment was not nearly loud enough to overpower the audience (citation). Today, a large arena rock show might implement sound systems demanding tens or hundreds of thousands of watts (citation), enough power to send strong vibrations through concertgoers' bodies or even, after enough exposure, cause hearing damage. Performers may take advantage of enormous screens that provide far-away fans with close-up views of the show. Complex lighting rigs, laser arrays, and flashy visualizations are also common methods of turning a regular performance into an awe-inspiring spectacle. Recently, improved webcasting technologies have allowed for the live streaming of concerts over the Internet; a U2 concert streaming on YouTube in 2009 generated ten million pageviews, vastly increasing the reach of their performance\footnote{\url{http://www.wired.com/business/2009/11/4-ways-live-and-digital-music-are-teaming-up-to-rock-your-world}}.

% Explain why we should focus more on the participatory:
%There is clearly a great deal invested in enhancing the presentational aspect of live music performances in Western culture. As Turino points out, when it comes to presentational performances, profit making is usually the primary goal. Louder speakers and bigger screens mean artists can play larger venues with more seats to sell to fans. On the other hand, Turino admits that participatory performance does not fit well within capitalist societies: ``Participatory traditions tend to be relegated to special cultural cohorts that stand in opposition to the broader cultural formation" (p. 36). Why, then, might we want consider technologies that enhance the participatory aspect of performances? As stated by Turino, disregarding its potential financial value, it is music's function as a social interaction that holds the most value for humans. Levitin (2008) agrees, stating that the social nature of music may have been an important evolutionary adaptation that helped early humans thrive in groups: ``Singing around the ancient campfire might have been a way to stay awake, to ward off predators" (p. 258). Thus, while we may no longer depend on participatory performance to the degree that our ancestors did, it seems as though it is against human nature to continue widening the gap between audience and performer. Artists are placed on brightly lit stages with booming sound systems, allowing their voices to echo through stadiums; the voices of audience members, meanwhile, become meaningless noise as more and more people are packed into venues. How might technologies instead be used to embrace the social functions of music and let everyone -- performer and audience -- be a participant? This attitude is, fortunately, reflected in some of the ways we are using technology today.
% Audiences use techniques (clapping, singing along) and tools (lighters, signs) to participate in concerts already

% Describe how the Internet is facilitating participatory interactions:
%The Internet has connected performer and audience in a new way. Social media allows for unique interactions between artists (big and small) and their fans. A small touring band, for example, might send out a message to their followers on Twitter asking for restaurant suggestions in a town they are passing through. More well-known artists may have difficulty connecting to their growing fan base, but events like ``Ask Me Anything" question-and-answer sessions on Reddit allow them to directly answer questions from their supporters. Beyond these social media interactions, the Internet is also continually supplying new ways for musicians and fans to connect. Crowdfunding platforms like Kickstarter rewards fans for directly funding artists' projects by giving them exclusive gifts; Feedbands\footnote{\url{http://www.feedbands.com}} lets users vote for their favourite musicians each month, and the winning artist gets their record pressed on vinyl in a limited-edition run; using Alive\footnote{\url{http://www.alive.mu}}, Japanese promoters can ensure shows get adequate turnouts by having concertgoers commit to buying tickets before artists are even booked to play. The Internet has afforded many new types of participatory experiences for musicians and fans. Recently, digital technologies are being used increasingly to enhance this connection during live performances themselves.
% Take some details away here and save them for the Background? Add in participatory research stuff?

% Describe how participatory technologies are starting to show up at concerts:
%As the music industry continues attempting to find its footing in the digital age, ticket prices for concerts are steadily increasing -- seeing a 40\% increase from 2000 to 2008, for example\footnote{\url{http://www.musicthinktank.com/blog/the-beatles-tell-us-that-weve-hit-the-concert-price-ceiling.html}}. It seems then that, in order to ensure patrons are getting their money's worth, it is important for performers to deliver a truly unforgettable show. Stages, lighting, and visuals are certainly becoming more extravagant. Some artists, however, are looking to more innovative solutions to wow their audiences. Arcade Fire and Coachella, as described above, succeeded in creating a memorable live music experience using hundreds of wirelessly controlled LEDs. Wham City Lights created a similar experience by using a smartphone application to turn audience members' personal devices into a synchronized light show\footnote{\url{http://whamcitylights.com}}. Other instances allow for more direct interactions. At a special performance by R\&B artist Usher, for example, fans could send tweets about the performance that would appear on the large screen onstage and then morph into abstract animations\footnote{\url{http://www.momentfactory.com/en/project/stage/Amex_Unstaged:_Usher}}. Plastikman, alter ego of Canadian DJ Richie Hawtin, released a smartphone app to accompany his 2010/2011 world tour; fans with the app could view a live video stream of the performer's perspective, reorganize audio samples that Hawtin would use in his performances, and participate in a live chat with other users during the show\footnote{\url{http://hexler.net/software/synk}}.

% Summarize why this is a topic worth investigating:
%There are countless types of performance exhibited by the world's various cultures -- some more presentational, some more participatory. In Western cultures, presentational performances draw the biggest crowds, and technologies are being developed to make them larger, louder, and more lucrative. We must remember, however, that music is an inherently social activity; by continuing to separate performer and audience, we are potentially reducing its impact as an event that brings people together. Thus, it is valuable to consider how we might use technology to enhance not the presentational aspect of a concert, but the participatory. The popularity of the aforementioned internet services proves the public's desire for deeper interactions with artists, and recent experiments with interactive systems at live performances have produced impressive results. Lastly, participatory technologies at rock music performances have yet to undergo an academic investigation. I believe, then, that it is worthwhile to survey this fledgling field and better understand its implications on the modern music performance.
% HCI papers have investigated this type of work from a design perspective, but none have deeply involved music fans and performers themselves to ensure the enjoyment (flow) of the event is preserved

% Live music > Recorded music:
%Music consumption in the last century has transformed due to a relatively recent development -- recorded music. Wax cylinders, records, cassette tapes, compact discs, and digital files have allowed listeners to experience music performances from the comfort of their homes, and, today, a concert's purpose is typically to `re-present' these recordings. This does not, however, mean that performance has become subordinate to recorded music. Small (1998) maintains that ``performance does not exist in order to present music works, but rather, musical works exist in order to give performers something to perform" (p. 8). Recordings are just blueprints for performance. He also suggests that, since technology now allows people to experience performances at home, the significance of a real live performance actually increases. There are certainly aspects of a live music performance that cannot come from recorded sound. Musicologist Jane Davidson (2006) explains, for instance, that many musicians seem to perform better when in front of an audience; thus, it is possible that the quality of a live performance often trumps that of a recorded performance. Live performances can also be great sources of cultural influence. Ian Inglis (2006), a sociologist studying music performance, agrees: ``In its ability to simultaneously reflect and influence patterns of socio-cultural activity, it is one of the principal avenues along which musical change and innovation can be introduced and recognized" (p. xv). For example, an event as momentary as the 1965 Newport Folk Festival -- where Bob Dylan first went electric -- seemed to change the course of both folk and pop music (Marshall, 2006). The loud guitar startled audience members, and Dylan was booed as he left the stage. 
% Kelly, 2007: Pop performances are typically re-presentations of recorded material ("reviving the musical corpse")
% Jourdain, 1997: Technology put music everywhere, lessening its "nourishment." "We live in an age of widespread musical obesity."
% Auslander, 1999:
% * The desire for live experiences is a product of mediatization. People want to see concerts so they can experience beloved recorded songs in person.
% * Live performances have great cultural value, but Auslander believes mediatized performances will soon be more valuable
% Davidson, 1997: Performers perform better in live situations due to "social facilitation"
% Inglis, 2006: 
% * Live performance is the principal source of musical change/innovation
% * It is unpredictability that makes live better than mediatized performance
% Marshall, 2006: When Bob Dylan went electric at a festival in 1965, it changed the course of pop music
% Small, 1998:
% * Treating the piece of music as the "the supreme reality of art" (Benjamin) implies some fallacies
% * "Musical works exist in order to give performers something to perform"
% * Music is not a thing. "Musicking" is an action.
% * Since people today can hear and see concerts without having to leave their home, the concert "takes on a new and more concentrated ritual significance"
% Wikstrom, 2013:
% * 

\section{Research Question}

How might rock performances be made more participatory using digital technology? This question leads to many others. For instance, how -- if at all -- do modern audiences want to be involved in performances, and how much control do today's rock musicians feel comfortable giving up? What aspect of the performance can be reasonably controlled by a large number of people? How can inputs from each audience member be captured and processed? And when such a system is implemented, does it actually improve the concert experience for both the audience and performer?
% Questions:
% * How might technology be used to add participatory elements to traditionally presentational rock concerts?
% * How do audiences want to participate? How do performers want audiences to participate?
% * Input: How can data best be captured from the audience?
% * Output: What can the audience interact with that will enhance the performance but not distract the audience or performer?
% * Feedback: What kind of feedback will be informative/useful without being distracting?
% * What's the threshold where a performance can no longer be social? What's the ratio where participation can be sustained?
% Provide my definitions for `participatory,' `enhance,' `feedback,' etc...

% A young research field:
These questions are not only relevant culturally, as discussed earlier, but technologically as well. The ubiquity of public displays, connected personal devices, and movement sensing technology means crowd-based interactions are becoming increasingly practical (Brown et al., 2009). Indeed, human-computer interaction (HCI) researchers are only recently beginning to investigating how systems might be designed for a large assembly of users. In addition to investigating crowd-based interfaces (Maynes-Aminzade et al., 2002; Feldmeier \& Paradiso, 2007), researchers have been asking how these systems should be implemented when a performer is introduced (Gates et al., 2006; Barkuus \& J{\o}rgensen, 2008). This work has only looked at dance performances, rap competitions, and nightclubs, however; there is insufficient research on incorporating these systems into rock performances.

% Artists are recently showing interest:
Despite the lack of research, many inquisitive rock musicians are already showing interest in participatory technologies. Groups like Coldplay\footnote{\url{http://xylobands.com}} and Kasabian\footnote{\url{http://nanikawa.com/projects/kasabian-tour-2011-interactives}} have experimented in recent years with new ways to involve audiences in their performances. These technologies have been appearing at large outdoor music festivals and incorporated into performances streamed online where anyone with an internet connection can get involved.
% Participation is limited. How can we amplify audience input?

% Live music is a growing industry:
Lastly, it should be noted that the live music industry is growing rapidly (Wikstr\"{o}m, 2013). The digital revolution caused record sales to plummet, whereas live music revenues are larger than ever. While touring used to be a method for promoting recorded music, the opposite is now true. Wikstr\"{o}m speculates that ``live music will soon dominate the entire music industry in the same fashion as recorded music has done during more than half a century" (p. 142-143). Thus, researching new methods for creating impactful live music experiences is a worthy investment.

\subsection{Research Methods}
% Mention how your personal experience -- going to shows, putting on shows, stage managing, personal relationships with musicians -- shape your perspective

To address the research question, I implemented two forms of qualitative research -- ethnographic study and user-centred design.
% Talk about qualitative research; justify your methodology

% Ethnographic study:
Ethnography is the exploration of a cultural phenomenon. The phenomenon I am interested in is a rock concert. I aimed to find what this event means to both performers and audience members. I posted a questionnaire for music fans; many people go to concerts frequently, and I was able to acquire over one hundred responses. To investigate the perspective of a performer, I conducted one-hour interviews with three active musicians.
% Ethnography: Examining cultural phenomena within a group. Surveying modern music fans will provide me with a general sense of their feelings towards music, live performance, and technology. I will interview active musicians to understand how and why they interact with fans, on and off stage, and what they think of new technologies in a performance setting. These methods will allow me to gain a deeper understand the users and environments that are related to my work.

% User-centred design:
In user-centred design, the users' wants and needs drive the design. The success of a participatory technology is inherently dependent on satisfying multiple users simultaneously; without question, this research method was crucial. I developed three prototypes and tested them with groups of users. The last prototype was implemented at a real-world concert; I worked closely with a local band to develop the prototype, and audience reactions were closely observed.
% User-centred design: The users' wants and needs drive the design. Prototypes will be tested with users. Observations and interviews with the participants will help answer research questions as well as validate or disprove any assumptions that have been made. Both audience and performer perspectives will be represented. This method will help establish design guidelines.
% Methods:
% * Questionnaire and interviews (User-centred design; Ethnography). How do artists interact with fans now -- onstage and offstage? What do artists and fans think of interactive technologies? How do artists and fans want to interact at shows?
% * Prototyping and user testing (User-centred design)
% * Real-world testing (Action research, participative objective observation; Ethnography)
% Justification:
% * Why are these useful methods for my project?

\subsection{Scope}

Some delimitations will be enforced to ensure the research remains focused. In this context, rock music Western rock and roll music, but this could also include pop, hip hop, rap, R\&B, folk, and more. The requirement is that the music is performed in an environment where audience and performer roles are clearly defined, but the atmosphere is relaxed and audience members are comfortable participating. Results of this research may be applicable to other performance environments, but this is not discussed in this thesis. While the related work involved large-scale events, this thesis deals with small-sized audiences and venues. This is due to limited access to large bands and venues; however, it also reflects my personal experience, as I am most comfortable working with small bands in small venues. While many principles are mainly unaffected by audience or venue size, it is in some cases an important factor, and I note when this is so. Lastly, market viability of the prototypes is not discussed at this point.

% Turino, 2008: Identifying the frame of a given type of music is important to guide your interpretation
% Auslander, 1999: Rock vs Pop = Authentic vs Inauthentic
% Contrary to Auslander, I will be grouping them together and referring to both as "rock" music. This encompasses any music that is typically performed at live music venues or festivals with audiences that are typically comfortable moving and being vocal throughout the performance.
% Small, 1998:
% * The focus on classical music is odd. It's claimed to be intellectual, but it's not very popular.
% * Classical music performance turned audience members into spectators, consumers
% * Modern orchestra concerts emphasize listening, specialization, formal settings, ticketed admission. They celebrate separation of producer and consumer.
% I am interviewing independent artists who play in small venues, so this is the lens through with I will be viewing the problem
% Brown, 2009: % Crowd psychology rarely ``takes a practical orientation to how crowds might interact with technological artifacts or events?


\section{Overview}

\begin{itemize}
	\item \textbf{Chapter 2: Literature Review}
	
	An overview of the history of music and performance is provided, and presentational and participatory performances types are compared. I examine how presentational performance has grown into what it is today and how rock music and modern technologies can bring participation back into live music. A survey of related research includes work in group-controlled systems, creative collaboration, and audience-performer interaction. Real-world participatory technologies are also examined.
			
	\item \textbf{Chapter 3: Ethnography}
	
	The research questions are examined from both the audience's and the performer's point of view. This is accomplished through a survey of music fans and interviews with musicians.
	
	\item \textbf{Chapter 4: Protoyping}
	
	I describe the production of three prototypes. This includes the objectives of the prototypes, their development processes, user testing, and analyses of the results.
		
	\item \textbf{Chapter 5: Conclusion}
	
	To conclude, I summarize the overall outcomes of my work and discuss possible future directions for the project.
\end{itemize}
