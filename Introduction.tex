\chapter{Introduction}

Music is my passion; I love discovering new bands, sharing songs with friends, and going to live shows. In my mind, few things can compare to attending an amazing concert. Sound, lighting, and atmosphere coming together in a particular way can evoke a truly visceral reaction. A standard music venue has multi-coloured lights that can be controlled to set the mood of the performance; some artists, however, are not satisfied by this traditional system. Performance artists and electronic musicians have been experimenting with new technologies for years in an attempt to create a more unique and engaging stage show. Recently, many mainstream musicians have started to follow suit, incorporating clever technologies into their performances, leading to exciting experiences for their thousands of fans. Montreal-based studio Moment Factory, for instance, works with musicians like Madonna and Bon Jovi and employs special effects, mechatronics, and electronic devices to create incredible stage shows. For rock band Coldplay's most recent tour, the Xyloband - a wristband containing radio-controlled LEDs - was developed. Stadiums full of people wearing these devices lit up and blinked along with the music. It is becoming increasingly popular to enhance live events using novel technologies; however, these technologies generally lack something that I believe could greatly increase their effectiveness - interactivity.

At a typical concert, the audience passively watches musicians perform as lights flash on the stage. The show can be entertaining, but the separation between audience and performer may leave some feeling disconnected from the event. I want to combine my passion for music with my knowledge in electrical engineering and design to create a new technology that enhances the live music experience by allowing the audience to connect to the performance and actually influence it. My research will investigate the roles of concertgoers, how musicians and music fans feel about bringing technology into a concert setting, and how an interactive concert experience might best be designed. This will involve collecting information from members of the music community and user-testing the prototypes I develop.