\chapter{Introduction}

% Describe Arcade Fire's 2011 Coachella finale:
At the 2011 Coachella Valley Music and Arts Festival, Monteal-based indie rock group Arcade Fire are about to play one of the final songs of their headlining set. The guitar riff from the band's hit song ``Wake Up" is instantly recognized by the audience, who cheer loudly with excitement. The song reaches the first chorus, and, suddenly, one thousand white beach balls begin tumbling over the top of the stage and gently falling onto the crowd below. The cheers swell into a roar as the balls disperse over the mass of people. When the band hits the song's final chorus, to the spectators' surprise, the balls begin to light up -- flashing different colours to the beat of the music. Arcade Fire finish their set, grins on the band members' faces, as they watch the glowing orbs bounce across the crowd. After the show, festivalgoers grab on to the beach balls; cars leaving the festival grounds are seen glowing with the light from what have now become souvenirs from an unforgettable live music experience.

% Outline the goal of the thesis:
This project was made possible by several teams that managed the logistics, developed the wireless LED devices, fabricated and tested hundreds of beach balls, and ultimately executed the launch\footnote{http://www.momentfactory.com/en/project/stage/Arcade\_Fire}. The result was an awe-inspiring, albeit momentary, event that extended a live music performance into the audience. Large rock and pop concerts seem to be growing more technically complex and spectacular all the time. Powerful equipment makes shows louder, larger, and flashier. Only recently, however, have many mainstream artists begun investigating how technology can benefit not just the performance on stage, but the interactions within the audience as well. This thesis examines these sorts of technologies, asking how they might be used to make conventional rock and pop concerts more participatory.


% This ensures page 1 is not numbered:
%\pagenumbering{arabic}
%\setcounter{page}{2}


\section{Motivation}

% Introduce Turino's idea of presentational vs participatory performances:
In his 2008 book \textit{Music As Social Life}, musicologist and anthropologist Thomas Turino divides live music performances into two categories -- presentational and participatory. In presentational performances, the artist prepares music and presents it to a separate group, the audience. An example of a presentational performance would be a typical rock concert; a band rehearses and plans a set list and then performs it for a generally attentive audience. Participatory performances, on the other hand, deal only with participants and potential participants, and there is no artist-audience distinction. Peruvian communities, for example, perform in large groups with each participant either dancing or playing a panpipe or flute. Contra dances in the midwestern United States can also be considered participatory performances, featuring musicians, pairs of dancers, and a ``caller" that provides the dancers with instructions -- each an integral part of the event.

% Describe technologies that promote the presentational:
Many technologies exist for creating enhanced presentational performances. In general, they are implemented to aid the artist in presenting their music to the audience. When The Beatles did their first tour in North America, for example, they had 100-watt amplifiers custom made to ensure their music could be heard over the incredibly loud cheers of the fans; ultimately, the equipment was not nearly loud enough to overpower the audience (citation). Today, a large arena rock show might implement sound systems demanding tens or hundreds of thousands of watts (citation), enough power to send strong vibrations through concertgoers' bodies or even, after enough exposure, cause hearing damage. Performers may take advantage of enormous screens that provide far-away fans with close-up views of the show. Complex lighting rigs, laser arrays, and flashy visualizations are also common methods of turning a regular performance into an awe-inspiring spectacle. Recently, improved webcasting technologies have allowed for the live streaming of concerts over the Internet; a U2 concert streaming on YouTube in 2009 generated ten million pageviews, vastly increasing the reach of their performance\footnote{http://www.wired.com/business/2009/11/4-ways-live-and-digital-music-are-teaming-up-to-rock-your-world/}.

% Explain why we should focus more on the participatory:
There is clearly a great deal invested in enhancing the presentational aspect of live music performances in Western culture. As Turino points out, when it comes to presentational performances, profit making is usually the primary goal. Louder speakers and bigger screens mean artists can play larger venues with more seats to sell to fans. On the other hand, Turino admits that participatory performance does not fit well within capitalist societies: ``Participatory traditions tend to be relegated to special cultural cohorts that stand in opposition to the broader cultural formation" (p. 36). Why, then, might we want consider technologies that enhance the participatory aspect of performances? As stated by Turino, disregarding its potential financial value, it is music's function as a social interaction that holds the most value for humans. Levitin (2008) agrees, stating that the social nature of music may have been an important evolutionary adaptation that helped early humans thrive in groups: ``Singing around the ancient campfire might have been a way to stay awake, to ward off predators" (p. 258). Thus, while we may no longer depend on participatory performance to the degree that our ancestors did, it seems as though it is against human nature to continue widening the gap between audience and performer. Artists are placed on brightly lit stages with booming sound systems, allowing their voices to echo through stadiums; the voices of audience members, meanwhile, become meaningless noise as more and more people are packed into venues. How might technologies instead be used to embrace the social functions of music and let everyone -- performer and audience -- be a participant? This attitude is, fortunately, reflected in some of the ways we are using technology today.\\

% Notes:
* Audiences use techniques (clapping, singing along) and tools (lighters, signs) to participate in concerts already

% Describe how the Internet is facilitating participatory interactions:
The Internet has connected performer and audience in a new way. Social media allows for unique interactions between artists (big and small) and their fans. A small touring band, for example, might send out a message to their followers on Twitter asking for restaurant suggestions in a town they are passing through. More well-known artists may have difficulty connecting to their growing fan base, but events like ``Ask Me Anything" question-and-answer sessions on Reddit allow them to directly answer questions from their supporters. Beyond these social media interactions, the Internet is also continually supplying new ways for musicians and fans to connect. Crowdfunding platforms like Kickstarter rewards fans for directly funding artists' projects by giving them exclusive gifts; Feedbands\footnote{http://www.feedbands.com} lets users vote for their favourite musicians each month, and the winning artist gets their record pressed on vinyl in a limited-edition run; using Alive\footnote{http://www.alive.mu}, Japanese promoters can ensure shows get adequate turnouts by having concertgoers commit to buying tickets before artists are even booked to play. The Internet has afforded many new types of participatory experiences for musicians and fans. Recently, digital technologies are being used increasingly to enhance this connection during live performances themselves.\\

% Notes:
* Take some details away here and save them for the Background? Add in participatory research stuff?

% Describe how participatory technologies are starting to show up at concerts:
As the music industry continues attempting to find its footing in the digital age, ticket prices for concerts are steadily increasing -- seeing a 40\% increase from 2000 to 2008, for example\footnote{http://www.musicthinktank.com/blog/the-beatles-tell-us-that-weve-hit-the-concert-price-ceiling.html}. It seems then that, in order to ensure patrons are getting their money's worth, it is important for performers to deliver a truly unforgettable show. Stages, lighting, and visuals are certainly becoming more extravagant. Some artists, however, are looking to more innovative solutions to wow their audiences. Arcade Fire and Coachella, as described above, succeeded in creating a memorable live music experience using hundreds of wirelessly controlled LEDs. Wham City Lights created a similar experience by using a smartphone application to turn audience members' personal devices into a synchronized light show\footnote{http://whamcitylights.com/}. Other instances allow for more direct interactions. At a special performance by R\&B artist Usher, for example, fans could send tweets about the performance that would appear on the large screen onstage and then morph into abstract animations\footnote{http://www.momentfactory.com/en/project/stage/Amex\_Unstaged:\_Usher}. Plastikman, alter ego of Canadian DJ Richie Hawtin, released a smartphone app to accompany his 2010/2011 world tour; fans with the app could view a live video stream of the performer's perspective, reorganize audio samples that Hawtin would use in his performances, and participate in a live chat with other users during the show\footnote{http://hexler.net/software/synk}.

% Summarize why this is a topic worth investigating:
There are countless types of performance exhibited by the world's various cultures -- some more presentational, some more participatory. In Western cultures, presentational performances draw the biggest crowds, and technologies are being developed to make them larger, louder, and more lucrative. We must remember, however, that music is an inherently social activity; by continuing to separate performer and audience, we are potentially reducing its impact as an event that brings people together. Thus, it is valuable to consider how we might use technology to enhance not the presentational aspect of a concert, but the participatory. The popularity of the aforementioned internet services proves the public's desire for deeper interactions with artists, and recent experiments with interactive systems at live performances have produced impressive results. Lastly, participatory technologies at rock and pop music performances have yet to undergo an academic investigation. I believe, then, that it is worthwhile to survey this fledgling field and better understand its implications on the modern music performance.\\

% Notes:
* HCI papers have investigated this type of work from a design perspective, but none have deeply involved music fans and performers themselves to ensure the enjoyment (flow) of the event is preserved


\section{Objectives}

% State the goal of the thesis and the objectives that make up this goal:
The goal of this thesis is to explore how digital technologies can be used to make live music performances more participatory experiences. I have divided this task into three objectives. The first objective is to study relevant literature and perform case studies on existing projects. Literature reviews will pull from works in musicology -- to examine the nature of music as a social activity -- and human-computer interaction -- to understand how interactive systems can be designed for large groups of people. Precedent projects will be analyzed, observing what was done right and what issues are outstanding. After reflecting on what I have learned, my next objective is to identify questions that remain unanswered and conduct my own primary research that addresses them. This will involve surveys and interviews with frequent concertgoers and performers themselves. The final objective is to determine the criteria for a successful participatory concert technology through multiple rounds of prototyping and user testing. A final version will be implemented at a live music performance, and its effectiveness will be evaluated.

% Outline the scope and limitations of the project:
This work brings up some other topics that, while interesting, will not be addressed here. For example, while there exist a wealth of music genres and performance styles, I will be focusing on Western rock and pop concerts performed in standard venues. These environments are commonplace and can accommodate large audiences who are generally free to move around and be vocal; additionally, they are places that I feel comfortable and familiar with. Many of the projects I reference were dependant on a large budget, but, to avoid placing limitations on my own work, financial matters will not be considered. These sorts of issues are outside the scope of this project.\\

% Notes:
* Since I cannot realistically fill a venue with hundreds of users, viability of my project in large-scale scenarios can not be tested


\section{Overview}

% Outline each upcoming chapter in the document:
\begin{itemize}
	\item \textbf{Chapter 2: Background}
	
	To begin, I will examine related academic and professional work. I will examine music as a social activity, outline research done on designing technology for crowds, and describe some relevant projects that have been implemented at real-world events.
	
	\item \textbf{Chapter 3: Research Question}
	
	In this chapter, I will establish and justify a research question. I will then describe the approach I will be taking to answer this question.
	
	\item \textbf{Chapter 4: Preliminary Research}
	
	I will begin tackling my research question by examining the problem from both the audience's and the performer's point of view. This will be accomplished through surveys and interviews with concertgoers and musicians.
		
	\item \textbf{Chapter 5: Development}
	
	This chapter will cover my prototyping process in detail. For each phase of development, I will first state my objective. Next, I will describe the steps taken to achieve the objective. Lastly, I will present the results of user testing conducted with each prototype.
		
	\item \textbf{Chapter 6: Implementation.}
	
	Here I will describe the production of the final version of my device. The implementation of the device at a real-world event will be thoroughly examined, and I will reflect on its execution.
	
	\item \textbf{Chapter 7: Conclusion}
	
	To conclude, I will summarize the overall outcomes of my work and briefly discuss possible future directions for the project.
\end{itemize}