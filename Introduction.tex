\chapter{Introduction}

A performance is a relationship between performer and audience. Whether it is a music concert, a theatre piece, or a magic show, the performer's goal is to communicate a message. Their message could serve as a diversion from day-to-day life, or it could be a thought-provoking commentary on reality. It is always favourable, in any case, that it gets delivered as clearly as possible. Consequently, there have continually been new tools and techniques developed in order to increase the fidelity of the message. Modern sound systems can amplify a signal so it is heard for miles. Huge screens are installed to ensure all spectators have an up-close view of the performer. Venues are precisely designed to maximize audience comfort and acoustic quality. In some cases, however, it is the audience members that wish to be heard more clearly; it is the audience members that uses tools and techniques to make their messages heard. At rock and pop concerts, in addition to the customary applause and cheers, it is common to see concertgoers waving illuminated lighters and mobile phones or even holding homemade signs in the air. In a performance, the audience is as important as the performer. It is useful, then, to put more consideration into how we might deliver clearer messages from audience to performer.

Today, musicians are more connected to their fans than ever. By leveraging services like Facebook, Twitter, and Topspin, artists can keep their followers up to date on new releases, tour dates, and other special events. Perhaps more significant is the unprecedented level of access that fans have to the musicians. Fans are talking directly to the artists, voicing their opinions on their work, providing travel tips during tours, and sharing original content inspired by them. If it is now so easy to communicate with performers via the Internet, why, then, is the communication channel at a concert still so one sided? Waving a lighter does not convey much information, and it is rare that homemade signs are actually read by the performer. Some artists have tried to develop solutions, but these have their flaws as well. Giving audience members wristbands embedded with blinking LEDs, for example, is an impressive way to expand the light show into the crowd, but it is not actually interactive, as the devices do not transmit any information from the users. Displaying audience members' Twitter posts during a show may cause some excitement, but it is also inviting them to look down at their phones, and it is not a coherent output in the context of a music performance. I believe that modern music performances can benefit from a new, reliable method of audience-to-performer communication.

% This ensures page 1 is not numbered:
%\pagenumbering{arabic}
%\setcounter{page}{2}

The goal of this thesis project is to find the most effective communication system that will enhance a live music performance for concertgoers. This requires considering some important research questions. First of all, how might we obtain meaningful information from the audience without being a distraction to them? What information do concertgoers find meaningful? On the other hand, I am interested in what data performers might find useful. What sort of feedback could they provide that would enhance the performance for the audience? This project has a strong focus on interacting with technology. Thus, my theoretical framework will be greatly based in human-computer interaction research, ensuring that the final product is intuitive to use. With the project centred around traditional rock and pop music concerts, I am also interested in how the mind interprets this sort of music and environment. Referencing work in music psychology will help me understand why people attend these concerts and what characteristics make one performance more captivating than another. This work brings up some other topics that, while interesting, will not be addressed here. For example, while inspiration will surely be taken from other genres of music and experimental performances, the final product will be created for a standard western pop music concert. The commercial viability of the final product -- matters of distribution, pricing, etc. -- will generally not be considered.  These sorts of issues are outside the scope of this project.

To begin, I will examine related academic and professional work. Next, the methods and results of my primary research will be discussed. My prototyping process will then be explained, followed by an analysis of the results from the user testing. Last is a discussion on the outcome of the project and possible future directions.