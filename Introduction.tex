\chapter{Introduction}

% Describe Arcade Fire's 2011 Coachella finale:
At the 2011 Coachella Valley Music and Arts Festival, Monteal-based indie rock group Arcade Fire are about to play one of the final songs of their headlining set. The guitar riff from the band's hit song ``Wake Up" is instantly recognized by the audience, who cheer loudly with excitement. The song reaches the first chorus, and, suddenly, one thousand white beach balls begin tumbling over the top of the stage and gently fall onto the crowd below. The cheers grow into a roar as the balls disperse over the mass of people. When the band hits the song's final chorus, to the audience's surprise, the balls begin to light up -- flashing different colours to the beat of the music. Arcade Fire finish their set, grins on the band members' faces, as they watch the glowing orbs bounce across the crowd. After the show, festivalgoers hold on to the beach balls; vehicles leaving the festival grounds are seen glowing with the light from what have now become souvenirs from an unforgettable live music experience.

% Outline the goal of the thesis:
This project was made possible by several teams that managed the logistics, developed the wireless LED devices, fabricated and tested hundreds of beach balls, and ultimately executed the launch\footnote{\url{http://www.momentfactory.com/en/project/stage/Arcade_Fire}}. The result was an awe-inspiring, albeit momentary, event that allowed the audience to participate in the performance. Rock concerts are growing more technically complex and spectacular all the time. Powerful equipment makes shows louder, larger, and flashier. Only recently, however, have artists and researchers begun investigating how technology can amplify not just the performer's actions, but those of the audience members as well. This thesis examines these sorts of technologies, asking how they might be used to make conventional rock concerts more participatory.

% The following ensures page 1 is not numbered; be aware of changes in ToC
\pagenumbering{arabic}
\setcounter{page}{2}

\section{Motivation}

% Introduce Turino's idea of presentational vs participatory performances:
In his 2008 book \textit{Music As Social Life}, musicologist and anthropologist Thomas Turino divides live music performances into two categories -- presentational and participatory. In presentational performances, the artist prepares music and presents it to a separate group, the audience. A performance by an orchestra might be considered purely presentational, for instance; musicians on an elevated stage perform rehearsed works to a silent and attentive crowd. Participatory performances, on the other hand, involve only participants and potential participants, and there is no artist-audience distinction. Peruvian communities, for example, perform in large groups with each participant dancing or playing a wind instrument.

% Evolution of music and performance:
Researchers have found evidence of musical activity from every known era of humanity (Levitin, 2006). While music has served various function for different cultures -- used to remove curses (Turner, 2011) or even settle lawsuits (Jourdain, 1997) -- it is most widely regarded as a relationship-forming social activity (Levitin, 2006; Turner, 2011). It is only in the past few centuries that presentational performances have become ordinary public events. These grew in popularity when entrepreneurial musicians began taking advantage of the emerging middle class, members of which were eager to display their newfound wealth (Small, 1998). Eventually, public performances established a strict divide between performer and audience; musicians dressed alike and performed composed and rehearsed pieces while audience members politely listened.

% Today's concerts are visual:
While today's popular music concerts are more relaxed, the divide between performer and audience is still prominent. Furthermore, performances today are in some ways more presentational than ever with a variety of additional stimuli being blasted at audiences. Giant screens display dynamic visuals and extreme closeups of performers, and heavy-duty mechatronics are incorporated to continually change the shape of the stage. Auslander (1999) suggests that this focus on the visual is a response to the visual culture established by mass media -- and by the music video in particular. Many concertgoers might be satisfied by an entirely lip-synced performance if it is enough of a spectacle for the eyes.

% Rock shows are participatory:
Despite this focus on the presentational, audience involvement has managed to survive as a relatively minor participatory element in rock concerts. Audience members become a part of the performance by singing or clapping along, holding their illuminated lighters or cellphones above their heads, or simply moving to the music. Some musicians take song requests, and some even invite audience members on stage alongside them. In rock music, the barrier between the audience and the performer is often crossed.

% Participatory culture:
Just as concerts became more visual in response to an increasingly visual culture, I believe that they are now becoming more participatory in response to an emerging ``participatory culture" (Wikstr\"{o}m, 2013). The unprecedented connectivity afforded by the Internet and digital technologies is bringing artists and their followers closer than ever. Musicians can talk to fans directly on social media websites and receive instant feedback on their work. Fans can remix their favourite artists' songs or create their own music videos to accompany them. Rather than flowing from performer to audience, information can now move freely between the two parties.

% Artists and researchers are embracing participatory technologies:
This level of connectivity is beginning to be embraced offline and even during performances. Popular artists like Coldplay, Usher, and Arcade Fire have experimented with new ways of using technology to make their audiences an integral part of their concerts. The modern technologies that make this possible are also piquing the interests of human-computer interaction researchers, who have recently started investigating how crowd-based interfaces for interactive systems might be best designed. Now, more than ever, those who were once only spectators might have the ability to become meaningful participants.
% By taking advantage of this participatory culture and the technology that produced it, we might begin seeing more performances in which music can better function as the communal activity it has always been.


\section{Purpose}

% Research questions:
This thesis investigates how rock performances might be made more participatory using digital technology. The research questions take the form of a design problem, asking how each component of a participatory system might be conceived such that both audience members and performers are satisfied and the overall live music experience is enhanced. What type of output do audience members want to influence? How might a system capture inputs from a large group of users in the context of a rock concert? What sort of feedback can simply but effectively inform users that they are in fact making contributions to the performance? How involved do rock musicians actually want their audiences to be? 
% Questions:
% * How might technology be used to add participatory elements to traditionally presentational rock concerts?
% * How do audiences want to participate? How do performers want audiences to participate?
% * Input: How can data best be captured from the audience?
% * Output: What can the audience interact with that will enhance the performance but not distract the audience or performer?
% * Feedback: What kind of feedback will be informative/useful without being distracting?
% * What's the threshold where a performance can no longer be social? What's the ratio where participation can be sustained?

% Research methods:
To address these questions, multiple participatory technologies were created through a user-centred design process. An initial exploratory phase took the form of an ethnographic study. In order to understanding the cultural phenomena surrounding rock performances, both music fans and artists were surveyed. Using an online questionnaire, frequent concertgoers were asked about their feelings towards participating in performances. Next, thorough semi-structured interviews with three experienced musicians provided me with performers' perspectives on audience-performer interaction and the use of new technologies at concerts. The results of these studies then informed the conceptualization of three prototypes -- three different interactive systems that were tested with users. Providing especially significant data was the final prototype, which was created in collaboration with a local band and tested at a real performance. By observing and interviewing both the audience members and performers, I was able to begin forming responses to my research questions.


\section{Scope and Limitations}

The breadth of the research completed was regulated in the interest of practicality. Only rock music performances, for instance, are considered herein. In the context of this work, `rock music' refers primarily to Western rock and roll music, but it could easily also include the pop, punk, hip hop, and folk genres. The fundamental requirement was that the music is performed in an environment where audience and performer roles are clearly defined but where audience members feel comfortable being active participants. This thesis does not, for example, consider implementing participatory technologies at classical music recitals. Financial matters were also generally disregarded. The market viability of the prototypes are not discussed in this work.
% Scope: How much of the topic the thesis covers

There are uncontrollable factors that limited the reach of this thesis as well. While much of the existing work that is referenced involves large-scale events, this research could only realistically deal with relatively small audiences and venues. Additionally, due to geographical limitations, user testing was only completed in the city of Toronto; thus, perspectives from other communities are not represented in this work.
% Limitations: Potential areas where the work may fall short

% Turino, 2008: Identifying the frame of a given type of music is important to guide your interpretation
% Auslander, 1999: Rock vs Pop = Authentic vs Inauthentic
% Contrary to Auslander, I will be grouping them together and referring to both as "rock" music. This encompasses any music that is typically performed at live music venues or festivals with audiences that are typically comfortable moving and being vocal throughout the performance.
% Small, 1998:
% * The focus on classical music is odd. It's claimed to be intellectual, but it's not very popular.
% * Classical music performance turned audience members into spectators, consumers
% * Modern orchestra concerts emphasize listening, specialization, formal settings, ticketed admission. They celebrate separation of producer and consumer.
% I am interviewing independent artists who play in small venues, so this is the lens through with I will be viewing the problem
% Brown, 2009: % Crowd psychology rarely ``takes a practical orientation to how crowds might interact with technological artifacts or events"


\section{Overview}

The remainder of the document is organized as follows:

\subsubsection{Chapter 2: Literature Review}
	
An overview of the history of music and performance is provided, and presentational and participatory performances are contrasted. I examine how presentational performance has grown into what it is today and how rock music and modern technologies are bringing participation back into live music. A survey of related research includes work in crowd-based interfaces and audience-performer interaction. Lastly, I present case studies of existing participatory technologies.
	
\subsubsection{Chapter 3: Research Approach}
	
The primary and secondary research questions are listed and justified. I describe the research methods that were implemented and explain why they are appropriate tools.
			
\subsubsection{Chapter 4: Ethnography}

An ethnographic study was conducted, including an online questionnaire for concertgoers and semi-structured interviews with musicians. The results of these surveys are presented and analyzed.
	
\subsubsection{Chapter 5: Prototyping}

I describe the production of three interactive systems that were developed in order to address the research questions. Each prototype was tested with users, and the findings were analyzed. The final prototype was created in collaboration with a local band and tested at a live performance.
		
\subsubsection{Chapter 6: Conclusion}
	
To conclude, I summarize the overall outcomes of my research and discuss possible future directions.