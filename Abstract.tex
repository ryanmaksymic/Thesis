\begin{center}
{Using digital technology to enable new forms of\\audience participation during rock music performances\\[0.25cm]
Ryan Maksymic\\[0.25cm]
Master of Design in Digital Futures\\[0.25cm]
OCAD University, 2014\\[0.75cm]
\large\textbf{Abstract}}\\
\end{center}

Technology has long been used to improve the presentational aspects of a live music performance, but it is less often employed to encourage participation from audience members. This thesis investigates how digital technologies might be used to make rock concerts more participatory. An ethnographic study was first carried out, surveying concertgoers and musicians to identify current attitudes towards audience participation and technology-enabled events. Prototypes were developed and tested to assess new methods for facilitating audience involvement. The final prototype was created in collaboration with a local band and implemented during a live performance; both the audience and performers found the experience enjoyable. Upon analysis, several characteristics for a successful participatory technology were identified. Limiting constraints on user inputs and promoting true interactivity between audience and performer yielded the best results. It was concluded that digital technology provides new opportunities for audiences to participate in music performances.

% Motivation: Why do we care?
% Problem statement: What problem are we trying to solve? What is the scope of your work? Don't use jargon.
% Approach: How did you go about addressing the problem? What was the extent of your work? What variables did you control, ignore, measure?
% Results: What's the answer? Be as specific as possible.
% Conclusions: What are the implications of the results? How impactful could they be? Are these results specific or generalizable?