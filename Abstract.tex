\begin{center}
{Using digital technology to enable new forms of\\audience participation during rock music performances\\[0.25cm]
Ryan Maksymic\\[0.25cm]
Master of Design in Digital Futures\\[0.25cm]
OCAD University, 2014\\[0.75cm]
\large\textbf{Abstract}}\\
\end{center}

Technology has long been used to improve the presentational aspects of a live music performance, but it is less often employed to encourage participation from audience members. This thesis investigates how digital technologies might be used to make traditional pop and rock concerts more participatory. An ethnographic study was first carried out, surveying concertgoers and conducting interviews with experienced musicians to identify current attitudes towards audience participation and technology-enabled events. Prototypes were developed and tested to investigate new methods for facilitating audience involvement during a performance. The final prototype was developed in collaboration with a local band and tested with twelve audience members during a twenty-minute performance. The users and performers found the experience satisfying but also indicated that the novelty of the interaction may not last through full-length performances. Future iterations will aim to include a larger number of users, and it will be investigated how the system can be made dynamic to allow for multiple modes of participation throughout a single performance. It was concluded that digital technology provides new opportunities for audiences to participate in music performances.
% Rethink what your major findings are

% Motivation: Why do we care?
% Problem statement: What problem are we trying to solve? What is the scope of your work? Don't use jargon.
% Approach: How did you go about addressing the problem? What was the extent of your work? What variables did you control, ignore, measure?
% Results: What's the answer? Be as specific as possible.
% Conclusions: What are the implications of the results? How impactful could they be? Are these results specific or generalizable?